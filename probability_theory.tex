\documentclass[dvipdfmx]{jsarticle}
\usepackage{amsfonts}
\usepackage{amsmath}
\usepackage{here}
\usepackage{mathrsfs}
\usepackage{tikz}
\usetikzlibrary{intersections, calc, arrows.meta}
\makeatletter
\newcommand{\subsubsubsection}{\@startsection{paragraph}{4}{\z@}%
 {1.0\Cvs \@plus.5\Cdp \@minus.2\Cdp}%
 {.1\Cvs \@plus.3\Cdp}%
 {\reset@font\sffamily\normalsize}
}
\makeatother
\setcounter{secnumdepth}{4}
\title{確率論}
\author{伊藤 太清}
\date{\today}
\begin{document}
 \maketitle
 \tableofcontents
 \section{離散確率変数}
$N$通りの場合$S=\{0,\cdots,N-1\}$の中からそれぞれの確率$p\left(0\right),\cdots,p\left(N-1\right)$で定まる変数がある.
 \begin{itemize}
  \item 全ての場合の集合$S=\{0,\cdots,N-1\}$を標本空間という.
  \item $p:S\to\left[0,1\right]$を確率質量関数という.
  \item $P:S\to\left[0,1\right];a\mapsto\sum_{x\in S|x<a}p\left(x\right)$を累積分布関数という.
 \end{itemize}
累積分布関数$P$が
 \begin{align}
  P\left(N\right)=1
 \end{align}
を満たすとき,$X=\left(S,P\right)$を離散確率変数という.
累積分布関数$P$が確率質量関数$p$を用いて
 \begin{align}
  P\left(a\right)=\sum_{x\in S|x<a}p\left(x\right)
 \end{align}
と表されるのに対し,確率質量関数$p$は累積分布関数$P$を用いて
 \begin{align}
  p\left(a\right)&=\left(\sum_{x\in S|x<a+1}p\left(x\right)\right)-\left(\sum_{x\in S|x<a}p\left(x\right)\right)\nonumber\\
  &=P\left(a+1\right)-P\left(a\right)
 \end{align}
と表される.
 \subsection{確率}
離散確率変数$X$から取り出された値$a$が,$S$の部分集合$S'$の元である確率は,事象からその事象の確率への関数である確率関数$\mathscr{P}$を用いて,
 \begin{align}
  \mathscr{P}\left(a\in S'\right)&=\sum_{x\in S'}p\left(x\right)
 \end{align}
である.
また,累積分布関数$P$は,確率関数$\mathscr{P}$を用いて,
 \begin{align}
  P\left(x\right)&=\mathscr{P}\left(a<x\right)
 \end{align}
と表される.
 \subsection{平均}
 \begin{align}
  E\left(X\right)=\sum_{x\in S}xp\left(x\right)
 \end{align}
を,離散確率変数$X$の平均という.
 \subsection{分散}
 \begin{align}
  V\left(X\right)=\sum_{x\in S}\left(x-E\left(X\right)\right)^2p\left(x\right)
 \end{align}
を,離散確率変数$X$の分散という.
 \subsection{関数の適用}
関数$f:S\to T$を離散確率変数$X=\left(S,P\right)$に適用した離散確率変数を$f\left(X\right)=\left(T,Q\right)$とし,その確率質量関数を$q$とすると,累積分布関数$Q$は,
 \begin{align}
  Q\left(b\right)&=\mathscr{P}\left(y<b\right)\nonumber\\
  &=\mathscr{P}\left(f\left(x\right)<b\right)\nonumber\\
  &=\sum_{x\in S|f\left(x\right)<b}p\left(x\right)
 \end{align}
であり,確率質量関数$q$は,
 \begin{align}
  q\left(b\right)&=Q\left(b+1\right)-Q\left(b\right)\nonumber\\
  &=\left(\sum_{x\in S|f\left(x\right)<b+1}p\left(x\right)\right)-\left(\sum_{x\in S|f\left(x\right)<b}p\left(x\right)\right)\nonumber\\
  &=\sum_{x\in S|f\left(x\right)=b}p\left(x\right)
 \end{align}
となる.
 \subsubsection{平均}
離散確率変数$f\left(X\right)$の平均は,
 \begin{align}
  E\left(f\left(X\right)\right)&=\sum_{y\in T}yq\left(y\right)\nonumber\\
  &=\sum_{y\in T}y\sum_{x\in S|f\left(x\right)=y}p\left(x\right)\nonumber\\
  &=\sum_{y\in T}\sum_{x\in S|f\left(x\right)=y}yp\left(x\right)\nonumber\\
  &=\sum_{y\in T}\sum_{x\in S|f\left(x\right)=y}f\left(x\right)p\left(x\right)\nonumber\\
  &=\sum_{x\in S|f\left(x\right)\in T}f\left(x\right)p\left(x\right)\nonumber\\
  &=\sum_{x\in S}f\left(x\right)p\left(x\right)\label{DiscreteFunctionAverage}
 \end{align}
となる.
この議論より分散は,
 \begin{align}
  V\left(X\right)=E\left(\left(X-E\left(X\right)\right)^2\right)
 \end{align}
と書けることが分かる.
 \subsubsection{分散}
離散確率変数$f\left(X\right)$の分散は,
 \begin{align}
  V\left(f\left(X\right)\right)&=\sum_{y\in T}\left(y-E\left(f\left(X\right)\right)\right)^2q\left(y\right)\nonumber\\
  &=\sum_{y\in T}\left(y-E\left(f\left(X\right)\right)\right)^2\sum_{x\in S|f\left(x\right)=y}p\left(x\right)\nonumber\\
  &=\sum_{y\in T}\sum_{x\in S|f\left(x\right)=y}\left(y-E\left(f\left(X\right)\right)\right)^2p\left(x\right)\nonumber\\
  &=\sum_{y\in T}\sum_{x\in S|f\left(x\right)=y}\left(f\left(x\right)-E\left(f\left(X\right)\right)\right)^2p\left(x\right)\nonumber\\
  &=\sum_{x\in S|f\left(x\right)\in T}\left(f\left(x\right)-E\left(f\left(X\right)\right)\right)^2p\left(x\right)\nonumber\\
  &=\sum_{x\in S}\left(f\left(x\right)-E\left(f\left(X\right)\right)\right)^2p\left(x\right)\nonumber\\
  &=E\left(\left(f\left(X\right)-E\left(f\left(X\right)\right)\right)^2\right)\nonumber\\
  &=E\left(\left(Y-E\left(Y\right)\right)^2\right)
 \end{align}
となる.
 \section{連続確率変数}
実数全体の集合$\mathbb{R}$の中から確率的に実数が選択される変数がある.
 \begin{itemize}
  \item 選択されうる実数の集合$S=\mathbb{R}$を標本空間という.
  \item $p:S\to\left[0,\infty\right)$を確率密度関数という.
  \item $P:S\to\left[0,1\right];a\mapsto\int_{-\infty}^ap\left(x\right)dx$を累積分布関数という.
 \end{itemize}
累積分布関数$P$が
 \begin{align}
  \lim_{a\to\infty}P\left(a\right)=1
 \end{align}
を満たすとき,$X=\left(S,P\right)$を連続確率変数という.
累積分布関数$P$が確率密度関数$p$を用いて
 \begin{align}
  P\left(a\right)=\int_{-\infty}^ap\left(x\right)dx
 \end{align}
と表されるのに対し,確率密度関数$p$は累積分布関数$P$を用いて,
 \begin{align}
  p\left(a\right)=P'\left(a\right)
 \end{align}
と表される.
 \subsection{確率}
連続確率変数$X$から取り出された値$a$が,$S$の部分集合$S'$の元である確率は,事象からその事象の確率への関数である確率関数$\mathscr{P}$を用いて,
 \begin{align}
  \mathscr{P}\left(a\in S'\right)&=\int_{x\in S'}p\left(x\right)dx
 \end{align}
である.また,累積分布関数$P$は,確率関数$\mathscr{P}$を用いて,
 \begin{align}
  P\left(x\right)&=\mathscr{P}\left(a<x\right)
 \end{align}
である.
 \subsection{平均}
 \begin{align}
  E\left(X\right)=\int_{-\infty}^\infty xp\left(x\right)dx
 \end{align}
を,連続確率変数$X$の平均という.
 \subsection{分散}
 \begin{align}
  V\left(X\right)=\int_{-\infty}^\infty \left(x-E\left(X\right)\right)^2p\left(x\right)dx
 \end{align}
を,連続確率変数$X$の分散という.
 \subsection{関数の適用}
微分可能で$f'\left(x\right)=0$の解が高々有限個である関数$f:S\to T$を連続確率変数$X=\left(S,P\right)$に適用した連続確率変数を$f\left(X\right)=\left(T,Q\right)$とし,その確率密度関数を$q$とすると,$f'\left(b\right)=0$を満たすような$b$の範囲内において,累積分布関数$Q$は,
 \begin{align}
  Q\left(b\right)&=\mathscr{P}\left(y<b\right)\nonumber\\
  &=\mathscr{P}\left(f\left(x\right)<b\right)\nonumber\\
  &=\int_{x\in S|f\left(x\right)<b}p\left(x\right)dx
 \end{align}
であり,確率密度関数$q$は,
 \begin{align}
  q\left(b\right)&=Q'\left(b\right)\nonumber\\
  &=\frac{d}{db}\int_{x\in S|f\left(x\right)<b}p\left(x\right)dx
 \end{align}
となる.ここで,上の微分を図を用いて解く.
 \begin{figure}[H]
  \begin{center}
   \begin{tikzpicture}
    \draw(0,0)node[below left]{$O$};
    \draw[->](-0.5,0)--(8,0)node[right]{$x$};
    \draw[->](0,-0.5)--(0,4)node[above]{$y$};
    \draw[domain=-2.5:2.5]plot({\x+4}, {(\x^3-4*\x)/3+2})node[right]{$y=f\left(x\right)$};
    \draw[dashed](2,2)--(2,0)node[below]{$a_0$};
    \draw[dashed](2.2236,2.5)--(2.2236,-0.5)node[below]{$a_0+\Delta a_0$};
    \draw[dashed](4,2)--(4,0)node[below]{$a_1$};
    \draw[dashed](3.61019,2.5)--(3.61019,-1)node[below]{$a_1+\Delta a_1$};
    \draw[dashed](6,2)--(6,0)node[below]{$a_2$};
    \draw[dashed](6.16621,2.5)--(6.16621,-0.5)node[below]{$a_2+\Delta a_2$};
    \draw[dashed](6.5,2)--(-0.5,2)node[left]{$b$};
    \draw[dashed](6.5,2.5)--(-0.5,2.5)node[left]{$b+\Delta b$};
   \end{tikzpicture}
   \caption{$b$の微小な変化による$\{x\in S|f\left(x\right)<b\}$の変化}\label{IntegrationRangeChange}
  \end{center}
 \end{figure}
図\ref{IntegrationRangeChange}のように,$f\left(x\right)=b$の全ての解を$\{a_0,\cdots,a_N\}=\{x\in S|f\left(x\right)=b\}$とし,$b$の微小な変化$\Delta b$に対する$a_0,\cdots,a_N$の反応をそれぞれ$\Delta a_0,\cdots,\Delta a_N$とする.
ここで,$\Delta b$は十分に小さいから,$\forall n\in\{0,\cdots,N\}$に対して,
 \begin{align}
  f'\left(a_n\right)&=\frac{\Delta b}{\Delta a_n}\nonumber\\
  \Delta a_n&=\frac{\Delta b}{f'\left(a_n\right)}
 \end{align}
である.
ただしここで$f'\left(a_n\right)=0$の場合,それは$a_n$が$f$の極大若しくは極小であることにより,$b$の微小な変化によって$a_n$が消滅または2つに分裂することを意味し,$\Delta a_n$は定義できない.
 \begin{figure}[H]
  \begin{center}
   \begin{tikzpicture}
    \draw(0,0)node[below left]{$O$};
    \draw[->](-0.5,0)--(8,0)node[right]{$x$};
    \draw[->](0,-3)--(0,3)node[above]{$z$};
    \draw[domain=1.5:6.5]plot({\x}, {5*(1/(pi*(1+(\x-4)^2)))})node[right]{$z=p\left(x\right)$};
    \draw[dashed](2,{5*(1/(pi*(1+(2-4)^2)))})--(2,0)node[below]{$a_0$};
    \draw[dashed](2.2236,{5*(1/(pi*(1+(2.2236-4)^2)))})--(2.2236,-0.5)node[below]{$a_0+\Delta a_0$};
    \draw({(2+2.2236)/2},{5*(1/(pi*(1+(2-4)^2)))})node[above]{$\Delta s_0$};
    \draw[dashed](4,{5*(1/(pi*(1+(4-4)^2)))})--(4,0)node[below]{$a_1$};
    \draw[dashed](3.61019,{5*(1/(pi*(1+(3.61019-4)^2)))})--(3.61019,-1)node[below]{$a_1+\Delta a_1$};
    \draw({(4+3.61019)/2},{5*(1/(pi*(1+(4-4)^2)))})node[above]{$\Delta s_1$};
    \draw[dashed](6,{5*(1/(pi*(1+(6-4)^2)))})--(6,0)node[below]{$a_2$};
    \draw[dashed](6.16621,{5*(1/(pi*(1+(6.16621-4)^2)))})--(6.16621,-0.5)node[below]{$a_2+\Delta a_2$};
    \draw({(6+6.16621)/2},{5*(1/(pi*(1+(6-4)^2)))})node[above]{$\Delta s_2$};
   \end{tikzpicture}
   \caption{$b$の微小な変化による$Q\left(b\right)$の変化}\label{QChange}
  \end{center}
 \end{figure}
図\ref{QChange}において,領域$\{(x,y)|a_n\le x\le a_n+\Delta a_n\land \left(0-y\right)\left(p\left(x\right)-y\right)\le0\}$を台形とみなすと,その面積$\Delta s_n$は,
 \begin{align}
  \Delta s_n&=\frac{1}{2}\Delta a_n\left(p\left(a_n\right)+p\left(a_n+\Delta a_n\right)\right)\nonumber\\
  &=\frac{\Delta b}{2f'\left(a_n\right)}\left(p\left(a_n\right)+p\left(a_n+\frac{\Delta b}{f'\left(a_n\right)}\right)\right)
 \end{align}
となり,これを$\Delta b$で割ると,
 \begin{align}
  \frac{\Delta s_n}{\Delta b}&=\frac{p\left(a_n\right)+p\left(a_n+\frac{\Delta b}{f'\left(a_n\right)}\right)}{2f'\left(a_n\right)}\nonumber\\
  &=\frac{p\left(a_n\right)+p\left(a_n+\frac{0}{f'\left(a_n\right)}\right)}{2f'\left(a_n\right)}\nonumber\\
  &=\frac{p\left(a_n\right)+p\left(a_n\right)}{2f'\left(a_n\right)}\nonumber\\
  &=\frac{2p\left(a_n\right)}{2f'\left(a_n\right)}\nonumber\\
  &=\frac{p\left(a_n\right)}{f'\left(a_n\right)}
 \end{align}
となる.$q\left(b\right)$は,これを$\forall a_n\in\{a_0,\cdots,a_N\}=\{x\in S|f\left(x\right)=b\}$について足し合わせたものであるから,
 \begin{align}
  q\left(b\right)&=Q'\left(b\right)\nonumber\\
  &=\frac{d}{db}\int_{x\in S|f\left(x\right)<b}p\left(x\right)dx\nonumber\\
  &=\sum_{x\in S|f\left(x\right)=b}\frac{p\left(x\right)}{f'\left(x\right)}
 \end{align}
となる.
 \subsubsection{平均}
連続確率変数$f\left(X\right)$の平均は
 \begin{align}
  E\left(f\left(X\right)\right)&=\int_{-\infty}^\infty yq\left(y\right)dy\nonumber\\
  &=\int_{-\infty}^\infty y\sum_{x\in S|f\left(x\right)=y}\frac{p\left(x\right)}{f'\left(x\right)}dy\nonumber\\
  &=\int_{-\infty}^\infty\sum_{x\in S|f\left(x\right)=y}\frac{yp\left(x\right)}{f'\left(x\right)}dy\nonumber\\
  &=\int_{-\infty}^\infty\sum_{x\in S|f\left(x\right)=y}\frac{f\left(x\right)p\left(x\right)}{f'\left(x\right)}dy\nonumber\\
  &=\int_{-\infty}^\infty\frac{f\left(x\right)p\left(x\right)}{f'\left(x\right)}\frac{dy}{dx}dx\nonumber\\
  &=\int_{-\infty}^\infty\frac{f\left(x\right)p\left(x\right)}{f'\left(x\right)}f'\left(x\right)dx\nonumber\\
  &=\int_{-\infty}^\infty f\left(x\right)p\left(x\right)dx\label{ContinuousFunctionAverate}
 \end{align}
となる.
ここで,$f'\left(x\right)=0$となる$x$は高々有限個しかないため,この積分の結果には影響しない.
この議論より分散は,
 \begin{align}
  V\left(X\right)=E\left(\left(X-E\left(X\right)\right)^2\right)
 \end{align}
と書けることが分かる.
これは離散確率変数及び連続確率変数で共通して使用できる分散の定義となる.
この分散の定義式の直感的理解について述べる.
確率変数$X$から平均$E\left(X\right)$を引いた確率変数$X-E\left(X\right)$の平均は$0$で,分布は平行移動するものの$X$と同様の広がり具合を持っている.
これをさらに2乗した確率変数$\left(X-E\left(X\right)\right)^2$は非負となる.
分布は変形するものの,その広がり具合は2乗する前の確率変数$X-E\left(X\right)$と連動しており,どちらかが広がったり狭まったりすればもう一方も同様に広がったり狭まったりする.
下限値$0$が設定された上で分布の広がり具合も連動しているので,その平均$E\left(\left(X-E\left(X\right)\right)^2\right)$も同様に$X$の分布の広がり具合と連動していると言える.
 \subsubsection{分散}
連続確率変数$f\left(X\right)$の分散は
 \begin{align}
  V\left(f\left(X\right)\right)&=\int_{-\infty}^\infty\left(y-E\left(f\left(X\right)\right)\right)^2q\left(y\right)dy\nonumber\\
  &=\int_{-\infty}^\infty\left(y-E\left(f\left(X\right)\right)\right)^2\sum_{x\in S|f\left(x\right)=y}\frac{p\left(x\right)}{f'\left(x\right)}dy\nonumber\\
  &=\int_{-\infty}^\infty\sum_{x\in S|f\left(x\right)=y}\left(y-E\left(f\left(X\right)\right)\right)^2\frac{p\left(x\right)}{f'\left(x\right)}dy\nonumber\\
  &=\int_{-\infty}^\infty\sum_{x\in S|f\left(x\right)=y}\left(f\left(x\right)-E\left(f\left(X\right)\right)\right)^2\frac{p\left(x\right)}{f'\left(x\right)}dy\nonumber\\
  &=\int_{-\infty}^\infty\left(f\left(x\right)-E\left(f\left(X\right)\right)\right)^2\frac{p\left(x\right)}{f'\left(x\right)}\frac{dy}{dx}dx\nonumber\\
  &=\int_{-\infty}^\infty\left(f\left(x\right)-E\left(f\left(X\right)\right)\right)^2\frac{p\left(x\right)}{f'\left(x\right)}f'\left(x\right)dx\nonumber\\
  &=\int_{-\infty}^\infty\left(f\left(x\right)-E\left(f\left(X\right)\right)\right)^2p\left(x\right)dx\nonumber\\
  &=E\left(\left(f\left(X\right)-E\left(f\left(X\right)\right)\right)^2\right)\nonumber\\
  &=E\left(\left(Y-E\left(Y\right)\right)^2\right)
 \end{align}
となる.
ここで,$f'\left(x\right)=0$となる$x$は高々有限個しかないため,この積分の結果には影響しない.
 \section{平均の線形性}
 \subsection{和の平均は平均の和}
ある2つの確率変数$X$と$Y$があるとき,
 \begin{align}
  E\left(X+Y\right)&=E\left(X\right)+E\left(Y\right)
 \end{align}
である.
 \subsubsection{証明}
 \subsubsubsection{離散確率変数同士の和の平均}
互いに独立とは限らない2つの離散確率変数$X$と$Y$があり,それぞれの標本空間を$S,T$とし,それらの確率質量変数が$p\left(x,y\right)$の形で与えられている.
この時,$X$単独の確率質量関数$p_X$は,
 \begin{align}
  p_X\left(x\right)&=\sum_{y\in T}p\left(x,y\right)
 \end{align}
であり,その平均$E\left(X\right)$は,
 \begin{align}
  E\left(X\right)&=\sum_{x\in S}xp_X\left(x\right)\nonumber\\
  &=\sum_{x\in S}x\sum_{y\in T}p\left(x,y\right)\nonumber\\
  &=\sum_{x\in S}\sum_{y\in T}xp\left(x,y\right)\nonumber\\
  &=\sum_{y\in T}\sum_{x\in S}xp\left(x,y\right)
 \end{align}
である.
同様に,$Y$単独の確率質量関数$p_Y$は,
 \begin{align}
  p_Y\left(y\right)&=\sum_{x\in S}p\left(x,y\right)
 \end{align}
であり,その平均$E\left(Y\right)$は,
 \begin{align}
  E\left(Y\right)&=\sum_{y\in T}yp_Y\left(y\right)\nonumber\\
  &=\sum_{y\in T}y\sum_{x\in S}p\left(x,y\right)\nonumber\\
  &=\sum_{y\in T}\sum_{x\in S}yp\left(x,y\right)
 \end{align}
である.
一方$X+Y$の平均$E\left(X+Y\right)$は,
 \begin{align}
  E\left(X+Y\right)&=\sum_{y\in T}\sum_{x\in S}\left(x+y\right)p\left(x,y\right)\nonumber\\
  &=\sum_{y\in T}\sum_{x\in S}\left(xp\left(x,y\right)+yp\left(x,y\right)\right)\nonumber\\
  &=\sum_{y\in T}\left(\left(\sum_{x\in S}xp\left(x,y\right)\right)+\left(\sum_{x\in S}yp\left(x,y\right)\right)\right)\nonumber\\
  &=\left(\sum_{y\in T}\sum_{x\in S}xp\left(x,y\right)\right)+\left(\sum_{y\in T}\sum_{x\in S}yp\left(x,y\right)\right)\nonumber\\
  &=E\left(X\right)+E\left(Y\right)
 \end{align}
である.
 \subsubsubsection{連続確率変数同士の和の平均}
互いに独立とは限らない2つの連続確率変数$X$と$Y$があり,それぞれの標本空間を$S,T$とし,それらの確率密度関数が$p\left(x,y\right)$の形で与えられている.
この時,$X$単独の確率密度関数$p_X$は,
 \begin{align}
  p_X\left(x\right)&=\int_{-\infty}^\infty p\left(x,y\right)dy
 \end{align}
であり,その平均$E\left(X\right)$は,
 \begin{align}
  E\left(X\right)&=\int_{-\infty}^\infty xp_X\left(x\right)dx\nonumber\\
  &=\int_{-\infty}^\infty x\int_{-\infty}^\infty p\left(x,y\right)dydx\nonumber\\
  &=\int_{-\infty}^\infty \int_{-\infty}^\infty xp\left(x,y\right)dydx\nonumber\\
  &=\int_{-\infty}^\infty \int_{-\infty}^\infty xp\left(x,y\right)dxdy\nonumber\\
 \end{align}
である.
同様に,$Y$単独の確率密度関数$p_Y$は,
 \begin{align}
  p_Y\left(y\right)&=\int_{-\infty}^\infty p\left(x,y\right)dx
 \end{align}
であり,その平均$E\left(Y\right)$は,
 \begin{align}
  E\left(Y\right)&=\int_{-\infty}^\infty yp_Y\left(y\right)dy\nonumber\\
  &=\int_{-\infty}^\infty y\int_{-\infty}^\infty p\left(x,y\right)dxdy\nonumber\\
  &=\int_{-\infty}^\infty \int_{-\infty}^\infty yp\left(x,y\right)dxdy\nonumber\\
 \end{align}
である.
一方$X+Y$の平均$E\left(X+Y\right)$は,
 \begin{align}
  E\left(X+Y\right)&=\int_{-\infty}^\infty\int_{-\infty}^\infty\left(x+y\right)p\left(x,y\right)dxdy\nonumber\\
  &=\int_{-\infty}^\infty\int_{-\infty}^\infty\left(xp\left(x,y\right)+yp\left(x,y\right)\right)dxdy\nonumber\\
  &=\int_{-\infty}^\infty\left(\int_{-\infty}^\infty xp\left(x,y\right)dx+\int_{-\infty}^\infty yp\left(x,y\right)dx\right)dy\nonumber\\
  &=\int_{-\infty}^\infty\int_{-\infty}^\infty xp\left(x,y\right)dxdy+\int_{-\infty}^\infty\int_{-\infty}^\infty yp\left(x,y\right)dxdy\nonumber\\
  &=E\left(X\right)+E\left(Y\right)
 \end{align}
 \subsubsubsection{離散確率変数と連続確率変数の和の平均}
互いに独立とは限らない離散確率変数$X$と連続確率変数$Y$があり,それぞれの標本空間を$S,T$とし,それらの確率密度関数が$p\left(x,y\right)$の形で与えられている.
この時,$X$単独の確率質量関数$p_X$は,
 \begin{align}
  p_X\left(x\right)&=\int_{-\infty}^\infty p\left(x,y\right)dy
 \end{align}
であり,その平均$E\left(X\right)$は,
 \begin{align}
  E\left(X\right)&=\sum_{x\in S}xp_X\left(x\right)\nonumber\\
  &=\sum_{x\in S}x\int_{-\infty}^\infty p\left(x,y\right)dy\nonumber\\
  &=\sum_{x\in S}\int_{-\infty}^\infty xp\left(x,y\right)dy\nonumber\\
  &=\int_{-\infty}^\infty\sum_{x\in S}xp\left(x,y\right)dy\nonumber\\
 \end{align}
である.
同様に,$Y$単独の確率密度関数$p_Y$は,
 \begin{align}
  p_Y\left(y\right)&=\sum_{x\in S}p\left(x,y\right)
 \end{align}
であり,その平均$E\left(Y\right)$は,
 \begin{align}
  E\left(Y\right)&=\int_{-\infty}^\infty yp_Y\left(y\right)dy\nonumber\\
  &=\int_{-\infty}^\infty y\sum_{x\in S}p\left(x,y\right)dy\nonumber\\
  &=\int_{-\infty}^\infty\sum_{x\in S}yp\left(x,y\right)dy\nonumber\\
 \end{align}
である.
一方$X+Y$の平均$E\left(X+Y\right)$は,
 \begin{align}
  E\left(X+Y\right)&=\int_{-\infty}^\infty\sum_{x\in S}\left(x+y\right)p\left(x,y\right)dy\nonumber\\
  &=\int_{-\infty}^\infty\sum_{x\in S}\left(xp\left(x,y\right)+yp\left(x,y\right)\right)dy\nonumber\\
  &=\int_{-\infty}^\infty\left(\sum_{x\in S}xp\left(x,y\right)+\sum_{x\in S}yp\left(x,y\right)\right)dy\nonumber\\
  &=\int_{-\infty}^\infty\sum_{x\in S}xp\left(x,y\right)dy+\int_{-\infty}^\infty\sum_{x\in S}yp\left(x,y\right)dy\nonumber\\
  &=E\left(X\right)+E\left(Y\right)
 \end{align}
である.
 \subsection{スカラー倍の平均は平均のスカラー倍}
ある確率変数$X$とある数$a$があるとき,
 \begin{align}
  E\left(aX\right)&=aE\left(X\right)
 \end{align}
である.
 \subsubsection{証明}
 \subsubsubsection{$X$が離散確率分布の場合}
式(\ref{DiscreteFunctionAverage})より,
 \begin{align}
  E\left(aX\right)&=\sum_{x\in S}axp\left(x\right)\nonumber\\
  &=a\sum_{x\in S}xp\left(x\right)\nonumber\\
  &=aE\left(X\right)\nonumber\\
 \end{align}
 \subsubsubsection{$X$が連続確率分布の場合}
式(\ref{ContinuousFunctionAverate})より,
 \begin{align}
  E\left(aX\right)&=\int_{-\infty}^\infty axp\left(x\right)dx\nonumber\\
  &=a\int_{-\infty}^\infty xp\left(x\right)dx\nonumber\\
  &=aE\left(X\right)
 \end{align}
 \section{平均と分散の関係に関する定理}
任意の確率変数$X$について,
 \begin{align}
  V\left(X\right)=E\left(X^2\right)-E\left(X\right)^2
 \end{align}
が成り立つ.
 \subsection{証明}
$E\left(X\right)$が定数であることに注意して平均の線形性を使えば証明できる.
 \begin{align}
  V\left(X\right)&=E\left(\left(X-E\left(X\right)\right)^2\right)\nonumber\\
  &=E\left(X^2-2XE\left(X\right)+E\left(X\right)^2\right)\nonumber\\
  &=E\left(X^2\right)-E\left(2XE\left(X\right)\right)+E\left(E\left(X\right)^2\right)\nonumber\\
  &=E\left(X^2\right)-2E\left(X\right)E\left(X\right)+E\left(X\right)^2\nonumber\\
  &=E\left(X^2\right)-2E\left(X\right)^2+E\left(X\right)^2\nonumber\\
  &=E\left(X^2\right)-E\left(X\right)^2
 \end{align}
 \section{独立な確率変数の積の平均}
互いに独立な2つの確率変数$X,Y$について,
 \begin{align}
  E\left(XY\right)&=E\left(X\right)E\left(Y\right)\label{ProductAverage}
 \end{align}
が成り立つ.
ただし,2つの確率変数が互いに独立であるということは,それらの組$(X,Y)$の確率質量関数若しくは確率密度関数を,それぞれの確率質量関数若しくは確率密度関数の積で表すことができることを意味する.
 \subsection{証明}
 \subsubsection{独立な離散確率変数同士の積の平均}
互いに独立な2つの離散確率変数$X,Y$があり,それぞれの標本空間を$S,T$とし,それぞれの確率質量関数を$p,q$とすると,
 \begin{align}
  E\left(XY\right)&=\sum_{x\in S}\sum_{y\in T}xyp\left(x\right)q\left(y\right)\nonumber\\
  &=\sum_{x\in S}xp\left(x\right)\sum_{y\in T}yq\left(y\right)\nonumber\\
  &=\left(\sum_{x\in S}xp\left(x\right)\right)\left(\sum_{y\in T}yq\left(y\right)\right)\nonumber\\
  &=E\left(X\right)E\left(Y\right)
 \end{align}
となる.
 \subsubsection{独立な連続確率変数同士の積の平均}
互いに独立な2つの連続確率変数$X,Y$があり,それぞれの標本空間を$S,T$とし,それぞれの確率質量関数を$p,q$とすると,
 \begin{align}
  E\left(XY\right)&=\int_{-\infty}^\infty\int_{-\infty}^\infty xyp\left(x\right)q\left(y\right)dydx\nonumber\\
  &=\int_{-\infty}^\infty xp\left(x\right)\int_{-\infty}^\infty yq\left(y\right)dydx\nonumber\\
  &=\int_{-\infty}^\infty xp\left(x\right)dx\int_{-\infty}^\infty yq\left(y\right)dy\nonumber\\
  &=E\left(X\right)E\left(Y\right)
 \end{align}
となる.
 \subsubsection{独立な離散確率変数と連続確率変数の積の平均}
互いに独立な離散確率変数$X$と連続確率変数$Y$があり,それぞれの標本空間を$S,T$とし,それぞれの確率質量関数を$p,q$とすると,
 \begin{align}
  E\left(XY\right)&=\sum_{x\in S}\int_{-\infty}^\infty xyp\left(x\right)q\left(y\right)dy\nonumber\\
  &=\sum_{x\in S}xp\left(x\right)\int_{-\infty}^\infty yq\left(y\right)dy\nonumber\\
  &=\left(\sum_{x\in S}xp\left(x\right)\right)\left(\int_{-\infty}^\infty yq\left(y\right)dy\right)\nonumber\\
  &=E\left(X\right)E\left(Y\right)
 \end{align}
となる.
 \section{標準偏差}
確率変数$X$に対して,
 \begin{align}
  \sigma\left(X\right)&=\sqrt{V\left(X\right)}
 \end{align}
を$X$の標準偏差といい,$X$の確率分布の広がり具合を表す$X$と次元を同じくする量である.
 \section{歪度}
確率変数$X$に対して,
 \begin{align}
  S\left(X\right)&=\frac{E\left(\left(X-E\left(X\right)\right)^3\right)}{\sigma\left(X\right)^3}
 \end{align}
を$X$の歪度といい,$X$の確率分布の平均周りの左右非対称具合を表す無次元量である.
歪度の定義式の直感的理解について述べる.
確率変数$X$の平均を原点に合わせるため,平均を引くと$X-E\left(X\right)$となる.
さらにこれを3乗すると,分布の平均に近い部分はより平均に近く寄せられ,平均から遠い部分はより平均から遠く伸ばされる.
分布が左右対称な確率変数$X$の場合その平均$E\left(\left(X-E\left(X\right)\right)^3\right)$をとっても0になる.
一方で,例えば平均より小さい方では分布が平均近くによっているのに対し,平均より大きい方では分布が遠くに引き伸ばされているような確率変数$X$を考えてみよう.
確率変数$\left(X-E\left(X\right)\right)^3$の分布は,平均より小さい側の平均近くの盛り上がりがより平均近くに寄せられ,平均より大きい側の伸びはより平均から遠くに伸ばされる.
結果的に3乗の影響で分布が全体的に正の方向にずれるので,その平均$E\left(\left(X-E\left(X\right)\right)^3\right)$も正の方向にずれる.
これは確率変数$X$の平均周りの左右非対称具合を表していると言える.
最後に,次元を無くすために$X$の分布の広がり具合と連動し,$X$の3乗の次元を持つ$\sigma\left(X\right)^3$で割れば,歪度の定義が現れる.
 \section{超過尖度}
確率変数$X$に対して,
 \begin{align}
  K\left(X\right)&=\frac{E\left(\left(X-E\left(X\right)\right)^4\right)}{\sigma\left(X\right)^4}-3
 \end{align}
を$X$の超過尖度といい,分散よりも外れ値への反応をより敏感にした指標であり,分布の裾の重さを表す.
超過尖度という名称から分布の平均周りの尖り具合を表していると誤解されやすいが,尖り具合を表してはいない.
 \section{エントロピー}
確率変数$X$に対して,
 \begin{align}
  H\left(X\right)&=E\left(-\log_2p\left(X\right)\right)
 \end{align}
を$X$のエントロピーといい,$X$の値を確定することによって生じる情報量の平均ビット数を意味する.
ただし$p$は離散確率変数であれば確率質量関数を意味し,連続確率変数であれば確率密度関数を意味する.
 \section{確率母関数}
ある離散確率変数$X$に対し,その確率母関数$G_X$は,
 \begin{align}
  G_X\left(t\right)&=E\left(t^X\right)\nonumber\\
  &=\sum_{x\in S}t^xp\left(x\right)
 \end{align}
と定義される.
 \subsection{確率母関数と確率質量関数の関係}
確率母関数$G_X$と確率質量関数$p$の間には,以下の関係が成り立つ.
 \begin{align}
  \frac{G_X^{\left(a\right)}\left(0\right)}{a!}&=p\left(a\right)
 \end{align}
 \subsubsection{証明}
確率母関数の定義をマクローリン展開することで証明できる.
 \begin{align}
  G_X\left(t\right)&=\sum_{x\in S}t^xp\left(x\right)\nonumber\\
  \sum_{a=0}^\infty\frac{G_X^{\left(a\right)}\left(0\right)t^a}{a!}&=\sum_{a=0}^\infty p\left(a\right)t^a\nonumber\\
  \frac{G_X^{\left(a\right)}\left(0\right)}{a!}&=p\left(a\right)
 \end{align}
 \subsection{確率母関数と階乗積率の関係}
確率母関数$G_X$と$n$次階乗積率$E\left(\frac{X!}{\left(X-n\right)!}\right)$の間には,以下の関係が成り立つ.
 \begin{align}
  G_X^{\left(n\right)}\left(1\right)&=E\left(\frac{X!}{\left(X-n\right)!}\right)
 \end{align}
 \subsubsection{証明}
確率母関数の定義を$t=1$でテイラー展開することで証明できる.
 \begin{align}
  G_X\left(t\right)&=E\left(t^X\right)\nonumber\\
  \sum_{n=0}^\infty\frac{\left(t-1\right)^n}{n!}G_X^{\left(n\right)}\left(1\right)&=\sum_{n=1}^\infty\frac{\left(t-1\right)^n}{n!}\left(\left.\frac{d^nE\left(t^X\right)}{dt^n}\right|_{t=1}\right)\nonumber\\
  G_X^{\left(n\right)}\left(1\right)&=\left.\frac{d^nE\left(t^X\right)}{dt^n}\right|_{t=1}\nonumber\\
  &=\left.\frac{d^n}{dt^n}\sum_{m=0}^\infty p\left(m\right)t^m\right|_{t=1}\nonumber\\
  &=\left.\sum_{m=0}^\infty p\left(m\right)\frac{d^n}{dt^n}t^m\right|_{t=1}\nonumber\\
  &=\left.\sum_{m=n}^\infty p\left(m\right)\frac{m!t^{m-n}}{\left(m-n\right)!}\right|_{t=1}\nonumber\\
  &=\sum_{m=n}^\infty p\left(m\right)\frac{m!}{\left(m-n\right)!}\nonumber\\
  &=E\left(\frac{X!}{\left(X-n\right)!}\right)
 \end{align}
$m<n$のとき,分母の$\left(m-n\right)!$が無限大に発散することにより,項が0に収束するので問題ない.
 \subsection{互いに独立な確率変数の和の確率母関数}
互いに独立な$N$個の確率変数$X_1,\cdots,X_N$の和$X=\sum_{n=1}^NX_n$の確率母関数$G_X$は,
 \begin{align}
  G_X\left(t\right)&=\prod_{n=1}^NG_{X_n}\left(t\right)\label{SummationProbabilityGeneratingFunction}
 \end{align}
である.
 \subsubsection{証明}
式(\ref{ProductAverage})を用いて証明する.
 \begin{align}
  G_X\left(t\right)&=E\left(t^X\right)\nonumber\\
  &=E\left(t^{\sum_{n=1}^NX_n}\right)\nonumber\\
  &=E\left(\prod_{n=1}^Nt^{X_n}\right)\nonumber\\
 \end{align}
ここで,$X_1,\cdots,X_N$は互いに独立だから式(\ref{ProductAverage})により,
 \begin{align}
  G_X\left(t\right)&=\prod_{n=1}^NE\left(t^{X_n}\right)\nonumber\\
  &=\prod_{n=1}^NG_{X_n}\left(t\right)
 \end{align}
 \section{積率母関数}
ある確率変数$X$に対し,その積率母関数$M_X$は,
 \begin{align}
  M_X\left(t\right)&=G_X\left(e^t\right)
 \end{align}
と定義される.
 \subsection{積率母関数と原点積率の関係}
 \begin{align}
  M_X^{\left(n\right)}\left(0\right)&=E\left(X^n\right)
 \end{align}
 \subsubsection{証明}
積率母関数の定義をマクローリン展開することで証明できる.
 \begin{align}
  M_X\left(t\right)&=G_X\left(e^t\right)\nonumber\\
  &=E\left(e^{tX}\right)\nonumber\\
  \sum_{n=0}^\infty\frac{t^n}{n!}M_X^{\left(n\right)}\left(0\right)&=\sum_{n=0}^\infty\frac{t^n}{n!}\left(\left.\frac{d^n}{dt^n}E\left(e^{tX}\right)\right|_{t=0}\right)\nonumber\\
  M_X^{\left(n\right)}\left(0\right)&=\left(\left.\frac{d^n}{dt^n}E\left(e^{tX}\right)\right|_{t=0}\right)\nonumber\\
  &=\left.\frac{d^n}{dt^n}E\left(\sum_{m=0}^\infty\frac{t^m}{m!}\left(\left.\frac{d^me^{tX}}{dt^m}\right|_{t=0}\right)\right)\right|_{t=0}\nonumber\\
  &=\left.\frac{d^n}{dt^n}E\left(\sum_{m=0}^\infty\frac{t^m}{m!}\left(\left.X^me^{tX}\right|_{t=0}\right)\right)\right|_{t=0}\nonumber\\
  &=\left.\frac{d^n}{dt^n}E\left(\sum_{m=0}^\infty\frac{t^m}{m!}X^m\right)\right|_{t=0}\nonumber\\
  &=\left.\frac{d^n}{dt^n}\sum_{m=0}^\infty\frac{t^m}{m!}E\left(X^m\right)\right|_{t=0}\nonumber\\
  &=\left.\sum_{m=0}^\infty\frac{E\left(X^m\right)}{m!}\frac{d^nt^m}{dt^n}\right|_{t=0}\nonumber\\
  &=\left.\sum_{m=n}^\infty\frac{E\left(X^m\right)}{m!}\frac{m!}{\left(m-n\right)!}t^{m-n}\right|_{t=0}\nonumber\\
  &=\left.\sum_{m=n}^\infty\frac{E\left(X^m\right)}{\left(m-n\right)!}t^{m-n}\right|_{t=0}\nonumber\\
  &=\left.\frac{E\left(X^n\right)}{\left(n-n\right)!}t^{n-n}+\sum_{m=n+1}^\infty\frac{E\left(X^m\right)}{\left(m-n\right)!}t^{m-n}\right|_{t=0}\nonumber\\
  &=\left.E\left(X^n\right)+\sum_{m=1}^\infty\frac{E\left(X^{m+n}\right)}{m!}t^m\right|_{t=0}\nonumber\\
  &=E\left(X^n\right)+\sum_{m=1}^\infty\frac{E\left(X^{m+n}\right)}{m!}0^m\nonumber\\
  &=E\left(X^n\right)
 \end{align}
 \subsubsection{平均と分散への応用}
この定理より,確率変数$X$の平均と分散はそれぞれ
 \begin{align}
  E\left(X\right)&=M_X^{\left(1\right)}\left(0\right)\\
  V\left(X\right)&=E\left(X^2\right)-E\left(X\right)^2\nonumber\\
  &=M_X^{\left(2\right)}\left(0\right)-M_X^{\left(1\right)}\left(0\right)^2
 \end{align}
と表される.
 \subsection{確率変数の一次式の積率母関数}
ある確率変数$X$の一次式$aX+b$の積率母関数$M_{aX+b}$は,
 \begin{align}
  M_{aX+b}\left(t\right)&=e^{bt}M_X\left(at\right)\label{LinearMomentGeneratingFunction}
 \end{align}
である.
 \subsubsection{証明}
 \begin{align}
  M_{aX+b}\left(t\right)&=G_{aX+b}\left(e^t\right)\nonumber\\
  &=E\left(e^{t\left(aX+b\right)}\right)\nonumber\\
  &=E\left(e^{atX+bt}\right)\nonumber\\
  &=E\left(e^{atX}e^{bt}\right)\nonumber\\
  &=e^{bt}E\left(e^{atX}\right)\nonumber\\
  &=e^{bt}G_X\left(e^{at}\right)\nonumber\\
  &=e^{bt}M_X\left(at\right)\nonumber\\
 \end{align}
 \subsection{互いに独立な確率変数の和の積率母関数}
互いに独立な$N$個の確率変数$X_1,\cdots,X_N$の和$X=\sum_{n=1}^NX_n$の積率母関数$M_X\left(t\right)$は,
 \begin{align}
  M_X\left(t\right)&=\prod_{n=1}^NM_{X_n}\left(t\right)\label{SummationMomentGeneratingFunction}
 \end{align}
である.
 \subsubsection{証明}
式(\ref{SummationProbabilityGeneratingFunction})より,
 \begin{align}
  M_X\left(t\right)&=G_X\left(e^t\right)\nonumber\\
  &=\prod_{n=1}^NG_{X_n}\left(e^t\right)\nonumber\\
  &=\prod_{n=1}^NM_{X_n}\left(t\right)
 \end{align}
 \section{特性関数}
確率変数$X$の特性関数$\phi_X$は,
 \begin{align}
  \phi_X\left(t\right)&=M_X\left(it\right)
 \end{align}
で定義される.
 \subsection{特性関数と原点積率の関係}
 \begin{align}
  \left(-i\right)^n\phi_X^{\left(n\right)}\left(0\right)&=E\left(X^n\right)
 \end{align}
 \subsubsection{証明}
特性関数の定義をマクローリン展開することで証明できる.
 \begin{align}
  \phi_X\left(t\right)&=M_X\left(it\right)\nonumber\\
  &=G_X\left(e^{it}\right)\nonumber\\
  &=E\left(e^{itX}\right)\nonumber\\
  \sum_{n=0}^\infty\frac{t^n}{n!}\phi_X^{\left(n\right)}\left(0\right)&=\sum_{n=0}^\infty\frac{t^n}{n!}\left(\left.\frac{d^n}{dt^n}E\left(e^{itX}\right)\right|_{t=0}\right)\nonumber\\
  \phi_X^{\left(n\right)}\left(0\right)&=\left.\frac{d^n}{dt^n}E\left(e^{itX}\right)\right|_{t=0}\nonumber\\
  &=\left.\frac{d^n}{dt^n}E\left(\sum_{m=0}^\infty\frac{t^m}{m!}\left(\left.\frac{d^me^{itX}}{dt^m}\right|_{t=0}\right)\right)\right|_{t=0}\nonumber\\
  &=\left.\frac{d^n}{dt^n}E\left(\sum_{m=0}^\infty\frac{t^m}{m!}\left(\left.\left(iX\right)^me^{itX}\right|_{t=0}\right)\right)\right|_{t=0}\nonumber\\
  &=\left.\frac{d^n}{dt^n}E\left(\sum_{m=0}^\infty\frac{t^m}{m!}\left(iX\right)^m\right)\right|_{t=0}\nonumber\\
  &=\left.\frac{d^n}{dt^n}E\left(\sum_{m=0}^\infty\frac{i^mt^m}{m!}X^m\right)\right|_{t=0}\nonumber\\
  &=\left.\frac{d^n}{dt^n}\sum_{m=0}^\infty\frac{i^mt^m}{m!}E\left(X^m\right)\right|_{t=0}\nonumber\\
  &=\left.\sum_{m=0}^\infty\frac{i^mE\left(X^m\right)}{m!}\frac{d^nt^m}{dt^n}\right|_{t=0}\nonumber\\
  &=\left.\sum_{m=n}^\infty\frac{i^mE\left(X^m\right)}{m!}\frac{m!}{\left(m-n\right)!}t^{m-n}\right|_{t=0}\nonumber\\
  &=\left.\sum_{m=n}^\infty\frac{i^mE\left(X^m\right)}{\left(m-n\right)!}t^{m-n}\right|_{t=0}\nonumber\\
  &=\left.\frac{i^nE\left(X^n\right)}{\left(n-n\right)!}t^{n-n}+\sum_{m=n+1}^\infty\frac{i^mE\left(X^m\right)}{\left(m-n\right)!}t^{m-n}\right|_{t=0}\nonumber\\
  &=\left.i^nE\left(X^n\right)+\sum_{m=1}^\infty\frac{i^{m+n}E\left(X^{m+n}\right)}{\left(m\right)!}t^{m}\right|_{t=0}\nonumber\\
  &=i^nE\left(X^n\right)+\sum_{m=1}^\infty\frac{i^{m+n}E\left(X^{m+n}\right)}{\left(m\right)!}0^{m}\nonumber\\
  &=i^nE\left(X^n\right)\nonumber\\
  \left(-i\right)^n\phi_X^{\left(n\right)}\left(0\right)&=E\left(X^n\right)
 \end{align}
 \subsection{確率変数の一次式の特性関数}
確率変数$X$の一次式$aX+b$の特性関数$\phi_{aX+b}$は,
 \begin{align}
  \phi_{aX+b}\left(t\right)&=e^{ibt}\phi_{X}\left(at\right)
 \end{align}
である.
 \subsubsection{証明}
式(\ref{LinearMomentGeneratingFunction})より,
 \begin{align}
  \phi_{aX+b}\left(t\right)&=M_{aX+b}\left(it\right)\nonumber\\
  &=e^{ibt}M_X\left(iat\right)\nonumber\\
  &=e^{ibt}\phi_{X}\left(at\right)
 \end{align}
 \subsection{互いに独立な確率変数の和の特性関数}
互いに独立な$N$個の確率変数$X_1,\cdots,X_N$の和$X=\sum_{n=1}^NX_n$の特性関数$\phi_X$は,
 \begin{align}
  \phi_X\left(t\right)&=\prod_{n=1}^N\phi_{X_n}\left(t\right)
 \end{align}
である.
 \subsubsection{証明}
式(\ref{SummationMomentGeneratingFunction})を用いて証明する.
 \begin{align}
  \phi_X\left(t\right)&=M_X\left(it\right)\nonumber\\
  &=\prod_{n=1}^NM_{X_n}\left(it\right)\nonumber\\
  &=\prod_{n=1}^N\phi_{X_n}\left(t\right)
 \end{align}
 \section{キュムラント母関数}
確率変数$X$のキュムラント母関数$K_X$は,
 \begin{align}
  K_X\left(t\right)&=\log M_X\left(t\right)\label{CumulantGeneratingFunction}
 \end{align}
と定義される.
$K_X$のマクローリン展開
 \begin{align}
  K_X\left(t\right)&=\sum_{k=0}^\infty \frac{K_X^{\left(k\right)}\left(0\right)t^k}{k!}
 \end{align}
における$K_X^{\left(k\right)}\left(0\right)$を確率変数$X$の$k$次キュムラントという.
 \subsection{Bellの多項式}
集合
 \begin{align}
  S=\{1,\cdots,|S|\}
 \end{align}
を,集合
 \begin{align}
  \{S_{ij}|i\in I=\{1,\cdots,|I|\}\land j\in J\left(i\right)=\{1,\cdots,|J\left(i\right)|\}\}\label{BellDividing}
 \end{align}
に分けたい.
但し,
 \begin{align}
  &S=\bigcup_{i\in I}\bigcup_{j\in J\left(i\right)}S_{ij}\\
  &\forall i_1\in I,\forall j_1\in J\left(i_1\right),\forall i_2\in I,\forall j_2\in J\left(i_2\right),\left(i_1,j_1\right)\ne\left(i_2,j_2\right)\implies S_{i_1j_1}\cup S_{i_2j_2}=\emptyset\\
  &\forall i\in I,\forall j\in J\left(i\right),|S_{ij}|=i\\
  &\sum_{i\in I}i|J\left(i\right)|=|S|
 \end{align}
とする.
このような分け方は何通りあるだろうか?
まず,
 \begin{align}
  S_i=\bigcup_{j\in J\left(i\right)}S_{ij}
 \end{align}
とし,集合$S$を$\{S_i|i\in I\}$に分ける.
$S_i$の濃度は,
 \begin{align}
  |S_i|&=|\bigcup_{j\in J\left(i\right)}S_{ij}|\nonumber\\
  &=\sum_{j\in J\left(i\right)}|S_{ij}|\nonumber\\
  &=\sum_{j\in J\left(i\right)}i\nonumber\\
  &=i|J\left(i\right)|
 \end{align}
だから,まず$S_1$の選び方は,
 \begin{align}
  \binom{|S|}{|J\left(1\right)|}
 \end{align}
次に$S_2$の選び方は,
 \begin{align}
  \binom{|S|-|J\left(1\right)|}{2|J\left(2\right)|}
 \end{align}
次に$S_3$の選び方は,
 \begin{align}
  \binom{|S|-|J\left(1\right)|-2|J\left(2\right)|}{3|J\left(3\right)|}
 \end{align}
一般に$S_i$の選び方は,
 \begin{align}
  \binom{|S|-\sum_{k=1}^{i-1}k|J\left(k\right)|}{i|J\left(i\right)|}
 \end{align}
最後に$S_{|I|}$の選び方は,
 \begin{align}
  \binom{|S|-\sum_{i=1}^{|I|-1}i|J\left(i\right)|}{|I||J\left(|I|\right)|}
 \end{align}
これらの場合の数は,
 \begin{align}
  \prod_{i\in I}\binom{|S|-\sum_{k=1}^{i-1}k|J\left(k\right)|}{i|J\left(i\right)|}\label{BellOutlineEnumeration}
 \end{align}
である.
次に,$\forall i\in I$について,$S_i$を$\{S_{ij}|j\in J\left(i\right)\}$に分ける.
まず$S_{i1}$の選び方は,
 \begin{align}
  \binom{i|J\left(i\right)|}{i}
 \end{align}
次に$S_{i2}$の選び方は,
 \begin{align}
  \binom{i\left(|J\left(i\right)|-1\right)}{i}
 \end{align}
次に$S_{i3}$の選び方は,
 \begin{align}
  \binom{i\left(|J\left(i\right)|-2\right)}{i}
 \end{align}
一般に$S_{ij}$の選び方は,
 \begin{align}
  \binom{i\left(|J\left(i\right)|-j+1\right)}{i}
 \end{align}
最後に$S_{i|J\left(i\right)|}$の選び方は,
 \begin{align}
  \binom{i\left(|J\left(i\right)|-|J\left(i\right)|+1\right)}{i}
 \end{align}
これらの場合の数は,
 \begin{align}
  &\prod_{j\in J\left(i\right)}\binom{i\left(|J\left(i\right)|-j+1\right)}{i}\nonumber\\
  =&\prod_{j\in J\left(i\right)}\binom{i|J\left(i\right)|-ij+i}{i}\nonumber\\
  =&\prod_{j\in J\left(i\right)}\frac{\left(i|J\left(i\right)|-ij+i\right)!}{\left(i|J\left(i\right)|-ij\right)!i!}\nonumber\\
  =&\frac{1}{\left(i!\right)^{|J\left(i\right)|}}\prod_{j\in J\left(i\right)}\frac{\left(i|J\left(i\right)|-ij+i\right)!}{\left(i|J\left(i\right)|-ij\right)!}\nonumber\\
  =&\frac{1}{\left(i!\right)^{|J\left(i\right)|}}\prod_{j\in J\left(i\right)}\prod_{k=1}^i\left(i|J\left(i\right)|-ij+k\right)\nonumber\\
  =&\frac{1}{\left(i!\right)^{|J\left(i\right)|}}\prod_{l\in\{ij-k|j\in J\left(i\right)\land k\in\{1,\cdots,i\}\}}\left(i|J\left(i\right)|-l\right)\nonumber\\
  =&\frac{1}{\left(i!\right)^{|J\left(i\right)|}}\prod_{l=0}^{i|J\left(i\right)|-1}\left(i|J\left(i\right)|-l\right)\nonumber\\
  =&\frac{\left(i|J\left(i\right)|\right)!}{\left(i!\right)^{|J\left(i\right)|}}\label{BellDetailEnumerationWithDuplication}
 \end{align}
ただし,$S_{i1},\cdots,S_{i|J\left(i\right)|}$の順序は区別しないため,これらの順序の個数で割る必要がある.
$S_{i1},\cdots,S_{i|J\left(i\right)|}$の順序の個数を数えてみよう.
まず$|J\left(i\right)|$個の中から$1$個選ぶ場合の数は,
 \begin{align}
  \binom{|J\left(i\right)|}{1}=|J\left(i\right)|
 \end{align}
次に$|J\left(i\right)-1|$個の中から$1$個選ぶ場合の数は,
 \begin{align}
  \binom{|J\left(i\right)|-1}{1}=|J\left(i\right)|-1
 \end{align}
次に$|J\left(i\right)-2|$個の中から$1$個選ぶ場合の数は,
 \begin{align}
  \binom{|J\left(i\right)|-2}{1}=|J\left(i\right)|-2
 \end{align}
一般に$|J\left(i\right)-j+1|$個の中から$1$個選ぶ場合の数は,
 \begin{align}
  \binom{|J\left(i\right)|-j+1}{1}=|J\left(i\right)|-j+1
 \end{align}
一般に$|J\left(i\right)-|J\left(i\right)|+1|$個の中から$1$個選ぶ場合の数は,
 \begin{align}
  \binom{|J\left(i\right)|-|J\left(i\right)|+1}{1}=|J\left(i\right)|-|J\left(i\right)|+1
 \end{align}
これらの場合の数は,
 \begin{align}
  \prod_{j\in J\left(i\right)}\left(|J\left(i\right)|-j+1\right)=|J\left(i\right)|!\label{BellDetailEnumerationDuplication}
 \end{align}
\ref{BellDetailEnumerationWithDuplication}を\ref{BellDetailEnumerationDuplication}で割ると,
 \begin{align}
  \frac{\left(i|J\left(i\right)|\right)!}{\left(i!\right)^{|J\left(i\right)|}|J\left(i\right)|!}\label{BellDetailEnumeration}
 \end{align}
\ref{BellOutlineEnumeration}および\ref{BellDetailEnumeration}を合わせると,場合の数は,
 \begin{align}
  &\prod_{i\in I}\binom{|S|-\sum_{k=1}^{i-1}k|J\left(k\right)|}{i|J\left(i\right)|}\frac{\left(i|J\left(i\right)|\right)!}{\left(i!\right)^{|J\left(i\right)|}|J\left(i\right)|!}\nonumber\\
  =&\prod_{i\in I}\frac{\left(|S|-\sum_{k=1}^{i-1}k|J\left(k\right)|\right)!}{\left(|S|-i|J\left(i\right)|-\sum_{k=1}^{i-1}k|J\left(k\right)|\right)!\left(i|J\left(i\right)|\right)!}\frac{\left(i|J\left(i\right)|\right)!}{\left(i!\right)^{|J\left(i\right)|}|J\left(i\right)|!}\nonumber\\
  =&\prod_{i\in I}\frac{\left(|S|-\sum_{k=1}^{i-1}k|J\left(k\right)|\right)!}{\left(|S|-i|J\left(i\right)|-\sum_{k=1}^{i-1}k|J\left(k\right)|\right)!\left(i!\right)^{|J\left(i\right)|}|J\left(i\right)|!}\nonumber\\
  =&\prod_{i\in I}\frac{1}{\left(i!\right)^{|J\left(i\right)|}|J\left(i\right)|!}\prod_{l=1}^{i|J\left(i\right)|}\left(|S|-i|J\left(i\right)|+l-\sum_{k=1}^{i-1}k|J\left(k\right)|\right)\nonumber\\
  =&\left(\prod_{i\in I}\frac{1}{\left(i!\right)^{|J\left(i\right)|}|J\left(i\right)|!}\right)\left(\prod_{i\in I}\prod_{l=1}^{i|J\left(i\right)|}\left(|S|-i|J\left(i\right)|+l-\sum_{k=1}^{i-1}k|J\left(k\right)|\right)\right)\nonumber\\
  =&\left(\prod_{i\in I}\frac{1}{\left(i!\right)^{|J\left(i\right)|}|J\left(i\right)|!}\right)\left(\prod_{m\in\{i|J\left(i\right)|-l+\sum_{k=1}^{i-1}k|J\left(k\right)||i\in I\land l\in\{1,\cdots,i|J\left(i\right)|\}\}}\left(|S|-m\right)\right)\nonumber\\
  =&\left(\prod_{i\in I}\frac{1}{\left(i!\right)^{|J\left(i\right)|}|J\left(i\right)|!}\right)\left(\prod_{m=0}^{|S|-1}\left(|S|-m\right)\right)\nonumber\\
  =&\left(\prod_{i\in I}\frac{1}{\left(i!\right)^{|J\left(i\right)|}|J\left(i\right)|!}\right)|S|!\nonumber\\
  =&|S|!\prod_{i\in I}\frac{1}{\left(i!\right)^{|J\left(i\right)|}|J\left(i\right)|!}\label{BellEnumerationWithoutJ}
 \end{align}
ここで,\ref{BellDividing}において,分ける個数を$k$とする場合,関数${}_{\max}J:I\to \mathbb{N};i\mapsto|J\left(i\right)|$の可能な形の集合${}_{\max} J_k$は,
 \begin{align}
  {}_{\max} J_k=\{{}_{\max}J:I\to \mathbb{N};i\mapsto|J\left(i\right)||\sum_{i\in I}|J\left(i\right)|=k\land\sum_{i\in I}i|J\left(i\right)|=|S|\}
 \end{align}
よって,関数$J:I\to 2^\mathbb{N};i\mapsto\{1,\cdots,{}_{\max}J\left(i\right)\}$の可能な形の集合$J_k$は,
 \begin{align}
  J_k=\{J:I\to 2^\mathbb{N};i\mapsto\{1,\cdots,{}_{\max}J\left(i\right)\}|{}_{\max}J\in{}_{\max} J_k\}
 \end{align}
となる.
ここから関数$J$を選ぶ場合の数も考慮して\ref{BellEnumerationWithoutJ}を拡張すると,集合$S$を$k$個の部分集合に分割する場合の数は,
 \begin{align}
  \sum_{J\in J_k}|S|!\prod_{i\in I}\frac{1}{\left(i!\right)^{|J\left(i\right)|}|J\left(i\right)|!}\label{BellEnumaration}
 \end{align}
となる.
ここで,部分集合$S_{ij}$の濃度に応じて何らかの実数を返す関数
 \begin{align}
  &f:I\to\mathbb{R}\\
  &g:\{S_{ij}|i\in I\land j\in J\left(i\right)\}\to\mathbb{R};S_{ij}\mapsto f\left(i\right)
 \end{align}
を考えた時,分けられた部分集合族に渡るそれらの総積
 \begin{align}
  \prod_{i\in I\land j\in J\left(i\right)}g\left(S_{ij}\right)&=\prod_{i\in I\land j\in J\left(i\right)}f\left(i\right)\nonumber\\
  &=\prod_{i\in I}f\left(i\right)^{|J\left(i\right)|}
 \end{align}
の,全ての分け方の場合に渡る総和がどうなるか考えよう.
場合の数を数えた時にまず$i$を固定して$j$を変動させたときの場合の数を計算し,その後さらに$i$を変動させて場合の数を計算した.
この流れに合わせて関数$f$を組み込むために,$i$を固定した形
 \begin{align}
  f\left(i\right)^{|J\left(i\right)|}\label{BellWeight}
 \end{align}
を用いる.
これは,ひとつの場合における関数の総積を各$i$ごとに因数分解し,$i$を固定して$j$を変動させたときの場合の数に組み込む形になるが,場合の数の算出において各$i$は積の形で再び合流するため,問題ない.
\ref{BellWeight}を\ref{BellDetailEnumerationWithDuplication}に組み込むと,
 \begin{align}
  \frac{\left(i|J\left(i\right)|\right)!}{\left(i!\right)^{|J\left(i\right)|}}f\left(i\right)^{|J\left(i\right)|}&=\left(i|J\left(i\right)|\right)!\frac{f\left(i\right)^{|J\left(i\right)|}}{\left(i!\right)^{|J\left(i\right)|}}\nonumber\\
  &=\left(i|J\left(i\right)|\right)!\left(\frac{f\left(i\right)}{i!}\right)^{|J\left(i\right)|}
 \end{align}
これを\ref{BellDetailEnumerationDuplication}で割ると,
 \begin{align}
  \frac{\left(i|J\left(i\right)|\right)!}{|J\left(i\right)|!}\left(\frac{f\left(i\right)}{i!}\right)^{|J\left(i\right)|}
 \end{align}
これを\ref{BellOutlineEnumeration}に組み込むと,
 \begin{align}
  \prod_{i\in I}\binom{|S|-\sum_{k=1}^{i-1}k|J\left(k\right)|}{i|J\left(i\right)|}\frac{\left(i|J\left(i\right)|\right)!}{|J\left(i\right)|!}\left(\frac{f\left(i\right)}{i!}\right)^{|J\left(i\right)|}=|S|!\prod_{i\in I}\frac{1}{|J\left(i\right)|!}\left(\frac{f\left(i\right)}{i!}\right)^{|J\left(i\right)|}
 \end{align}
さらに$J$を選ぶ場合の数も考慮した
 \begin{align}
  B_{|S|,k}\left(f\left(1\right),\cdots,f\left(|I|=|S|-k+1\right)\right)=\sum_{J\in J_k}|S|!\prod_{i\in I}\frac{1}{|J\left(i\right)|!}\left(\frac{f\left(i\right)}{i!}\right)^{|J\left(i\right)|}
 \end{align}
を部分Bell多項式といい,大きさ$|S|$の集合の全ての可能な$k$個の部分集合への分割方法を考え,元の集合から分割された各部分集合の大きさ$i$に応じた値$f\left(i\right)$の総積として再統合し,それを全ての分割方法に渡って総和を取った値を意味する.
また,
 \begin{align}
  B_{|S|}\left(f\left(1\right),\cdots,f\left(|S|\right)\right)=\sum_{k=1}^{|S|}B_{|S|,k}\left(f\left(1\right),\cdots,f\left(|S|-k+1\right)\right)
 \end{align}
を完全Bell多項式といい,大きさ$|S|$の集合の全ての分割方法を考え,元の集合から分割された各部分集合の大きさ$i$に応じた値$f\left(i\right)$の総積として再統合し,それを全ての分割方法に渡って総和を取った値を意味する.
 \subsubsection{マクローリン展開の$k$乗の展開}
$\mathbb{R}$から$\mathbb{R}$への$C^\infty$級関数$f\in C^\infty\left(\mathbb{R}^\mathbb{R}\right)$ただし$f\left(0\right)=0$を$\forall k\in\mathbb{N}$乗してみよう.
 \begin{align}
  f\left(x\right)^k&=\left(\sum_{n=0}^\infty\frac{f^{\left(n\right)}\left(0\right)}{n!}x^n\right)^k\nonumber\\
  &=\left(\frac{f^{\left(0\right)}\left(0\right)}{0!}x^0+\sum_{n=1}^\infty\frac{f^{\left(n\right)}\left(0\right)}{n!}x^n\right)^k\nonumber\\
  &=\left(\sum_{n=1}^\infty\frac{f^{\left(n\right)}\left(0\right)}{n!}x^n\right)^k\nonumber\\
  &=\sum_{n_1=1}^\infty\cdots\sum_{n_k=1}^\infty\prod_{m=1}^k\frac{f^{\left(n_m\right)}\left(0\right)}{n_m!}x^{n_m}\nonumber\\
  &=\sum_{n=1}^\infty\sum_{\left(n_1,\cdots,n_k\right)\in\mathbb{N}^k|\sum_{j=1}^k n_j=n}\prod_{i\in\left(n_1,\cdots,n_k\right)}\frac{f^{\left(i\right)}\left(0\right)}{i!}x^i
 \end{align}
ここから,数列$\left(n_1,\cdots,n_k\right)$を,集合$\{n_1,\cdots,n_k\}$に書き換える.
数列$\left(n_1,\cdots,n_k\right)$は値が等しくても場所が異なれば区別するのに対し,集合$\{n_1,\cdots,n_k\}$は等しい値を区別しない.
従って,数列$\left(n_1,\cdots,n_k\right)$を集合$\{n_1,\cdots,n_k\}$に書き換えた際に式を保存するために,数列$\left(n_1,\cdots,n_k\right)$に登場する自然数を同じ値ごとに分類し,各値について同じ値がいくつ数列$\left(n_1,\cdots,n_k\right)$に現れるかを知っておく必要がある.
そのために,
 \begin{align}
  {}_{\max}J\left(i\right)=|\{j|n_j=i\}|
 \end{align}
とすると,
 \begin{align}
  \sum_{i\in\{n_1,\cdots,n_k\}}{}_{\max}J\left(i\right)&=\sum_{i\in\{n_1,\cdots,n_k\}}|\{j|n_j=i\}|\nonumber\\
  &=\sum_{i\in\left(n_1,\cdots,n_k\right)}1\nonumber\\
  &=k
 \end{align}
また,
 \begin{align}
  \sum_{i\in\{n_1,\cdots,n_k\}}i{}_{\max}J\left(i\right)&=\sum_{i\in\{n_1,\cdots,n_k\}}i|\{j|n_j=i\}|\nonumber\\
  &=\sum_{i\in\left(n_1,\cdots,n_k\right)}i\nonumber\\
  &=\sum_{m=1}^kn_m\nonumber\\
  &=n
 \end{align}
であり,これを用いて総積の範囲として現れている数列$\left(n_1,\cdots,n_k\right)$を集合$\{n_1,\cdots,n_k\}$に変えると,
 \begin{align}
  f\left(x\right)^k&=\sum_{n=1}^\infty\sum_{\left(n_1,\cdots,n_k\right)\in\mathbb{N}^k|\sum_{j=1}^k n_j=n}\prod_{i\in\left(n_1,\cdots,n_k\right)}\frac{f^{\left(i\right)}\left(0\right)}{i!}x^i\nonumber\\
  &=\sum_{n=1}^\infty\sum_{\left(n_1,\cdots,n_k\right)\in\mathbb{N}^k|\sum_{j=1}^k n_j=n}\prod_{i\in\{n_1,\cdots,n_k\}}\left(\frac{f^{\left(i\right)}\left(0\right)}{i!}x^i\right)^{{}_{\max}J\left(i\right)}
 \end{align}
となる.
次に総和の変数として現れている数列$\left(n_1,\cdots,n_k\right)$も集合$\{n_1,\cdots,n_k\}$として扱いたい.
まず,集合$\{n_1,\cdots,n_k\}$から数列$\left(n_1,\cdots,n_k\right)$を見ると,$k$個の順番が与えられているので$k!$を掛ける必要がある.
しかし,$n_{j_1}=n_{j_2}$であるような要素は集合$\{n_1,\cdots,n_k\}$においても数列$\left(n_1,\cdots,n_k\right)$においても順番を区別しないので,${}_{\max}J\left(i\right)!$で割る必要がある.
さらに,数列$\left(n_1,\cdots,n_k\right)$が保証していた条件$\sum_{j=1}^k n_j=n$に関して,集合$\{n_1,\cdots,n_k\}$は値の等しい要素を区別する能力がないため,これと同値な条件を保証する能力はないことがわかる.
代わりに,関数${}_{\max}J$を
 \begin{align}
  \mathbb{J}=\{{}_{\max}J:\{1,\cdots,n-k+1\}\to\{1,\cdots,n\};i\mapsto|\{j|n_j=i\}||\sum_{i\in\{n_1,\cdots,n_k\}}{}_{\max}J\left(i\right)=k\land\sum_{i\in\{n_1,\cdots,n_k\}}i{}_{\max}J\left(i\right)=n\}
 \end{align}
から選ぶようにすることで条件を保証する.
これらを考慮すると,
 \begin{align}
  f\left(x\right)^k&=\sum_{n=1}^\infty\sum_{\left(n_1,\cdots,n_k\right)\in\mathbb{N}^k|\sum_{j=1}^k n_j=n}\prod_{i\in\{n_1,\cdots,n_k\}}\left(\frac{f^{\left(i\right)}\left(0\right)}{i!}x^i\right)^{{}_{\max}J\left(i\right)}\nonumber\\
  &=\sum_{n=1}^\infty\sum_{{}_{\max}J\in\mathbb{J}}k!\prod_{i\in\{n_1,\cdots,n_k\}}\frac{1}{{}_{\max}J\left(i\right)!}\left(\frac{f^{\left(i\right)}\left(0\right)}{i!}x^i\right)^{{}_{\max}J\left(i\right)}\nonumber\\
  &=\sum_{n=1}^\infty\sum_{{}_{\max}J\in\mathbb{J}}k!\prod_{i\in\{n_1,\cdots,n_k\}}\frac{1}{{}_{\max}J\left(i\right)!}\left(\frac{f^{\left(i\right)}\left(0\right)}{i!}\right)^{{}_{\max}J\left(i\right)}x^{i{}_{\max}J\left(i\right)}\nonumber\\
  &=\sum_{n=1}^\infty\sum_{{}_{\max}J\in\mathbb{J}}k!\left(\prod_{i\in\{n_1,\cdots,n_k\}}\frac{1}{{}_{\max}J\left(i\right)!}\left(\frac{f^{\left(i\right)}\left(0\right)}{i!}\right)^{{}_{\max}J\left(i\right)}\right)\left(\prod_{i\in\{n_1,\cdots,n_k\}}x^{i{}_{\max}J\left(i\right)}\right)\nonumber\\
  &=\sum_{n=1}^\infty\sum_{{}_{\max}J\in\mathbb{J}}k!\left(\prod_{i\in\{n_1,\cdots,n_k\}}\frac{1}{{}_{\max}J\left(i\right)!}\left(\frac{f^{\left(i\right)}\left(0\right)}{i!}\right)^{{}_{\max}J\left(i\right)}\right)\left(x^{\sum_{i\in\{n_1,\cdots,n_k\}}i{}_{\max}J\left(i\right)}\right)\nonumber\\
  &=\sum_{n=1}^\infty\sum_{{}_{\max}J\in\mathbb{J}}k!\left(\prod_{i\in\{n_1,\cdots,n_k\}}\frac{1}{{}_{\max}J\left(i\right)!}\left(\frac{f^{\left(i\right)}\left(0\right)}{i!}\right)^{{}_{\max}J\left(i\right)}\right)x^n\nonumber\\
  &=\sum_{n=1}^\infty\left(\sum_{{}_{\max}J\in\mathbb{J}}k!\prod_{i\in\{n_1,\cdots,n_k\}}\frac{1}{{}_{\max}J\left(i\right)!}\left(\frac{f^{\left(i\right)}\left(0\right)}{i!}\right)^{{}_{\max}J\left(i\right)}\right)x^n\nonumber\\
  &=\sum_{n=1}^\infty\left(\sum_{{}_{\max}J\in\mathbb{J}}\frac{k!}{n!}n!\prod_{i\in\{n_1,\cdots,n_k\}}\frac{1}{{}_{\max}J\left(i\right)!}\left(\frac{f^{\left(i\right)}\left(0\right)}{i!}\right)^{{}_{\max}J\left(i\right)}\right)x^n\nonumber\\
  &=\sum_{n=1}^\infty\left(\frac{k!}{n!}\sum_{{}_{\max}J\in\mathbb{J}}n!\prod_{i\in\{n_1,\cdots,n_k\}}\frac{1}{{}_{\max}J\left(i\right)!}\left(\frac{f^{\left(i\right)}\left(0\right)}{i!}\right)^{{}_{\max}J\left(i\right)}\right)x^n\nonumber\\
  &=\sum_{n=1}^\infty\frac{k!}{n!}B_{n,k}\left(f^{\left(1\right)}\left(0\right),\cdots,f^{\left(n-k+1\right)}\left(0\right)\right)x^n\nonumber\\
 \end{align}
 \subsubsection{マクローリン展開の合成関数}
$\mathbb{R}$から$\mathbb{R}$への$C^\infty$級関数$\left(f,g\right)\in C^\infty\left(\mathbb{R}^\mathbb{R}\right)^2$ただし$g\left(0\right)=0$の合成は,
 \begin{align}
  f\left(g\left(x\right)\right)&=\sum_{k=0}^\infty\frac{f^{\left(k\right)}\left(0\right)}{k!}\left(\sum_{n=0}^\infty\frac{g^{\left(n\right)}\left(0\right)}{n!}x^n\right)^k\nonumber\\
  &=f\left(0\right)+\sum_{k=0}^\infty\frac{f^{\left(k\right)}\left(0\right)}{k!}\left(g\left(0\right)+\sum_{n=1}^\infty\frac{g^{\left(n\right)}\left(0\right)}{n!}x^n\right)^k\nonumber\\
  &=f\left(0\right)+\sum_{k=0}^\infty\frac{f^{\left(k\right)}\left(0\right)}{k!}\left(\sum_{n=1}^\infty\frac{g^{\left(n\right)}\left(0\right)}{n!}x^n\right)^k\nonumber\\
  &=f\left(0\right)+\sum_{k=0}^\infty\frac{f^{\left(k\right)}\left(0\right)}{k!}\sum_{n=1}^\infty\frac{k!}{n!}B_{n,k}\left(g^{\left(1\right)}\left(0\right),\cdots,g^{\left(n-k+1\right)}\left(0\right)\right)x^n\nonumber\\
  &=f\left(0\right)+\sum_{n=1}^\infty\sum_{k=0}^\infty\frac{f^{\left(k\right)}\left(0\right)}{n!}B_{n,k}\left(g^{\left(1\right)}\left(0\right),\cdots,g^{\left(n-k+1\right)}\left(0\right)\right)x^n\nonumber\\
  &=f\left(0\right)+\sum_{n=1}^\infty\sum_{k=0}^n\frac{f^{\left(k\right)}\left(0\right)}{n!}B_{n,k}\left(g^{\left(1\right)}\left(0\right),\cdots,g^{\left(n-k+1\right)}\left(0\right)\right)x^n\nonumber\\
 \end{align}
 \subsubsection{Fa\`a di Brunoの公式}
$\mathbb{R}$から$\mathbb{R}$への$C^\infty$級関数$\forall\left(f,g\right)\in C^\infty\left(\mathbb{R}^\mathbb{R}\right)^2$の合成関数の$n\in\mathbb{N}$回微分は,
 \begin{align}
  \frac{d^n}{dx^n}f\left(g\left(x\right)\right)=\sum_{k=1}^nf^{\left(k\right)}\left(g\left(x\right)\right)B_{n,k}\left(g^{\left(1\right)}\left(x\right),\cdots,g^{\left(n-k+1\right)}\left(x\right)\right)
 \end{align}
である.
これがFa\`a di Brunoの公式である.
これを証明するために以下の補題
 \begin{align}
  \forall n\in\mathbb{N},\forall f\in C^n\left(\mathbb{R}^\mathbb{R}\right),\frac{d^nf\left(x\right)}{dx^n}=\frac{d^nf\left(x\right)}{d\left(x+y\right)^n}
 \end{align}
を数学的帰納法により証明する.
$n=1$の時,
 \begin{align}
  \frac{df\left(x\right)}{dx}&=\frac{df\left(x\right)}{d\left(x+y\right)}\frac{d\left(x+y\right)}{dx}\nonumber\\
  &=\frac{df\left(x\right)}{d\left(x+y\right)}
 \end{align}
また,
 \begin{align}
  \exists n\in\mathbb{N},\frac{d^nf\left(x\right)}{dx^n}=\frac{d^nf\left(x\right)}{d\left(x+y\right)^n}
 \end{align}
のとき,
 \begin{align}
  \frac{d^{n+1}f\left(x\right)}{dx^{n+1}}&=\frac{d}{dx}\frac{d^nf\left(x\right)}{dx^n}\nonumber\\
  &=\frac{d}{dx}\frac{d^nf\left(x\right)}{d\left(x+y\right)^n}\nonumber\\
  &=\frac{d}{d\left(x+y\right)}\frac{d^nf\left(x\right)}{d\left(x+y\right)^n}\frac{d\left(x+y\right)}{dx}\nonumber\\
  &=\frac{d}{d\left(x+y\right)}\frac{d^nf\left(x\right)}{d\left(x+y\right)^n}\nonumber\\
  &=\frac{d^{n+1}f\left(x\right)}{d\left(x+y\right)^{n+1}}
 \end{align}
よって数学的帰納法により補題が成り立つ.
これでFa\`a di Brunoの公式を証明する準備が整った.
$\mathbb{R}$から$\mathbb{R}$への$C^\infty$級関数$\forall\left(f,g\right)\in C^\infty\left(\mathbb{R}^\mathbb{R}\right)^2$をもとに,以下の関数を定義する.
 \begin{align}
  F\left(x,y\right)&=f\left(g\left(x\right)+y\right)\\
  G\left(x,y\right)&=g\left(x+y\right)-g\left(x\right)
 \end{align}
ここで,$F\left(x,y\right)$を$y$周りでマクローリン展開すると,
 \begin{align}
  F\left(x,y\right)&=\sum_{k=0}^\infty\frac{1}{k!}\left(\left.\frac{\partial^nF\left(x,y\right)}{\partial y^n}\right|_{y=0}\right)y^k\nonumber\\
  &=\sum_{k=0}^\infty\frac{1}{k!}\left(\left.\frac{\partial^nf\left(g\left(x\right)+y\right)}{\partial y^n}\right|_{y=0}\right)y^k\nonumber\\
  &=\sum_{k=0}^\infty\frac{1}{k!}\left(\left.\frac{\partial^nf\left(g\left(x\right)+y\right)}{\partial\left(g\left(x\right)+y\right)^n}\right|_{y=0}\right)y^k\nonumber\\
  &=\sum_{k=0}^\infty\frac{f^{\left(n\right)}\left(g\left(x\right)\right)}{k!}y^k\nonumber\\
 \end{align}
また,$G\left(x,y\right)$を$y$周りでマクローリン展開すると,
 \begin{align}
  G\left(x,y\right)&=\sum_{n=0}^\infty\frac{1}{n!}\left(\left.\frac{\partial^nG\left(x,y\right)}{\partial y^n}\right|_{y=0}\right)y^n\nonumber\\
  &=G\left(x,0\right)+\sum_{n=1}^\infty\frac{1}{n!}\left(\left.\frac{\partial^nG\left(x,y\right)}{\partial y^n}\right|_{y=0}\right)y^n\nonumber\\
  &=g\left(x\right)-g\left(x\right)+\sum_{n=1}^\infty\frac{1}{n!}\left(\left.\frac{\partial^n\left(g\left(x+y\right)-g\left(x\right)\right)}{\partial y^n}\right|_{y=0}\right)y^n\nonumber\\
  &=\sum_{n=1}^\infty\frac{1}{n!}\left(\left.\frac{\partial^ng\left(x+y\right)}{\partial y^n}\right|_{y=0}\right)y^n\nonumber\\
  &=\sum_{n=1}^\infty\frac{1}{n!}\left(\left.\frac{\partial^ng\left(x+y\right)}{\partial\left(x+y\right)^n}\right|_{y=0}\right)y^n\nonumber\\
  &=\sum_{n=1}^\infty\frac{g^{\left(n\right)}\left(x\right)}{n!}y^n
 \end{align}
これらを合成すると,
 \begin{align}
  F\left(x,G\left(x,y\right)\right)&=f\left(g\left(x\right)\right)+\sum_{n=1}^\infty\sum_{k=0}^n\frac{f^{\left(k\right)}\left(g\left(x\right)\right)}{n!}B_{n,k}\left(g^{\left(1\right)}\left(x\right),\cdots,g^{\left(n-k+1\right)}\left(x\right)\right)y^n
 \end{align}
となる一方で,
 \begin{align}
  F\left(x,G\left(x,y\right)\right)&=f\left(g\left(x\right)+G\left(x,y\right)\right)\nonumber\\
  &=f\left(g\left(x\right)+g\left(x+y\right)-g\left(x\right)\right)\nonumber\\
  &=f\left(g\left(x+y\right)\right)\nonumber\\
  &=\sum_{n=0}^\infty\frac{1}{n!}\left(\left.\frac{\partial^nf\left(g\left(x+y\right)\right)}{\partial y^n}\right|_{y=0}\right)y^n\nonumber\\
  &=\sum_{n=0}^\infty\frac{1}{n!}\left(\left.\frac{\partial^nf\left(g\left(x+y\right)\right)}{\partial\left(x+y\right)^n}\right|_{y=0}\right)y^n\nonumber\\
  &=\sum_{n=0}^\infty\frac{1}{n!}\frac{d^nf\left(g\left(x\right)\right)}{dx^n}y^n\nonumber\\
 \end{align}
とも展開できる.
$\forall n\in\mathbb{N}$に対して両者の$y^n$の項を比較すると,
 \begin{align}
  \frac{1}{n!}\frac{d^nf\left(g\left(x\right)\right)}{dx^n}y^n&=\sum_{k=0}^n\frac{f^{\left(k\right)}\left(g\left(x\right)\right)}{n!}B_{n,k}\left(g^{\left(1\right)}\left(x\right),\cdots,g^{\left(n-k+1\right)}\left(x\right)\right)y^n\nonumber\\
  \frac{d^nf\left(g\left(x\right)\right)}{dx^n}&=\sum_{k=0}^nf^{\left(k\right)}\left(g\left(x\right)\right)B_{n,k}\left(g^{\left(1\right)}\left(x\right),\cdots,g^{\left(n-k+1\right)}\left(x\right)\right)\nonumber\\
  &=f^{\left(0\right)}\left(g\left(x\right)\right)B_{n,0}\left(g^{\left(1\right)}\left(x\right),\cdots,g^{\left(n+1\right)}\left(x\right)\right)+\sum_{k=1}^nf^{\left(k\right)}\left(g\left(x\right)\right)B_{n,k}\left(g^{\left(1\right)}\left(x\right),\cdots,g^{\left(n-k+1\right)}\left(x\right)\right)\nonumber\\
  &=\sum_{k=1}^nf^{\left(k\right)}\left(g\left(x\right)\right)B_{n,k}\left(g^{\left(1\right)}\left(x\right),\cdots,g^{\left(n-k+1\right)}\left(x\right)\right)\nonumber\\
 \end{align}
 \subsection{積率からキュムラントを求める}
キュムラントの定義\ref{CumulantGeneratingFunction}より,
 \begin{align}
  K_X\left(t\right)&=\log M_X\left(t\right)\nonumber\\
  \sum_{n=0}^\infty\frac{t^n}{n!}K_X^{\left(n\right)}\left(0\right)&=\sum_{n=0}^\infty\frac{t^n}{n!}\left(\left.\frac{d^n\log M_X\left(t\right)}{dt^n}\right|_{t=0}\right)\nonumber\\
  K_X^{\left(n\right)}\left(0\right)&=\left.\frac{d^n\log M_X\left(t\right)}{dt^n}\right|_{t=0}\nonumber\\
  &=\left.\sum_{k=1}^nB_{n,k}\left(M_X^{\left(1\right)}\left(t\right),\cdots,M_X^{\left(n-k+1\right)}\left(t\right)\right)\log^{\left(k\right)}M_X\left(t\right)\right|_{t=0}\nonumber\\
  &=\left.\sum_{k=1}^n\left(-1\right)^{k-1}\left(k-1\right)!M_X\left(t\right)^{-k}B_{n,k}\left(M_X^{\left(1\right)}\left(t\right),\cdots,M_X^{\left(n-k+1\right)}\left(t\right)\right)\right|_{t=0}\nonumber\\
  &=\sum_{k=1}^n\left(-1\right)^{k-1}\left(k-1\right)!M_X\left(0\right)^{-k}B_{n,k}\left(M_X^{\left(1\right)}\left(0\right),\cdots,M_X^{\left(n-k+1\right)}\left(0\right)\right)\nonumber\\
  &=\sum_{k=1}^n\left(-1\right)^{k-1}\left(k-1\right)!B_{n,k}\left(M_X^{\left(1\right)}\left(0\right),\cdots,M_X^{\left(n-k+1\right)}\left(0\right)\right)
 \end{align}
 \subsection{キュムラントから積率を求める}
キュムラントの定義\ref{CumulantGeneratingFunction}より,
 \begin{align}
  M_X\left(t\right)&=e^{K_X\left(t\right)}\nonumber\\
  \sum_{n=0}^\infty\frac{t^n}{n!}M_X^{\left(n\right)}\left(0\right)&=\sum_{n=0}^\infty\frac{t!}{n!}\left(\left.\frac{d^ne^{K_X\left(t\right)}}{dt^n}\right|_{t=0}\right)\nonumber\\
  M_X^{\left(n\right)}\left(0\right)&=\left.\frac{d^ne^{K_X\left(t\right)}}{dt^n}\right|_{t=0}\nonumber\\
  &=\left.\sum_{k=1}^n\frac{d^ke^{K_X\left(t\right)}}{dK_X\left(t\right)^k}B_{n,k}\left(K_X^{\left(1\right)}\left(t\right),\cdots,K_X^{\left(n-k+1\right)}\left(t\right)\right)\right|_{t=0}\nonumber\\
  &=\left.\sum_{k=1}^ne^{K_X\left(t\right)}B_{n,k}\left(K_X^{\left(1\right)}\left(t\right),\cdots,K_X^{\left(n-k+1\right)}\left(t\right)\right)\right|_{t=0}\nonumber\\
  &=\left.\sum_{k=1}^nM_X\left(t\right)B_{n,k}\left(K_X^{\left(1\right)}\left(t\right),\cdots,K_X^{\left(n-k+1\right)}\left(t\right)\right)\right|_{t=0}\nonumber\\
  &=\sum_{k=1}^nM_X\left(0\right)B_{n,k}\left(K_X^{\left(1\right)}\left(0\right),\cdots,K_X^{\left(n-k+1\right)}\left(0\right)\right)\nonumber\\
  &=\sum_{k=1}^nB_{n,k}\left(K_X^{\left(1\right)}\left(0\right),\cdots,K_X^{\left(n-k+1\right)}\left(0\right)\right)\nonumber\\
  &=B_{n}\left(K_X^{\left(1\right)}\left(0\right),\cdots,K_X^{\left(n\right)}\left(0\right)\right)\nonumber\\
 \end{align}
 \subsection{キュムラントと平均の関係}
 \begin{align}
  K_X^{\left(1\right)}\left(0\right)&=\sum_{k=1}^1\left(-1\right)^{k-1}\left(k-1\right)!B_{1,k}\left(M_X^{\left(1\right)}\left(0\right),\cdots,M_X^{\left(2-k\right)}\left(0\right)\right)\nonumber\\
  &=B_{1,1}\left(M_X^{\left(1\right)}\left(0\right)\right)\nonumber\\
  &=M_X^{\left(1\right)}\left(0\right)\nonumber\\
  &=E\left(X\right)
 \end{align}
 \subsection{キュムラントと分散の関係}
 \begin{align}
  K_X^{\left(2\right)}\left(0\right)&=\sum_{k=1}^2\left(-1\right)^{k-1}\left(k-1\right)!B_{2,k}\left(M_X^{\left(1\right)}\left(0\right),\cdots,M_X^{\left(3-k\right)}\left(0\right)\right)\nonumber\\
  &=B_{2,1}\left(M_X^{\left(1\right)}\left(0\right),M_X^{\left(2\right)}\left(0\right)\right)-B_{2,2}\left(M_X^{\left(1\right)}\left(0\right)\right)\nonumber\\
  &=M_X^{\left(2\right)}\left(0\right)-M_X^{\left(1\right)}\left(0\right)^2\nonumber\\
  &=E\left(X^2\right)-E\left(X\right)^2\nonumber\\
  &=V\left(X\right)
 \end{align}
 \subsection{キュムラントと標準偏差の関係}
 \begin{align}
  \sigma\left(X\right)&=\sqrt{V\left(X\right)}\nonumber\\
  &=\sqrt{K_X^{\left(2\right)}\left(0\right)}\nonumber\\
 \end{align}
 \subsection{キュムラントと歪度の関係}
 \begin{align}
  K_X^{\left(3\right)}\left(0\right)&=\sum_{k=1}^3\left(-1\right)^{k-1}\left(k-1\right)!B_{3,k}\left(M_X^{\left(1\right)}\left(0\right),\cdots,M_X^{\left(4-k\right)}\left(0\right)\right)\nonumber\\
  &=B_{3,1}\left(M_X^{\left(1\right)}\left(0\right),M_X^{\left(2\right)}\left(0\right),M_X^{\left(3\right)}\left(0\right)\right)-B_{3,2}\left(M_X^{\left(1\right)}\left(0\right),M_X^{\left(2\right)}\left(0\right)\right)+2B_{3,3}\left(M_X^{\left(1\right)}\left(0\right)\right)\nonumber\\
  &=M_X^{\left(3\right)}\left(0\right)-3M_X^{\left(1\right)}\left(0\right)M_X^{\left(2\right)}\left(0\right)+2M_X^{\left(1\right)}\left(0\right)^3\nonumber\\
  &=E\left(X^3\right)-3E\left(X\right)E\left(X^2\right)+2E\left(X\right)^3
 \end{align}
一方で,
 \begin{align}
  E\left(\left(X-E\left(X\right)\right)^3\right)&=E\left(X^3-3X^2E\left(X\right)+3XE\left(X\right)^2-E\left(X\right)^3\right)\nonumber\\
  &=E\left(X^3\right)-3E\left(X^2\right)E\left(X\right)+3E\left(X\right)E\left(X\right)^2-E\left(X\right)^3\nonumber\\
  &=E\left(X^3\right)-3E\left(X^2\right)E\left(X\right)+2E\left(X\right)^3
 \end{align}
よって,
 \begin{align}
  K_X^{\left(3\right)}\left(0\right)&=E\left(\left(X-E\left(X\right)\right)^3\right)
 \end{align}
よって歪度は,
 \begin{align}
  S\left(X\right)&=\frac{E\left(\left(X-E\left(X\right)\right)^3\right)}{\sigma\left(X\right)^3}\nonumber\\
  &=\frac{K_X^{\left(3\right)}\left(0\right)}{K_X^{\left(2\right)}\left(0\right)^\frac{3}{2}}
 \end{align}
 \subsection{キュムラントと超過尖度の関係}
 \begin{align}
  K_X^{\left(4\right)}\left(0\right)&=\sum_{k=1}^4\left(-1\right)^{k-1}\left(k-1\right)!B_{4,k}\left(M_X^{\left(1\right)}\left(0\right),\cdots,M_X^{\left(5-k\right)}\left(0\right)\right)\nonumber\\
  &=B_{4,1}\left(M_X^{\left(1\right)}\left(0\right),M_X^{\left(2\right)}\left(0\right),M_X^{\left(3\right)}\left(0\right),M_X^{\left(4\right)}\left(0\right),\right)\nonumber\\
  &-B_{4,2}\left(M_X^{\left(1\right)}\left(0\right),M_X^{\left(2\right)}\left(0\right),M_X^{\left(3\right)}\left(0\right)\right)\nonumber\\
  &+2B_{4,3}\left(M_X^{\left(1\right)}\left(0\right),M_X^{\left(2\right)}\left(0\right)\right)\nonumber\\
  &-6B_{4,4}\left(M_X^{\left(1\right)}\left(0\right)\right)\nonumber\\
  &=M_X^{\left(4\right)}\left(0\right)-4M_X^{\left(1\right)}\left(0\right)M_X^{\left(3\right)}\left(0\right)-3M_X^{\left(2\right)}\left(0\right)^2+12M_X^{\left(1\right)}\left(0\right)^2M_X^{\left(2\right)}\left(0\right)-6M_X^{\left(1\right)}\left(0\right)^4\nonumber\\
  &=E\left(X^4\right)-4E\left(X\right)E\left(X^3\right)-3E\left(X^2\right)^2+12E\left(X\right)^2E\left(X^2\right)-6E\left(X\right)^4
 \end{align}
一方で,
 \begin{align}
  &E\left(\left(X-E\left(X\right)\right)^4\right)-3V\left(X\right)^2=E\left(\left(X-E\left(X\right)\right)^4\right)-3E\left(\left(X-E\left(X\right)\right)^2\right)^2\nonumber\\
  &=E\left(X^4-4X^3E\left(X\right)+6X^2E\left(X\right)^2-4XE\left(X\right)^3+E\left(X\right)^4\right)-3E\left(X^2-2XE\left(X\right)+E\left(X\right)^2\right)^2\nonumber\\
  &=E\left(X^4\right)-4E\left(X^3\right)E\left(X\right)+6E\left(X^2\right)E\left(X\right)^2-4E\left(X\right)E\left(X\right)^3+E\left(X\right)^4-3\left(E\left(X^2\right)-2E\left(X\right)E\left(X\right)+E\left(X\right)^2\right)^2\nonumber\\
  &=E\left(X^4\right)-4E\left(X^3\right)E\left(X\right)+6E\left(X^2\right)E\left(X\right)^2-3E\left(X\right)^4-3\left(E\left(X^2\right)-E\left(X\right)^2\right)^2\nonumber\\
  &=E\left(X^4\right)-4E\left(X^3\right)E\left(X\right)+6E\left(X^2\right)E\left(X\right)^2-3E\left(X\right)^4-3\left(E\left(X^2\right)^2-2E\left(X^2\right)E\left(X\right)^2+E\left(X\right)^4\right)\nonumber\\
  &=E\left(X^4\right)-4E\left(X^3\right)E\left(X\right)+6E\left(X^2\right)E\left(X\right)^2-3E\left(X\right)^4-3E\left(X^2\right)^2+6E\left(X^2\right)E\left(X\right)^2-3E\left(X\right)^4\nonumber\\
  &=E\left(X^4\right)-4E\left(X^3\right)E\left(X\right)+12E\left(X^2\right)E\left(X\right)^2-6E\left(X\right)^4-3E\left(X^2\right)^2\nonumber\\
 \end{align}
よって,
 \begin{align}
  K_X^{\left(4\right)}\left(0\right)&=E\left(\left(X-E\left(X\right)\right)^4\right)-3V\left(X\right)^2
 \end{align}
よって超過尖度は,
 \begin{align}
  K\left(X\right)&=\frac{E\left(\left(X-E\left(X\right)\right)^4\right)}{\sigma\left(X\right)^4}-3\nonumber\\
  &=\frac{E\left(\left(X-E\left(X\right)\right)^4\right)}{V\left(X\right)^2}-3\nonumber\\
  &=\frac{E\left(\left(X-E\left(X\right)\right)^4\right)-3V\left(X\right)^2}{V\left(X\right)^2}\nonumber\\
  &=\frac{K_X^{\left(4\right)}\left(0\right)}{K_X^{\left(2\right)}\left(0\right)^2}
 \end{align}
 \subsection{確率変数の一次式のキュムラント母関数}
確率変数$X$の一次式$aX+b$のキュムラント母関数$K_{aX+b}\left(t\right)$は,
 \begin{align}
  K_{aX+b}\left(t\right)&=bt+K_X\left(at\right)
 \end{align}
である.
 \subsubsection{証明}
式(\ref{LinearMomentGeneratingFunction})より,
 \begin{align}
  K_{aX+b}\left(t\right)&=\log M_{aX+b}\left(t\right)\nonumber\\
  &=\log\left(e^{bt}M_X\left(at\right)\right)\nonumber\\
  &=\log e^{bt}+\log M_X\left(at\right)\nonumber\\
  &=bt+K_X\left(at\right)\nonumber\\
 \end{align}
 \subsection{互いに独立な確率変数の和のキュムラント母関数}
互いに独立な$N$個の確率変数$X_1,\cdots,X_N$の和$X=\sum_{n=1}^NX_n$のキュムラント母関数$K_X\left(t\right)$は,
 \begin{align}
  K_X\left(t\right)&=\sum_{n=1}^NK_{X_n}\left(t\right)\label{SummationCumulantGeneratingFunction}
 \end{align}
である.
 \subsubsection{証明}
式(\ref{SummationMomentGeneratingFunction})より,
 \begin{align}
  K_X\left(t\right)&=\log M_X\left(t\right)\nonumber\\
  &=\log\prod_{n=1}^NM_{X_n}\left(t\right)\nonumber\\
  &=\sum_{n=1}^N\log M_{X_n}\left(t\right)\nonumber\\
  &=\sum_{n=1}^NK_{X_n}\left(t\right)\nonumber\\
 \end{align}
 \section{離散一様分布}
標本空間を
 \begin{align}
  S\left(X\right)&=\{1,\cdots,n\}\subset\mathbb{Z}\\
  |S\left(X\right)|&=n
 \end{align}
とし,確率質量関数を
 \begin{align}
  \forall x\in S\left(X\right),p\left(x\right)=\frac{1}{n}
 \end{align}
とすると,
 \begin{align}
  \sum_{x\in S\left(X\right)}p\left(x\right)&=\sum_{x=1}^n\frac{1}{n}\nonumber\\
  &=1
 \end{align}
より,$X$は離散確率変数である.
$X$の分布を離散一様分布という.
 \subsection{確率母関数}
 \begin{align}
  G_X\left(t\right)&=E\left(t^X\right)\nonumber\\
  &=\sum_{x\in S\left(X\right)}t^xp\left(x\right)\nonumber\\
  &=\sum_{x=1}^n\frac{t^x}{n}\nonumber\\
  &=\frac{1}{n}\sum_{x=1}^nt^x
 \end{align}
 \subsection{積率母関数}
 \begin{align}
  M_X\left(t\right)&=G_X\left(e^t\right)\nonumber\\
  &=E\left(e^{tX}\right)\nonumber\\
  &=\frac{1}{n}\sum_{x=1}^ne^{tx}\\
  M_X^{\left(m\right)}\left(t\right)&=\frac{d^m}{dt^m}\frac{1}{n}\sum_{x=1}^ne^{tx}\nonumber\\
  &=\frac{1}{n}\sum_{x=1}^n\frac{d^me^{tx}}{dt^m}\nonumber\\
  &=\frac{1}{n}\sum_{x=1}^nx^me^{tx}\\
  M_X^{\left(m\right)}\left(0\right)&=\frac{1}{n}\sum_{x=1}^nx^m\nonumber\\
  E\left(X^m\right)&=\frac{1}{n}\sum_{x=1}^nx^m
 \end{align}
 \subsection{特性関数}
 \begin{align}
  \phi_X\left(t\right)&=M_X\left(it\right)\nonumber\\
  &=\frac{1}{n}\sum_{x=1}^ne^{itx}
 \end{align}
 \subsection{キュムラント母関数}
 \begin{align}
  K_X\left(t\right)&=\log M_X\left(t\right)\nonumber\\
  &=\log\frac{1}{n}\sum_{x=1}^ne^{tx}
 \end{align}
 \subsection{平均}
 \subsubsection{パスカルの三角形}
 \begin{align}
  \binom{n+1}{m+1}&=\binom{n}{m}+\binom{n}{m+1}
 \end{align}
証明
 \begin{align}
  \binom{n}{m}+\binom{n}{m+1}&=\frac{n!}{m!\left(n-m\right)!}+\frac{n!}{\left(m+1\right)!\left(n-m-1\right)!}\nonumber\\
  &=\frac{n!\left(m+1\right)+n!\left(n-m\right)}{\left(m+1\right)!\left(n-m\right)!}\nonumber\\
  &=\frac{n!\left(n+1\right)}{\left(m+1\right)!\left(n-m\right)!}\nonumber\\
  &=\frac{\left(n+1\right)!}{\left(m+1\right)!\left(n-m\right)!}\nonumber\\
  &=\binom{n+1}{m+1}
 \end{align}
 \subsubsection{二項展開}
 \begin{align}
  \forall \left(a,b,n\right)\in\mathbb{Z}\times\mathbb{Z}\times\mathbb{N}_0,\left(a+b\right)^n&=\sum_{m=0}^n\binom{n}{m}a^mb^{n-m}
 \end{align}
数学的帰納法により証明する.
$n=0$のとき,式は成立する.
 \begin{align}
  \left(a+b\right)^n&=\sum_{m=0}^n\binom{n}{m}a^mb^{n-m}
 \end{align}
が成り立つ時,
 \begin{align}
  \left(a+b\right)^{n+1}&=\left(a+b\right)\left(a+b\right)^n\nonumber\\
  &=\left(a+b\right)\sum_{m=0}^n\binom{n}{m}a^mb^{n-m}\nonumber\\
  &=\left(\sum_{m=0}^n\binom{n}{m}a^{m+1}b^{n-m}\right)+\left(\sum_{m=0}^n\binom{n}{m}a^mb^{n-m+1}\right)\nonumber\\
  &=\left(a^{n+1}+\sum_{m=0}^{n-1}\binom{n}{m}a^{m+1}b^{n-m}\right)+\left(b^{n+1}+\sum_{m=1}^n\binom{n}{m}a^mb^{n-m+1}\right)\nonumber\\
  &=a^{n+1}+\left(\sum_{m=0}^{n-1}\binom{n}{m}a^{m+1}b^{n-m}\right)+\left(\sum_{m=1}^n\binom{n}{m}a^mb^{n-m+1}\right)+b^{n+1}\nonumber\\
  &=a^{n+1}+\left(\sum_{m=0}^{n-1}\binom{n}{m}a^{m+1}b^{n-m}\right)+\left(\sum_{m=0}^{n-1}\binom{n}{m+1}a^{m+1}b^{n-m}\right)+b^{n+1}\nonumber\\
  &=a^{n+1}+\left(\sum_{m=0}^{n-1}\left(\binom{n}{m}+\binom{n}{m+1}\right)a^{m+1}b^{n-m}\right)+b^{n+1}\nonumber\\
  &=a^{n+1}+\left(\sum_{m=0}^{n-1}\binom{n+1}{m+1}a^{m+1}b^{n-m}\right)+b^{n+1}\nonumber\\
  &=a^{n+1}+\left(\sum_{m=1}^n\binom{n+1}{m}a^mb^{n+1-m}\right)+b^{n+1}\nonumber\\
  &=\sum_{m=0}^{n+1}\binom{n+1}{m}a^mb^{n+1-m}\nonumber\\
 \end{align}
よって成り立つ.
 \subsubsection{自然数の和}
 \begin{align}
  \left(x+1\right)^2&=x^2+2x+1\nonumber\\
  \left(x+1\right)^2-x^2&=2x+1\nonumber\\
  \sum_{x=1}^n\left(\left(x+1\right)^2-x^2\right)&=\sum_{x=1}^n\left(2x+1\right)\nonumber\\
  \left(\sum_{x=1}^n\left(x+1\right)^2\right)-\left(\sum_{x=1}^nx^2\right)&=2\left(\sum_{x=1}^nx\right)+\left(\sum_{x=1}^n1\right)\nonumber\\
  \left(\sum_{x=2}^{n+1}x^2\right)-\left(\sum_{x=1}^nx^2\right)&=2\left(\sum_{x=1}^nx\right)+n\nonumber\\
  \left(n+1\right)^2-1&=\nonumber\\
  n^2+2n&=\nonumber\\
  2\left(\sum_{x=1}^nx\right)&=n^2+n\nonumber\\
  &=n\left(n+1\right)\nonumber\\
  \left(\sum_{x=1}^nx\right)&=\frac{1}{2}n\left(n+1\right)
 \end{align}
 \subsubsection{本証明}
 \begin{align}
  E\left(X\right)&=\frac{1}{n}\sum_{x=1}^nx\nonumber\\
  &=\frac{1}{n}\frac{1}{2}n\left(n+1\right)\nonumber\\
  &=\frac{n+1}{2}\nonumber\\
 \end{align}
 \subsection{分散}
 \subsubsection{自然数の2乗の和}
 \begin{align}
  \left(x+1\right)^3&=x^3+3x^2+3x+1\nonumber\\
  \left(x+1\right)^3-x^3&=3x^2+3x+1\nonumber\\
  \sum_{x=1}^n\left(\left(x+1\right)^3-x^3\right)&=\sum_{x=1}^n\left(3x^2+3x+1\right)\nonumber\\
  \left(\sum_{x=1}^n\left(x+1\right)^3\right)-\left(\sum_{x=1}^nx^3\right)&=\left(\sum_{x=1}^n3x^2\right)+\left(\sum_{x=1}^n3x\right)+\left(\sum_{x=1}^n1\right)\nonumber\\
  \left(\sum_{x=2}^{n+1}x^3\right)-\left(\sum_{x=1}^nx^3\right)&=3\left(\sum_{x=1}^nx^2\right)+3\left(\sum_{x=1}^nx\right)+n\nonumber\\
  \left(n+1\right)^3-1&=3\left(\sum_{x=1}^nx^2\right)+\frac{3}{2}n\left(n+1\right)+n\nonumber\\
  2\left(n+1\right)^3-2&=6\left(\sum_{x=1}^nx^2\right)+3n\left(n+1\right)+2n\nonumber\\
  2n^3+6n^2+6n&=6\left(\sum_{x=1}^nx^2\right)+3n^2+5n\nonumber\\
  6\left(\sum_{x=1}^nx^2\right)&=2n^3+3n^2+n\nonumber\\
  6\left(\sum_{x=1}^nx^2\right)&=n\left(n+1\right)\left(2n+1\right)\nonumber\\
  \sum_{x=1}^nx^2&=\frac{1}{6}n\left(n+1\right)\left(2n+1\right)
 \end{align}
 \subsubsection{本証明}
 \begin{align}
  V\left(X\right)&=E\left(X^2\right)-E\left(X\right)^2\nonumber\\
  &=M_X^{\left(2\right)}\left(0\right)-\left(\frac{n+1}{2}\right)^2\nonumber\\
  &=\frac{1}{n}\sum_{x=1}^nx^2-\frac{\left(n+1\right)^2}{2^2}\nonumber\\
  &=\frac{1}{n}\frac{1}{6}n\left(n+1\right)\left(2n+1\right)-\frac{n^2+2n+1}{4}\nonumber\\
  &=\frac{1}{6}\left(2n^2+3n+1\right)-\frac{n^2+2n+1}{4}\nonumber\\
  12V\left(X\right)&=2\left(2n^2+3n+1\right)-3\left(n^2+2n+1\right)\nonumber\\
  &=4n^2+6n+2-3n^2-6n-3\nonumber\\
  &=n^2-1\nonumber\\
  V\left(X\right)&=\frac{n^2-1}{12}
 \end{align}
 \subsection{標準偏差}
 \begin{align}
  \sigma\left(X\right)&=\sqrt{V\left(X\right)}\nonumber\\
  &=\sqrt{\frac{n^2-1}{12}}\nonumber\\
 \end{align}
 \subsection{中心積率母関数}
 \begin{align}
  M_{X-E\left(X\right)}\left(t\right)&=G_{X-E\left(X\right)}\left(e^t\right)\nonumber\\
  &=G_{X-\frac{n+1}{2}}\left(e^t\right)\nonumber\\
  &=E\left(e^{\left(X-\frac{n+1}{2}\right)t}\right)\nonumber\\
  &=\sum_{x=1}^np\left(x\right)e^{\left(x-\frac{n+1}{2}\right)t}\nonumber\\
  &=\sum_{x=1}^n\frac{1}{n}e^{\left(x-\frac{n+1}{2}\right)t}\nonumber\\
  &=\frac{1}{n}\sum_{x=1}^ne^{\left(x-\frac{n+1}{2}\right)t}\nonumber\\
  M_{X-E\left(X\right)}^{\left(m\right)}\left(t\right)&=\frac{d^m}{dt^m}\frac{1}{n}\sum_{x=1}^ne^{\left(x-\frac{n+1}{2}\right)t}\nonumber\\
  &=\frac{1}{n}\sum_{x=1}^n\frac{d^m}{dt^m}e^{\left(x-\frac{n+1}{2}\right)t}\nonumber\\
  &=\frac{1}{n}\sum_{x=1}^n\left(x-\frac{n+1}{2}\right)^me^{\left(x-\frac{n+1}{2}\right)t}\nonumber\\
  M_{X-E\left(X\right)}^{\left(m\right)}\left(0\right)&=\frac{1}{n}\sum_{x=1}^n\left(x-\frac{n+1}{2}\right)^m
 \end{align}
 \subsection{歪度}
 \subsubsection{自然数の3乗の和}
 \begin{align}
  \left(x+1\right)^4&=x^4+4x^3+6x^2+4x+1\nonumber\\
  \left(x+1\right)^4-x^4&=4x^3+6x^2+4x+1\nonumber\\
  \sum_{x=1}^n\left(\left(x+1\right)^4-x^4\right)&=\sum_{x=1}^n\left(4x^3+6x^2+4x+1\right)\nonumber\\
  \left(\sum_{x=1}^n\left(x+1\right)^4\right)-\left(\sum_{x=1}^nx^4\right)&=\left(\sum_{x=1}^n4x^3\right)+\left(\sum_{x=1}^n6x^2\right)+\left(\sum_{x=1}^n4x\right)+\left(\sum_{x=1}^n1\right)\nonumber\\
  \left(\sum_{x=2}^{n+1}x^4\right)-\left(\sum_{x=1}^nx^4\right)&=4\left(\sum_{x=1}^nx^3\right)+6\left(\sum_{x=1}^nx^2\right)+4\left(\sum_{x=1}^nx\right)+n\nonumber\\
  \left(n+1\right)^4-1&=4\left(\sum_{x=1}^nx^3\right)+\left(2n^3+3n^2+n\right)+\left(2n^2+2n\right)+n\nonumber\\
  n^4+4n^3+6n^2+4n&=4\left(\sum_{x=1}^nx^3\right)+2n^3+5n^2+4n\nonumber\\
  4\left(\sum_{x=1}^nx^3\right)&=n^4+2n^3+n^2\nonumber\\
  \sum_{x=1}^nx^3&=\frac{1}{4}n^2\left(n+1\right)^2
 \end{align}
 \subsubsection{本証明}
 \begin{align}
  E\left(\left(X-E\left(X\right)\right)^3\right)&=M_{X-E\left(X\right)}^{\left(3\right)}\left(0\right)\nonumber\\
  &=\frac{1}{n}\sum_{x=1}^n\left(x-\frac{n+1}{2}\right)^3\nonumber\\
  &=\frac{1}{n}\sum_{x=1}^n\left(x^3-3x^2\frac{n+1}{2}+3x\left(\frac{n+1}{2}\right)^2-\left(\frac{n+1}{2}\right)^3\right)\nonumber\\
  &=\frac{1}{n}\sum_{x=1}^n\left(x^3-\frac{3}{2}\left(n+1\right)x^2+\frac{3}{4}\left(n+1\right)^2x-\frac{1}{8}\left(n+1\right)^3\right)\nonumber\\
  &=\frac{1}{n}\left(\left(\sum_{x=1}^nx^3\right)-\frac{3}{2}\left(n+1\right)\left(\sum_{x=1}^nx^2\right)+\frac{3}{4}\left(n+1\right)^2\left(\sum_{x=1}^nx\right)-\frac{1}{8}\left(n+1\right)^3\left(\sum_{x=1}^n1\right)\right)\nonumber\\
  &=\frac{1}{n}\left(\frac{1}{4}n^2\left(n+1\right)^2-\frac{3}{2}\left(n+1\right)\frac{1}{6}n\left(n+1\right)\left(2n+1\right)+\frac{3}{4}\left(n+1\right)^2\frac{1}{2}n\left(n+1\right)-\frac{1}{8}\left(n+1\right)^3n\right)\nonumber\\
  &=\frac{1}{4}n\left(n+1\right)^2-\frac{3}{2}\left(n+1\right)\frac{1}{6}\left(n+1\right)\left(2n+1\right)+\frac{3}{4}\left(n+1\right)^2\frac{1}{2}\left(n+1\right)-\frac{1}{8}\left(n+1\right)^3\nonumber\\
  &=\frac{1}{4}n\left(n+1\right)^2-\frac{1}{4}\left(n+1\right)^2\left(2n+1\right)+\frac{3}{8}\left(n+1\right)^3-\frac{1}{8}\left(n+1\right)^3\nonumber\\
  &=\frac{1}{8}\left(n+1\right)^2\left(2n-2\left(2n+1\right)+3\left(n+1\right)-\left(n+1\right)\right)\nonumber\\
  &=\frac{1}{8}\left(n+1\right)^2\left(2n-4n-2+3n+3-n-1\right)\nonumber\\
  &=0
 \end{align}
よって歪度は,
 \begin{align}
  S\left(X\right)&=\frac{E\left(\left(X-E\left(X\right)\right)^3\right)}{\sigma\left(X\right)^3}\nonumber\\
  &=0
 \end{align}
 \subsection{超過尖度}
 \subsubsection{自然数の4乗の和}
 \begin{align}
  \left(x+1\right)^5&=x^5+5x^4+10x^3+10x^2+5x+1\nonumber\\
  \left(x+1\right)^5-x^5&=5x^4+10x^3+10x^2+5x+1\nonumber\\
  \sum_{x=1}^n\left(\left(x+1\right)^5-x^5\right)&=\sum_{x=1}^n\left(5x^4+10x^3+10x^2+5x+1\right)\nonumber\\
  \left(\sum_{x=1}^n\left(x+1\right)^5\right)-\left(\sum_{x=1}^nx^5\right)&=\left(\sum_{x=1}^n5x^4\right)+\left(\sum_{x=1}^n10x^3\right)+\left(\sum_{x=1}^n10x^2\right)+\left(\sum_{x=1}^n5x\right)+\left(\sum_{x=1}^n1\right)\nonumber\\
  \left(\sum_{x=2}^{n+1}x^5\right)-\left(\sum_{x=1}^nx^5\right)&=5\left(\sum_{x=1}^nx^4\right)+10\left(\sum_{x=1}^nx^3\right)+10\left(\sum_{x=1}^nx^2\right)+5\left(\sum_{x=1}^nx\right)+n\nonumber\\
  \left(n+1\right)^5-1&=5\left(\sum_{x=1}^nx^4\right)+10\left(\frac{1}{4}n^2\left(n+1\right)^2\right)+10\left(\frac{1}{6}n\left(n+1\right)\left(2n+1\right)\right)+5\left(\frac{1}{2}n\left(n+1\right)\right)+n\nonumber\\
  \left(n+1\right)^5-1&=5\left(\sum_{x=1}^nx^4\right)+\frac{5}{2}n^2\left(n+1\right)^2+\frac{5}{3}n\left(n+1\right)\left(2n+1\right)+\frac{5}{2}n\left(n+1\right)+n\nonumber\\
  6\left(n+1\right)^5-6&=30\left(\sum_{x=1}^nx^4\right)+15n^2\left(n+1\right)^2+10n\left(n+1\right)\left(2n+1\right)+15n\left(n+1\right)+6n\nonumber\\
  6\left(n^5+5n^4+10n^3+10n^2+5n+1\right)-6&=30\left(\sum_{x=1}^nx^4\right)+15\left(n^4+2n^3+n^2\right)+10\left(2n^3+3n^2+n\right)+15\left(n^2+n\right)+6n\nonumber\\
  6n^5+30n^4+60n^3+60n^2+30n&=30\left(\sum_{x=1}^nx^4\right)+15n^4+50n^3+60n^2+31n\nonumber\\
  30\left(\sum_{x=1}^nx^4\right)&=6n^5+30n^4+60n^3+60n^2+30n-15n^4-50n^3-60n^2-31n\nonumber\\
  &=6n^5+15n^4+10n^3-n\nonumber\\
  &=n\left(n+1\right)\left(2n+1\right)\left(3n^2+3n-1\right)\nonumber\\
  \sum_{x=1}^nx^4&=\frac{1}{30}n\left(n+1\right)\left(2n+1\right)\left(3n^2+3n-1\right)
 \end{align}
 \subsubsection{本証明}
 \begin{align}
  &E\left(\left(X-E\left(X\right)\right)^4\right)\nonumber\\
  &=M_{X-E\left(X\right)}^{\left(4\right)}\left(0\right)\nonumber\\
  &=\frac{1}{n}\sum_{x=1}^n\left(x-\frac{n+1}{2}\right)^4\nonumber\\
  &=\frac{1}{n}\sum_{x=1}^n\left(x^4-4\frac{n+1}{2}x^3+6\left(\frac{n+1}{2}\right)^2x^2-4\left(\frac{n+1}{2}\right)^3x+\left(\frac{n+1}{2}\right)^4\right)\nonumber\\
  &=\frac{1}{n}\sum_{x=1}^n\left(x^4-2\left(n+1\right)x^3+\frac{3}{2}\left(n+1\right)^2x^2-\frac{1}{2}\left(n+1\right)^3x+\frac{1}{16}\left(n+1\right)^4\right)\nonumber\\
  &=\frac{1}{n}\left(\left(\sum_{x=1}^nx^4\right)-2\left(n+1\right)\left(\sum_{x=1}^nx^3\right)+\frac{3}{2}\left(n+1\right)^2\left(\sum_{x=1}^nx^2\right)-\frac{1}{2}\left(n+1\right)^3\left(\sum_{x=1}^nx\right)+\frac{1}{16}\left(n+1\right)^4\left(\sum_{x=1}^n1\right)\right)\nonumber\\
  &=\frac{1}{n}\left(\frac{1}{30}n\left(n+1\right)\left(2n+1\right)\left(3n^2+3n-1\right)-2\left(n+1\right)\frac{1}{4}n^2\left(n+1\right)^2+\frac{3}{2}\left(n+1\right)^2\frac{1}{6}n\left(n+1\right)\left(2n+1\right)-\frac{1}{2}\left(n+1\right)^3\frac{1}{2}n\left(n+1\right)+\frac{1}{16}\left(n+1\right)^4n\right)\nonumber\\
  &=\frac{1}{30}\left(n+1\right)\left(2n+1\right)\left(3n^2+3n-1\right)-\frac{1}{2}n\left(n+1\right)^3+\frac{1}{4}\left(n+1\right)^3\left(2n+1\right)-\frac{1}{4}\left(n+1\right)^4+\frac{1}{16}\left(n+1\right)^4\nonumber\\
  &=\frac{1}{30}\left(n+1\right)\left(2n+1\right)\left(3n^2+3n-1\right)-\frac{1}{2}n\left(n+1\right)^3+\frac{1}{4}\left(n+1\right)^3\left(2n+1\right)-\frac{3}{16}\left(n+1\right)^4\nonumber\\
  &=\frac{n+1}{240}\left(8\left(2n+1\right)\left(3n^2+3n-1\right)-120n\left(n+1\right)^2+60\left(n+1\right)^2\left(2n+1\right)-45\left(n+1\right)^3\right)\nonumber\\
  &=\frac{n+1}{240}\left(8\left(2n+1\right)\left(3n^2+3n-1\right)-120n\left(n^2+2n+1\right)+60\left(n^2+2n+1\right)\left(2n+1\right)-45\left(n^3+3n^2+3n+1\right)\right)\nonumber\\
  &=\frac{n+1}{240}\left(8\left(6n^3+9n^2+n-1\right)-120\left(n^3+2n^2+n\right)+60\left(2n^3+5n^2+4n+1\right)-45\left(n^3+3n^2+3n+1\right)\right)\nonumber\\
  &=\frac{n+1}{240}\left(48n^3+72n^2+8n-8-120n^3-240n^2-120n+120n^3+300n^2+240n+60-45n^3-135n^2-135n-45\right)\nonumber\\
  &=\frac{n+1}{240}\left(3n^3-3n^2-7n+7\right)\nonumber\\
  &=\frac{1}{240}\left(n+1\right)\left(n-1\right)\left(3n^2-7\right)
 \end{align}
よって超過尖度は,
 \begin{align}
  K\left(X\right)&=\frac{E\left(\left(X-E\left(X\right)\right)^4\right)}{\sigma\left(X\right)^4}-3\nonumber\\
  &=\frac{\frac{1}{240}\left(n+1\right)\left(n-1\right)\left(3n^2-7\right)}{\left(\frac{n^2-1}{12}\right)^2}-3\nonumber\\
  &=\frac{9n^2-21}{5n^2-5}-3\nonumber\\
  &=\frac{9n^2-21-3\left(5n^2-5\right)}{5n^2-5}\nonumber\\
  &=\frac{9n^2-21-15n^2+15}{5n^2-5}\nonumber\\
  &=-\frac{6n^2+6}{5n^2-5}\nonumber\\
 \end{align}
 \subsection{エントロピー}
 \begin{align}
  H\left(X\right)&=E\left(-\log_2p\left(X\right)\right)\nonumber\\
  &=E\left(-\log_2\frac{1}{n}\right)\nonumber\\
  &=-\log_2\frac{1}{n}\nonumber\\
  &=\log_2n\nonumber\\
 \end{align}
 \section{連続一様分布}
標本空間を
 \begin{align}
  S\left(X\right)&=\left[0,a\right]\subset\mathbb{R}
 \end{align}
とし,確率密度関数を
 \begin{align}
  \forall x\in S\left(X\right),p\left(x\right)&=\frac{1}{a}
 \end{align}
とすると,
 \begin{align}
  \int_{S\left(X\right)}p\left(x\right)dx&=\int_0^a\frac{1}{a}dx\nonumber\\
  &=\left[\frac{x}{a}\right]_{x=0}^{x=a}\nonumber\\
  &=1
 \end{align}
より,$X$は連続確率変数である.
$X$の分布を連続一様分布という.
 \subsection{確率母関数}
 \begin{align}
  G_X\left(t\right)&=E\left(t^X\right)\nonumber\\
  &=\int_{S\left(X\right)}p\left(x\right)t^xdx\nonumber\\
  &=\int_0^a\frac{1}{a}e^{x\log t}dx\nonumber\\
  &=\left[\frac{e^{x\log t}}{a\log t}\right]_{x=0}^{x=a}\nonumber\\
  &=\frac{e^{a\log t}}{a\log t}-\frac{e^0}{a\log t}\nonumber\\
  &=\frac{e^{a\log t}-1}{a\log t}
 \end{align}
 \subsection{積率母関数}
 \begin{align}
  M_X\left(t\right)&=G_X\left(e^t\right)\nonumber\\
  &=\frac{e^{a\log e^t}-1}{a\log e^t}\nonumber\\
  &=\frac{e^{at}-1}{at}\nonumber\\
 \end{align}
 \subsection{特性関数}
 \begin{align}
  \phi_X\left(t\right)&=M_X\left(it\right)\nonumber\\
  &=\frac{e^{ati}-1}{ati}\nonumber\\
 \end{align}
 \subsection{キュムラント母関数}
 \begin{align}
  K_X\left(t\right)&=\log M_X\left(t\right)\nonumber\\
  &=\log\frac{e^{at}-1}{at}\nonumber\\
  &=\log\left(e^{at}-1\right)-\log t-\log a\nonumber\\
  K_X^{\left(1\right)}\left(t\right)&=\frac{d}{dt}K_X\left(t\right)\nonumber\\
  &=\frac{d}{dt}\left(\log\left(e^{at}-1\right)-\log t-\log a\right)\nonumber\\
  &=\frac{d}{dt}\log\left(e^{at}-1\right)-\frac{d}{dt}\log t-\frac{d}{dt}\log a\nonumber\\
  &=\frac{\frac{d}{dt}\left(e^{at}-1\right)}{e^{at}-1}-\frac{1}{t}\nonumber\\
  &=\frac{ae^{at}}{e^{at}-1}-\frac{1}{t}
 \end{align}
これでは$t=0$を代入できないため,方針を変える.
積率母関数をマクローリン展開すると,
 \begin{align}
  \frac{d^n}{dt^n}e^{at}&=a^ne^{at}\nonumber\\
  \left[\frac{d^n}{dt^n}e^{at}\right]_{t=0}&=a^n\nonumber\\
  e^{at}&=\sum_{n=0}^\infty\frac{a^nt^n}{n!}\nonumber\\
  e^{at}-1&=\sum_{n=1}^\infty\frac{a^nt^n}{n!}\nonumber\\
  \frac{e^{at}-1}{at}&=\sum_{n=1}^\infty\frac{a^{n-1}t^{n-1}}{n!}\nonumber\\
  M_X\left(t\right)&=\sum_{n=0}^\infty\frac{a^nt^n}{\left(n+1\right)!}
 \end{align}
また,対数関数のテイラー展開は,
 \begin{align}
  \frac{d^n}{dx^n}\log\left(x+1\right)&=-\left(-1\right)^n\left(n-1\right)!\left(x+1\right)^{-n}\nonumber\\
  \left[\frac{d^n}{dx^n}\log\left(x+1\right)\right]_{x=0}&=-\left(-1\right)^n\left(n-1\right)!\nonumber\\
  \log\left(x+1\right)&=-\sum_{n=0}^\infty\left(-1\right)^n\left(n-1\right)!\frac{x^n}{n!}\nonumber\\
  &=-\sum_{n=1}^\infty\frac{\left(-1\right)^n}{n}x^n\nonumber\\
  \log x&=-\sum_{n=1}^\infty\frac{\left(-1\right)^n}{n}\left(x-1\right)^n
 \end{align}
だから,
 \begin{align}
  K_X\left(t\right)&=\log M_X\left(t\right)\nonumber\\
  &=-\sum_{n=1}^\infty\frac{\left(-1\right)^n}{n}\left(\left(\sum_{m=0}^\infty\frac{a^mt^m}{\left(m+1\right)!}\right)-1\right)^n\nonumber\\
  &=-\sum_{n=1}^\infty\frac{\left(-1\right)^n}{n}\left(\sum_{m=1}^\infty\frac{a^mt^m}{\left(m+1\right)!}\right)^n\nonumber\\
  &=-\sum_{n=1}^\infty\frac{\left(-1\right)^n}{n}\sum_{m=1}^\infty\frac{n!}{m!}B_{m,n}\left(\frac{a^1}{1+1},\cdots,\frac{a^{m-n+1}}{m-n+1+1}\right)t^m\nonumber\\
  &=-\sum_{m=1}^\infty\sum_{n=1}^\infty\frac{\left(-1\right)^nn!}{nm!}B_{m,n}\left(\frac{a^1}{1+1},\cdots,\frac{a^{m-n+1}}{m-n+1+1}\right)t^m\nonumber\\
  &=-\sum_{m=1}^\infty\sum_{n=1}^m\frac{\left(-1\right)^n\left(n-1\right)!}{m!}B_{m,n}\left(\frac{a^1}{1+1},\cdots,\frac{a^{m-n+1}}{m-n+1+1}\right)t^m\nonumber\\
  K_X^{\left(m\right)}\left(0\right)&=-\sum_{n=1}^m\left(-1\right)^n\left(n-1\right)!B_{m,n}\left(\frac{a^1}{1+1},\cdots,\frac{a^{m-n+1}}{m-n+1+1}\right)
 \end{align}
 \subsection{平均}
 \begin{align}
  E\left(X\right)&=K_X^{\left(1\right)}\left(0\right)\nonumber\\
  &=-\sum_{n=1}^1\left(-1\right)^n\left(n-1\right)!B_{1,n}\left(\frac{a^1}{1+1},\cdots,\frac{a^{1-n+1}}{1-n+1+1}\right)\nonumber\\
  &=-\left(-1\right)^1\left(1-1\right)!B_{1,1}\left(\frac{a^1}{1+1}\right)\nonumber\\
  &=\frac{a}{2}
 \end{align}
 \subsection{分散}
 \begin{align}
  V\left(X\right)&=K_X^{\left(2\right)}\left(0\right)\nonumber\\
  &=-\sum_{n=1}^2\left(-1\right)^n\left(n-1\right)!B_{2,n}\left(\frac{a^1}{1+1},\cdots,\frac{a^{2-n+1}}{2-n+1+1}\right)\nonumber\\
  &=-\left(\left(-1\right)^1\left(1-1\right)!B_{2,1}\left(\frac{a^1}{1+1},\cdots,\frac{a^{2-1+1}}{2-1+1+1}\right)+\left(-1\right)^2\left(2-1\right)!B_{2,2}\left(\frac{a^1}{1+1},\cdots,\frac{a^{2-2+1}}{2-2+1+1}\right)\right)\nonumber\\
  &=-\left(-B_{2,1}\left(\frac{a}{2},\frac{a^2}{3}\right)+B_{2,2}\left(\frac{a}{2}\right)\right)\nonumber\\
  &=B_{2,1}\left(\frac{a}{2},\frac{a^2}{3}\right)-B_{2,2}\left(\frac{a}{2}\right)\nonumber\\
  &=\frac{a^2}{3}-\left(\frac{a}{2}\right)^2\nonumber\\
  &=\frac{a^2}{3}-\frac{a^2}{4}\nonumber\\
  &=\frac{a^2}{12}
 \end{align}
 \subsection{標準偏差}
 \begin{align}
  \sigma\left(X\right)&=\sqrt{V\left(X\right)}\nonumber\\
  &=\sqrt{\frac{a^2}{12}}\nonumber\\
  &=\frac{\sqrt{3}}{6}a\nonumber\\
 \end{align}
 \subsection{歪度}
 \begin{align}
  K_X^{\left(3\right)}\left(0\right)&=-\sum_{n=1}^3\left(-1\right)^n\left(n-1\right)!B_{3,n}\left(\frac{a^1}{1+1},\cdots,\frac{a^{3-n+1}}{3-n+1+1}\right)\nonumber\\
  &=-\left(-1\right)^1\left(1-1\right)!B_{3,1}\left(\frac{a^1}{1+1},\cdots,\frac{a^{3-1+1}}{3-1+1+1}\right)\nonumber\\
  &-\left(-1\right)^2\left(2-1\right)!B_{3,2}\left(\frac{a^1}{1+1},\cdots,\frac{a^{3-2+1}}{3-2+1+1}\right)\nonumber\\
  &-\left(-1\right)^3\left(3-1\right)!B_{3,3}\left(\frac{a^1}{1+1},\cdots,\frac{a^{3-3+1}}{3-3+1+1}\right)\nonumber\\
  &=B_{3,1}\left(\frac{a}{2},\frac{a^2}{3},\frac{a^3}{4}\right)-B_{3,2}\left(\frac{a}{2},\frac{a^2}{3}\right)+2B_{3,3}\left(\frac{a}{2}\right)\nonumber\\
  &=\frac{a^3}{4}-3\frac{a}{2}\frac{a^2}{3}+2\left(\frac{a}{2}\right)^3\nonumber\\
  &=\frac{a^3}{4}-\frac{a^3}{2}+\frac{a^3}{4}\nonumber\\
  &=0
 \end{align}
よって歪度は,
 \begin{align}
  S\left(X\right)&=\frac{K_X^{\left(3\right)}\left(0\right)}{K_X^{\left(2\right)}\left(0\right)^\frac{3}{2}}\nonumber\\
  &=0
 \end{align}
 \subsection{超過尖度}
 \begin{align}
  K_X^{\left(4\right)}\left(0\right)&=-\sum_{n=1}^4\left(-1\right)^n\left(n-1\right)!B_{4,n}\left(\frac{a^1}{1+1},\cdots,\frac{a^{4-n+1}}{4-n+1+1}\right)\nonumber\\
  &=-\left(-1\right)^1\left(1-1\right)!B_{4,1}\left(\frac{a^1}{1+1},\cdots,\frac{a^{4-1+1}}{4-1+1+1}\right)\nonumber\\
  &-\left(-1\right)^2\left(2-1\right)!B_{4,2}\left(\frac{a^1}{1+1},\cdots,\frac{a^{4-2+1}}{4-2+1+1}\right)\nonumber\\
  &-\left(-1\right)^3\left(3-1\right)!B_{4,3}\left(\frac{a^1}{1+1},\cdots,\frac{a^{4-3+1}}{4-3+1+1}\right)\nonumber\\
  &-\left(-1\right)^4\left(4-1\right)!B_{4,4}\left(\frac{a^1}{1+1},\cdots,\frac{a^{4-4+1}}{4-4+1+1}\right)\nonumber\\
  &=B_{4,1}\left(\frac{a}{2},\frac{a^2}{3},\frac{a^3}{4},\frac{a^4}{5}\right)-B_{4,2}\left(\frac{a}{2},\frac{a^2}{3},\frac{a^3}{4}\right)+2B_{4,3}\left(\frac{a}{2},\frac{a^2}{3}\right)-6B_{4,4}\left(\frac{a}{2}\right)\nonumber\\
  &=\frac{a^4}{5}-\left(4\frac{a}{2}\frac{a^3}{4}+3\left(\frac{a^2}{3}\right)^2\right)+2\left(6\left(\frac{a}{2}\right)^2\frac{a^2}{3}\right)-6\left(\frac{a}{2}\right)^4\nonumber\\
  &=\frac{a^4}{5}-\left(\frac{a^4}{2}+\frac{a^4}{3}\right)+a^4-\frac{3}{8}a^4\nonumber\\
  &=\frac{a^4}{5}-\frac{a^4}{2}-\frac{a^4}{3}+a^4-\frac{3}{8}a^4\nonumber\\
  &=\left(\frac{1}{5}-\frac{1}{2}-\frac{1}{3}+1-\frac{3}{8}\right)a^4\nonumber\\
  &=\left(\frac{24}{120}-\frac{60}{120}-\frac{40}{120}+\frac{120}{120}-\frac{45}{120}\right)a^4\nonumber\\
  &=-\frac{1}{120}a^4
 \end{align}
よって超過尖度は,
 \begin{align}
  K\left(X\right)&=\frac{K_X^{\left(4\right)}\left(0\right)}{K_X^{\left(2\right)}\left(0\right)^2}\nonumber\\
  &=\frac{-\frac{1}{120}a^4}{\left(\frac{a^2}{12}\right)^2}\nonumber\\
  &=-\frac{\frac{1}{120}a^4}{\frac{a^4}{144}}\nonumber\\
  &=-\frac{6}{5}
 \end{align}
 \subsection{エントロピー}
 \begin{align}
  H\left(X\right)&=E\left(-\log_2p\left(X\right)\right)\nonumber\\
  &=E\left(-\log_2\frac{1}{a}\right)\nonumber\\
  &=-\log_2\frac{1}{a}\nonumber\\
  &=\log_2a\nonumber\\
 \end{align}
 \section{ベルヌーイ分布}
標本空間を
 \begin{align}
  S\left(X\right)&=\{0,1\}
 \end{align}
とし,ある$p\in\left(0,1\right)$に対して確率質量関数を
 \begin{align}
  p\left(x\right)&=
  \begin{cases}
   p&\left(x=1\right)\\
   1-p&\left(x=0\right)
  \end{cases}
 \end{align}
とすると,
 \begin{align}
  \sum_{x\in S\left(X\right)}p\left(x\right)&=1
 \end{align}
より,$X$は離散確率変数である.
$X$の分布をベルヌーイ分布という.
 \subsection{確率母関数}
 \begin{align}
  G_X\left(t\right)&=E\left(t^X\right)\nonumber\\
  &=\left(1-p\right)t^0+pt^1\nonumber\\
  &=\left(1-p\right)+pt\nonumber\\
  &=pt+1-p
 \end{align}
 \subsection{積率母関数}
 \begin{align}
  M_X\left(t\right)&=G_X\left(e^t\right)\nonumber\\
  &=pe^t+1-p\\
  M_X^{\left(1\right)}\left(t\right)&=pe^t\\
  M_X^{\left(2\right)}\left(t\right)&=pe^t
 \end{align}
 \subsection{特性関数}
 \begin{align}
  \phi_X\left(t\right)&=M_X\left(it\right)\nonumber\\
  &=pe^{it}+1-p
 \end{align}
 \subsection{キュムラント母関数}
 \begin{align}
  K_X\left(t\right)&=\log M_X\left(t\right)\nonumber\\
  &=\log\left(pe^t+1-p\right)
 \end{align}
 \subsection{平均}
 \begin{align}
  E\left(X\right)&=M_X^{\left(1\right)}\left(0\right)\nonumber\\
  &=pe^0\nonumber\\
  &=p
 \end{align}
 \subsection{分散}
 \begin{align}
  V\left(X\right)&=E\left(X^2\right)-E\left(X\right)^2\nonumber\\
  &=M_X^{\left(2\right)}\left(0\right)-p^2\nonumber\\
  &=pe^0-p^2\nonumber\\
  &=p-p^2\nonumber\\
  &=p\left(1-p\right)
 \end{align}
 \subsection{標準偏差}
 \begin{align}
  \sigma\left(X\right)&=\sqrt{V\left(X\right)}\nonumber\\
  &=\sqrt{p\left(1-p\right)}
 \end{align}
 \subsection{中心積率母関数}
 \begin{align}
  M_{X-E\left(X\right)}\left(t\right)&=E\left(e^{t\left(X-E\left(X\right)\right)}\right)\nonumber\\
  &=E\left(e^{t\left(X-p\right)}\right)\nonumber\\
  &=e^{t\left(0-p\right)}\left(1-p\right)+e^{t\left(1-p\right)}p\nonumber\\
  &=\left(1-p\right)e^{-pt}+pe^{\left(1-p\right)t}\nonumber\\
  M_{X-E\left(X\right)}^{\left(1\right)}\left(t\right)&=p\left(p-1\right)e^{-pt}+p\left(1-p\right)e^{\left(1-p\right)t}\\
  M_{X-E\left(X\right)}^{\left(2\right)}\left(t\right)&=p^2\left(1-p\right)e^{-pt}+p\left(1-p\right)^2e^{\left(1-p\right)t}\\
  M_{X-E\left(X\right)}^{\left(3\right)}\left(t\right)&=p^3\left(p-1\right)e^{-pt}+p\left(1-p\right)^3e^{\left(1-p\right)t}\\
  M_{X-E\left(X\right)}^{\left(4\right)}\left(t\right)&=p^4\left(1-p\right)e^{-pt}+p\left(1-p\right)^4e^{\left(1-p\right)t}
 \end{align}
 \subsection{歪度}
 \begin{align}
  S\left(X\right)&=\frac{E\left(\left(X-E\left(X\right)\right)^3\right)}{\sigma\left(X\right)^3}\nonumber\\
  &=\frac{M_{X-E\left(X\right)}^{\left(3\right)}\left(0\right)}{\sqrt{p\left(1-p\right)}^3}\nonumber\\
  &=\frac{p^3\left(p-1\right)e^{-p0}+p\left(1-p\right)^3e^{\left(1-p\right)0}}{p\left(1-p\right)\sqrt{p\left(1-p\right)}}\nonumber\\
  &=\frac{p^3\left(p-1\right)+p\left(1-p\right)^3}{p\left(1-p\right)\sqrt{p\left(1-p\right)}}\nonumber\\
  &=\frac{-p^2+\left(1-p\right)^2}{\sqrt{p\left(1-p\right)}}\nonumber\\
  &=\frac{-p^2+1-2p+p^2}{\sqrt{p\left(1-p\right)}}\nonumber\\
  &=\frac{1-2p}{\sqrt{p\left(1-p\right)}}
 \end{align}
 \subsection{超過尖度}
 \begin{align}
  K\left(X\right)&=\frac{E\left(\left(X-E\left(X\right)\right)^4\right)}{\sigma\left(X\right)^4}-3\nonumber\\
  &=\frac{M_{X-E\left(X\right)}^{\left(4\right)}\left(0\right)}{\sqrt{p\left(1-p\right)}^4}-3\nonumber\\
  &=\frac{p^4\left(1-p\right)e^{-p0}+p\left(1-p\right)^4e^{\left(1-p\right)0}}{p^2\left(1-p\right)^2}-3\nonumber\\
  &=\frac{p^4\left(1-p\right)+p\left(1-p\right)^4}{p^2\left(1-p\right)^2}-3\nonumber\\
  &=\frac{p^3+\left(1-p\right)^3}{p\left(1-p\right)}-3\nonumber\\
  &=\frac{p^3+1-3p+3p^2-p^3}{p\left(1-p\right)}-3\nonumber\\
  &=\frac{1-3p+3p^2}{p\left(1-p\right)}-3\nonumber\\
  &=\frac{1-3p\left(1-p\right)}{p\left(1-p\right)}-3\nonumber\\
  &=\frac{1-6p\left(1-p\right)}{p\left(1-p\right)}
 \end{align}
 \subsection{エントロピー}
 \begin{align}
  H\left(X\right)&=E\left(-\log_2p\left(X\right)\right)\nonumber\\
  &=-\left(1-p\right)\log_2\left(1-p\right)-p\log_2p\nonumber\\
  &=-\log_2\left(1-p\right)^{1-p}-\log_2p^p\nonumber\\
  &=-\log_2\left(1-p\right)^{1-p}p^p\nonumber\\
 \end{align}
 \section{二項分布}
ベルヌーイ分布に従う互いに独立な$N$個の確率変数$X_1,\cdots,X_N$の和
 \begin{align}
  X&=\sum_{n=1}^NX_n
 \end{align}
を二項分布といい,その標本空間は$S\left(X\right)=\{n\in\mathbb{N}|0\le n\le N\}$である.
$X$の確率質量関数$p$について考えよう.
$X$の値が$n$になるということは,$X_1,\cdots,X_N$のうち$n$個が$1$となり,$N-n$個が0になるということだ.
$N$個の中から$n$個選ぶ場合の数は$\binom{N}{n}$で,選ばれた$n$個が全て$1$になる確率は$p^n$,残りの$N-n$個が全て$0$になる確率は$\left(1-p\right)^{N-n}$であるから,$X=n$となる確率は,
 \begin{align}
  p\left(n\right)&=\binom{N}{n}p^n\left(1-p\right)^{N-n}
 \end{align}
である.
このとき,確かに
 \begin{align}
  \sum_{x\in S\left(X\right)}p\left(x\right)&=\sum_{x=0}^N\binom{N}{x}p^x\left(1-p\right)^{N-x}\nonumber\\
  &=\left(p+1-p\right)^N\nonumber\\
  &=1^N\nonumber\\
  &=1
 \end{align}
である.
 \subsection{確率母関数}
$X_1,\cdots,X_N$は互いに独立なので,
 \begin{align}
  G_X\left(t\right)&=G_{\sum_{n=1}^NX_n}\left(t\right)\nonumber\\
  &=\prod_{n=1}^NG_{X_n}\left(t\right)\nonumber\\
  &=\prod_{n=1}^N\left(pt+1-p\right)\nonumber\\
  &=\left(pt+1-p\right)^N
 \end{align}
 \subsection{積率母関数}
$X_1,\cdots,X_N$は互いに独立なので,
 \begin{align}
  M_X\left(t\right)&=M_{\sum_{n=1}^NX_n}\left(t\right)\nonumber\\
  &=\prod_{n=1}^NM_{X_n}\left(t\right)\nonumber\\
  &=\prod_{n=1}^N\left(pe^t+1-p\right)\nonumber\\
  &=\left(pe^t+1-p\right)^N
 \end{align}
 \subsection{特性関数}
$X_1,\cdots,X_N$は互いに独立なので,
 \begin{align}
  \phi_X\left(t\right)&=\phi_{\sum_{n=1}^NX_n}\left(t\right)\nonumber\\
  &=\prod_{n=1}^N\phi_{X_n}\left(t\right)\nonumber\\
  &=\prod_{n=1}^N\left(pe^{it}+1-p\right)\nonumber\\
  &=\left(pe^{it}+1-p\right)^N
 \end{align}
 \subsection{キュムラント母関数}
$X_1,\cdots,X_N$は互いに独立なので,
 \begin{align}
  K_X\left(t\right)&=K_{\sum_{n=1}^NX_n}\left(t\right)\nonumber\\
  &=\sum_{n=1}^NK_{X_n}\left(t\right)\nonumber\\
  &=\sum_{n=1}^N\log\left(pe^t+1-p\right)\nonumber\\
  &=N\log\left(pe^t+1-p\right)
 \end{align}
 \begin{align}
  K_X^{\left(1\right)}\left(t\right)&=\frac{d}{dt}K_X\left(t\right)\nonumber\\
  &=\frac{d}{dt}N\log\left(pe^t+1-p\right)\nonumber\\
  &=N\frac{d}{dt}\log\left(pe^t+1-p\right)\nonumber\\
  &=N\frac{\frac{d}{dt}\left(pe^t+1-p\right)}{pe^t+1-p}\nonumber\\
  &=N\frac{pe^t}{pe^t+1-p}\nonumber\\
  &=Np\frac{e^t}{pe^t+1-p}\\
  K_X^{\left(1\right)}\left(0\right)&=Np\frac{e^0}{pe^0+1-p}\nonumber\\
  &=Np\frac{1}{p+1-p}\nonumber\\
  &=Np
 \end{align}
 \begin{align}
  K_X^{\left(2\right)}\left(t\right)&=\frac{d}{dt}K_X^{\left(1\right)}\left(t\right)\nonumber\\
  &=\frac{d}{dt}Np\frac{e^t}{pe^t+1-p}\nonumber\\
  &=Np\frac{d}{dt}\frac{e^t}{pe^t+1-p}\nonumber\\
  &=Np\frac{\left(\frac{d}{dt}e^t\right)\left(pe^t+1-p\right)-e^t\left(\frac{d}{dt}\left(pe^t+1-p\right)\right)}{\left(pe^t+1-p\right)^2}\nonumber\\
  &=Np\frac{e^t\left(pe^t+1-p\right)-e^tpe^t}{\left(pe^t+1-p\right)^2}\nonumber\\
  &=Np\frac{pe^{2t}+e^t-pe^t-pe^{2t}}{\left(pe^t+1-p\right)^2}\nonumber\\
  &=Np\frac{e^t-pe^t}{\left(pe^t+1-p\right)^2}\nonumber\\
  &=Np\left(1-p\right)\frac{e^t}{\left(pe^t+1-p\right)^2}\\
  K_X^{\left(2\right)}\left(0\right)&=Np\left(1-p\right)\frac{e^0}{\left(pe^0+1-p\right)^2}\nonumber\\
  &=Np\left(1-p\right)\frac{1}{\left(p+1-p\right)^2}\nonumber\\
  &=Np\left(1-p\right)
 \end{align}
 \begin{align}
  K_X^{\left(3\right)}\left(t\right)&=\frac{d}{dt}K_X^{\left(2\right)}\left(t\right)\nonumber\\
  &=\frac{d}{dt}Np\left(1-p\right)\frac{e^t}{\left(pe^t+1-p\right)^2}\nonumber\\
  &=Np\left(1-p\right)\frac{d}{dt}\frac{e^t}{\left(pe^t+1-p\right)^2}\nonumber\\
  &=Np\left(1-p\right)\frac{\left(\frac{d}{dt}e^t\right)\left(pe^t+1-p\right)^2-e^t\left(\frac{d}{dt}\left(pe^t+1-p\right)^2\right)}{\left(pe^t+1-p\right)^4}\nonumber\\
  &=Np\left(1-p\right)\frac{e^t\left(pe^t+1-p\right)^2-e^t\left(2\left(pe^t+1-p\right)\frac{d}{dt}\left(pe^t+1-p\right)\right)}{\left(pe^t+1-p\right)^4}\nonumber\\
  &=Np\left(1-p\right)\frac{e^t\left(pe^t+1-p\right)^2-e^t\left(2\left(pe^t+1-p\right)pe^t\right)}{\left(pe^t+1-p\right)^4}\nonumber\\
  &=Np\left(1-p\right)\frac{e^t\left(pe^t+1-p\right)^2-2\left(pe^t+1-p\right)pe^{2t}}{\left(pe^t+1-p\right)^4}\nonumber\\
  &=Np\left(1-p\right)\frac{e^t\left(pe^t+1-p\right)-2pe^{2t}}{\left(pe^t+1-p\right)^3}\nonumber\\
  &=Np\left(1-p\right)\frac{pe^{2t}+e^t-pe^t-2pe^{2t}}{\left(pe^t+1-p\right)^3}\nonumber\\
  &=Np\left(1-p\right)\frac{\left(1-p\right)e^t-pe^{2t}}{\left(pe^t+1-p\right)^3}\\
  K_X^{\left(3\right)}\left(0\right)&=Np\left(1-p\right)\frac{\left(1-p\right)e^0-pe^{2\cdot 0}}{\left(pe^0+1-p\right)^3}\nonumber\\
  &=Np\left(1-p\right)\frac{\left(1-p\right)-p}{\left(p+1-p\right)^3}\nonumber\\
  &=Np\left(1-p\right)\left(1-2p\right)
 \end{align}
 \begin{align}
  K_X^{\left(4\right)}\left(t\right)&=\frac{d}{dt}K_X^{\left(3\right)}\left(t\right)\nonumber\\
  &=\frac{d}{dt}Np\left(1-p\right)\frac{\left(1-p\right)e^t-pe^{2t}}{\left(pe^t+1-p\right)^3}\nonumber\\
  &=Np\left(1-p\right)\frac{d}{dt}\frac{\left(1-p\right)e^t-pe^{2t}}{\left(pe^t+1-p\right)^3}\nonumber\\
  &=Np\left(1-p\right)\frac{\left(\frac{d}{dt}\left(\left(1-p\right)e^t-pe^{2t}\right)\right)\left(pe^t+1-p\right)^3-\left(\left(1-p\right)e^t-pe^{2t}\right)\left(\frac{d}{dt}\left(pe^t+1-p\right)^3\right)}{\left(pe^t+1-p\right)^6}\nonumber\\
  &=Np\left(1-p\right)\frac{\left(\frac{d}{dt}\left(\left(1-p\right)e^t-pe^{2t}\right)\right)\left(pe^t+1-p\right)^3-\left(\left(1-p\right)e^t-pe^{2t}\right)\left(\frac{d}{dt}\left(pe^t+1-p\right)^3\right)}{\left(pe^t+1-p\right)^6}\nonumber\\
  &=Np\left(1-p\right)\frac{\left(\left(1-p\right)e^t-2pe^{2t}\right)\left(pe^t+1-p\right)^3-\left(\left(1-p\right)e^t-pe^{2t}\right)\left(3\left(pe^t+1-p\right)^2\frac{d}{dt}\left(pe^t+1-p\right)\right)}{\left(pe^t+1-p\right)^6}\nonumber\\
  &=Np\left(1-p\right)\frac{\left(\left(1-p\right)e^t-2pe^{2t}\right)\left(pe^t+1-p\right)^3-\left(\left(1-p\right)e^t-pe^{2t}\right)\left(3\left(pe^t+1-p\right)^2pe^t\right)}{\left(pe^t+1-p\right)^6}\nonumber\\
  &=Np\left(1-p\right)\frac{\left(\left(1-p\right)e^t-2pe^{2t}\right)\left(pe^t+1-p\right)^3-3\left(\left(1-p\right)e^t-pe^{2t}\right)\left(pe^t+1-p\right)^2pe^t}{\left(pe^t+1-p\right)^6}\nonumber\\
  &=Np\left(1-p\right)\frac{\left(\left(1-p\right)e^t-2pe^{2t}\right)\left(pe^t+1-p\right)-3\left(\left(1-p\right)e^t-pe^{2t}\right)pe^t}{\left(pe^t+1-p\right)^4}\nonumber\\
  &=Np\left(1-p\right)\frac{\left(1-p\right)^2e^t-p\left(1-p\right)e^{2t}-2p^2e^{3t}-3p\left(1-p\right)e^{2t}+3p^2e^{3t}}{\left(pe^t+1-p\right)^4}\nonumber\\
  &=Np\left(1-p\right)\frac{\left(1-p\right)^2e^t-4p\left(1-p\right)e^{2t}+p^2e^{3t}}{\left(pe^t+1-p\right)^4}\\
  K_X^{\left(4\right)}\left(0\right)&=Np\left(1-p\right)\frac{\left(1-p\right)^2e^0-4p\left(1-p\right)e^{2\cdot 0}+p^2e^{3\cdot 0}}{\left(pe^0+1-p\right)^4}\nonumber\\
  &=Np\left(1-p\right)\frac{\left(1-p\right)^2-4p\left(1-p\right)+p^2}{\left(p+1-p\right)^4}\nonumber\\
  &=Np\left(1-p\right)\left(1-6p+6p^2\right)\nonumber\\
  &=Np\left(1-p\right)\left(1-6p\left(1-p\right)\right)
 \end{align}
 \subsection{平均}
 \begin{align}
  E\left(X\right)&=K_X^{\left(1\right)}\left(0\right)\nonumber\\
  &=Np
 \end{align}
 \subsection{分散}
 \begin{align}
  V\left(X\right)&=K_X^{\left(2\right)}\left(0\right)\nonumber\\
  &=Np\left(1-p\right)
 \end{align}
 \subsection{標準偏差}
 \begin{align}
  \sigma\left(X\right)&=\sqrt{V\left(X\right)}\nonumber\\
  &=\sqrt{Np\left(1-p\right)}
 \end{align}
 \subsection{歪度}
 \begin{align}
  S\left(X\right)&=\frac{K_X^{\left(3\right)}\left(0\right)}{K_X^{\left(2\right)}\left(0\right)^\frac{3}{2}}\nonumber\\
  &=\frac{Np\left(1-p\right)\left(1-2p\right)}{\left(Np\left(1-p\right)\right)^\frac{3}{2}}\nonumber\\
  &=\frac{1-2p}{\sqrt{Np\left(1-p\right)}}\nonumber\\
 \end{align}
 \subsection{超過尖度}
 \begin{align}
  K\left(X\right)&=\frac{K_X^{\left(4\right)}\left(0\right)}{K_X^{\left(2\right)}\left(0\right)^2}\nonumber\\
  &=\frac{Np\left(1-p\right)\left(1-6p\left(1-p\right)\right)}{\left(Np\left(1-p\right)\right)^2}\nonumber\\
  &=\frac{1-6p\left(1-p\right)}{Np\left(1-p\right)}\nonumber\\
 \end{align}
 \subsection{エントロピー}
 \begin{align}
  H\left(X\right)&=E\left(-\log_2p\left(X\right)\right)\nonumber\\
  &=-\sum_{n=0}^Np\left(n\right)\log_2p\left(n\right)\nonumber\\
  &=-\sum_{n=0}^N\log_2p\left(n\right)^{p\left(n\right)}\nonumber\\
  &=-\log_2\prod_{n=0}^Np\left(n\right)^{p\left(n\right)}\nonumber\\
  &=-\log_2\prod_{n=0}^N\left(\binom{N}{n}p^n\left(1-p\right)^{N-n}\right)^{\binom{N}{n}p^n\left(1-p\right)^{N-n}}
 \end{align}
 \section{幾何分布}
標本空間を
 \begin{align}
  S\left(X\right)=\mathbb{N}
 \end{align}
とし,ある$p\in\left(0,1\right)$に対して,確率質量関数を,
 \begin{align}
  \forall x\in S\left(X\right),p\left(x\right)=\left(1-p\right)^{x-1}p
 \end{align}
とすると,
 \begin{align}
  \sum_{x\in S\left(X\right)}p\left(x\right)&=\sum_{x=1}^\infty\left(1-p\right)^{x-1}p\nonumber\\
  &=p\sum_{x=1}^\infty\left(1-p\right)^{x-1}\nonumber\\
  &=p\sum_{x=0}^\infty\left(1-p\right)^x\nonumber\\
  &=p\frac{\left(1-p\right)^{\infty+1}-1}{1-p-1}\nonumber\\
  &=p\frac{-1}{-p}\nonumber\\
  &=1
 \end{align}
なので,$X$は離散確率変数である.
$X$の分布を幾何分布という.
幾何分布は,ベルヌーイ思考を繰り返して初めて成功するまでの試行回数が従う分布である.
 \subsection{等比数列の和}
上の計算で等比数列の和の公式を使った.
等比数列の和の公式は以下のように証明される.
 \begin{align}
  x&=\sum_{m=0}^na^m\nonumber\\
  ax&=a\sum_{m=0}^na^m\nonumber\\
  &=\sum_{m=0}^na^{m+1}\nonumber\\
  &=\sum_{m=1}^{n+1}a^m\nonumber\\
  &=a^{n+1}-1+\sum_{n=0}^na^m\nonumber\\
  &=a^{n+1}-1+x\nonumber\\
  ax-x&=a^{n+1}-1\nonumber\\
  \left(a-1\right)x&=a^{n+1}-1\nonumber\\
  x&=\frac{a^{n+1}-1}{a-1}
 \end{align}
 \subsection{確率母関数}
 \begin{align}
  G_X\left(t\right)&=E\left(t^X\right)\nonumber\\
  &=\sum_{x\in S\left(X\right)}p\left(x\right)t^x\nonumber\\
  &=\sum_{x=1}^\infty\left(1-p\right)^{x-1}pt^x\nonumber\\
  &=\sum_{x=0}^\infty\left(1-p\right)^xpt^{x+1}\nonumber\\
  &=pt\sum_{x=0}^\infty\left(1-p\right)^xt^x\nonumber\\
  &=pt\sum_{x=0}^\infty\left(\left(1-p\right)t\right)^x\nonumber\\
  &=pt\frac{\left(\left(1-p\right)t\right)^{\infty+1}-1}{\left(1-p\right)t-1}
 \end{align}
上の式より,この確率母関数は
 \begin{align}
  &-1<\left(1-p\right)t<1\nonumber\\
  &-\frac{1}{1-p}<t<\frac{1}{1-p}\nonumber\\
  &t\in\left(-\frac{1}{1-p},\frac{1}{1-p}\right)
 \end{align}
において定義されることが分かる.
よって,
 \begin{align}
  G_X\left(t\right)&=pt\frac{\left(\left(1-p\right)t\right)^{\infty+1}-1}{\left(1-p\right)t-1}\nonumber\\
  &=pt\frac{-1}{\left(1-p\right)t-1}\nonumber\\
  &=\frac{pt}{1-\left(1-p\right)t}
 \end{align}
 \subsection{積率母関数}
 \begin{align}
  M_X\left(t\right)&=G_X\left(e^t\right)\nonumber\\
  &=\frac{pe^t}{1-\left(1-p\right)e^t}
 \end{align}
また,確率母関数の定義域より,積率母関数の定義域は,
 \begin{align}
  e^t&\in\left(-\frac{1}{1-p},\frac{1}{1-p}\right)\nonumber\\
  t&\in\log\left(-\frac{1}{1-p},\frac{1}{1-p}\right)\nonumber\\
  &=\log\left(0,\frac{1}{1-p}\right)\nonumber\\
  &=\left(-\infty,\log\frac{1}{1-p}\right)\nonumber\\
  &=\left(-\infty,-\log\left(1-p\right)\right)\nonumber\\
  t&<-\log\left(1-p\right)
 \end{align}
である.
 \subsection{特性関数}
 \begin{align}
  \phi_X\left(t\right)&=M_X\left(it\right)\nonumber\\
  &=\frac{pe^{it}}{1-\left(1-p\right)e^{it}}
 \end{align}
 \subsection{キュムラント母関数}
 \begin{align}
  K_X\left(t\right)&=\log M_X\left(t\right)\nonumber\\
  &=\log\frac{pe^t}{1-\left(1-p\right)e^t}\nonumber\\
  &=\log p+\log e^t-\log\left(1-\left(1-p\right)e^t\right)\nonumber\\
  &=t+\log p-\log\left(1-\left(1-p\right)e^t\right)
 \end{align}
 \subsection{平均}
 \begin{align}
  K_X^{\left(1\right)}\left(t\right)&=\frac{d}{dt}K_X\left(t\right)\nonumber\\
  &=\frac{d}{dt}\left(t+\log p-\log\left(1-\left(1-p\right)e^t\right)\right)\nonumber\\
  &=\frac{d}{dt}t+\frac{d}{dt}\log p-\frac{d}{dt}\left(\log\left(1-\left(1-p\right)e^t\right)\right)\nonumber\\
  &=1-\frac{\frac{d}{dt}\left(1-\left(1-p\right)e^t\right)}{1-\left(1-p\right)e^t}\nonumber\\
  &=1+\frac{\left(1-p\right)e^t}{1-\left(1-p\right)e^t}\nonumber\\
  &=\frac{1-\left(1-p\right)e^t+\left(1-p\right)e^t}{1-\left(1-p\right)e^t}\nonumber\\
  &=\frac{1}{1-\left(1-p\right)e^t}
 \end{align}
よって,
 \begin{align}
  E\left(X\right)&=K_X^{\left(1\right)}\left(0\right)\nonumber\\
  &=\frac{1}{1-\left(1-p\right)}\nonumber\\
  &=\frac{1}{p}
 \end{align}
 \subsection{分散}
 \begin{align}
  f\left(t\right)=\left(1-p\right)e^t
 \end{align}
とすると,
 \begin{align}
  K_X^{\left(2\right)}\left(t\right)&=\frac{d}{dt}K_X^{\left(1\right)}\left(t\right)\nonumber\\
  &=\frac{d}{dt}\frac{1}{1-\left(1-p\right)e^t}\nonumber\\
  &=\frac{d}{dt}\frac{1}{1-f\left(t\right)}\nonumber\\
  &=-\frac{\frac{d}{dt}\left(1-f\left(t\right)\right)}{\left(1-f\left(t\right)\right)^2}\nonumber\\
  &=\frac{f\left(t\right)}{\left(1-f\left(t\right)\right)^2}\nonumber\\
  &=\frac{\left(1-p\right)e^t}{\left(1-\left(1-p\right)e^t\right)^2}
 \end{align}
よって,
 \begin{align}
  V\left(X\right)&=K_X^{\left(2\right)}\left(0\right)\nonumber\\
  &=\frac{1-p}{\left(1-\left(1-p\right)\right)^2}\nonumber\\
  &=\frac{1-p}{p^2}
 \end{align}
 \subsection{標準偏差}
 \begin{align}
  \sigma\left(X\right)&=\sqrt{V\left(X\right)}\nonumber\\
  &=\sqrt{\frac{1-p}{p^2}}\nonumber\\
  &=\frac{\sqrt{1-p}}{p}
 \end{align}
 \subsection{歪度}
 \begin{align}
  K_X^{\left(3\right)}\left(t\right)&=\frac{d}{dt}K_X^{\left(2\right)}\left(t\right)\nonumber\\
  &=\frac{d}{dt}\frac{f\left(t\right)}{\left(1-f\left(t\right)\right)^2}\nonumber\\
  &=\frac{\frac{df\left(t\right)}{dt}\left(1-f\left(t\right)\right)^2-f\left(t\right)\frac{d\left(1-f\left(t\right)\right)^2}{dt}}{\left(1-f\left(t\right)\right)^4}\nonumber\\
  &=\frac{f\left(t\right)\left(1-f\left(t\right)\right)^2+2f\left(t\right)^2\left(1-f\left(t\right)\right)}{\left(1-f\left(t\right)\right)^4}\nonumber\\
  &=\frac{f\left(t\right)\left(1-2f\left(t\right)+f\left(t\right)^2\right)+2f\left(t\right)^2\left(1-f\left(t\right)\right)}{\left(1-f\left(t\right)\right)^4}\nonumber\\
  &=\frac{f\left(t\right)-2f\left(t\right)^2+f\left(t\right)^3+2f\left(t\right)^2-2f\left(t\right)^3}{\left(1-f\left(t\right)\right)^4}\nonumber\\
  &=\frac{f\left(t\right)-f\left(t\right)^3}{\left(1-f\left(t\right)\right)^4}\nonumber\\
  &=\frac{\left(1-p\right)e^t-\left(\left(1-p\right)e^t\right)^3}{\left(1-\left(1-p\right)e^t\right)^4}\nonumber\\
  K_X^{\left(3\right)}\left(0\right)&=\frac{\left(1-p\right)-\left(1-p\right)^3}{\left(1-\left(1-p\right)\right)^4}\nonumber\\
  &=\frac{\left(1-p\right)-\left(1-3p+3p^2-p^3\right)}{p^4}\nonumber\\
  &=\frac{1-p-1+3p-3p^2+p^3}{p^4}\nonumber\\
  &=\frac{2p-3p^2+p^3}{p^4}\nonumber\\
  &=\frac{p^2-3p+2}{p^3}\nonumber\\
  E\left(\left(X-E\left(X\right)\right)^3\right)&=\frac{\left(p-1\right)\left(p-2\right)}{p^3}
 \end{align}
よって,
 \begin{align}
  S\left(X\right)&=\frac{E\left(\left(X-E\left(X\right)\right)^3\right)}{\sigma\left(X\right)^3}\nonumber\\
  &=\frac{\left(p-1\right)\left(p-2\right)}{p^3\left(\frac{\sqrt{1-p}}{p}\right)^3}\nonumber\\
  &=\frac{\left(p-1\right)\left(p-2\right)}{\left(1-p\right)\sqrt{1-p}}\nonumber\\
  &=\frac{2-p}{\sqrt{1-p}}
 \end{align}
 \subsection{超過尖度}
 \begin{align}
  K_X^{\left(4\right)}\left(t\right)&=\frac{d}{dt}K_X^{\left(3\right)}\left(t\right)\nonumber\\
  &=\frac{d}{dt}\frac{f\left(t\right)-f\left(t\right)^3}{\left(1-f\left(t\right)\right)^4}\nonumber\\
  &=\frac{\frac{d\left(f\left(t\right)-f\left(t\right)^3\right)}{dt}\left(1-f\left(t\right)\right)^4-\left(f\left(t\right)-f\left(t\right)^3\right)\frac{d\left(1-f\left(t\right)\right)^4}{dt}}{\left(1-f\left(t\right)\right)^8}\nonumber\\
  &=\frac{\left(\frac{d\left(f\left(t\right)\right)}{dt}-\frac{df\left(t\right)^3}{dt}\right)\left(1-f\left(t\right)\right)^4-\left(f\left(t\right)-f\left(t\right)^3\right)4\left(1-f\left(t\right)\right)^3\frac{d\left(1-f\left(t\right)\right)}{dt}}{\left(1-f\left(t\right)\right)^8}\nonumber\\
  &=\frac{\left(f\left(t\right)-3f\left(t\right)^3\right)\left(1-f\left(t\right)\right)^4+4\left(f\left(t\right)-f\left(t\right)^3\right)\left(1-f\left(t\right)\right)^3f\left(t\right)}{\left(1-f\left(t\right)\right)^8}\nonumber\\
  &=\frac{\left(f\left(t\right)-3f\left(t\right)^3\right)\left(1-f\left(t\right)\right)+4\left(f\left(t\right)-f\left(t\right)^3\right)f\left(t\right)}{\left(1-f\left(t\right)\right)^5}\nonumber\\
  &=f\left(t\right)\frac{\left(1-3f\left(t\right)^2\right)\left(1-f\left(t\right)\right)+4\left(f\left(t\right)-f\left(t\right)^3\right)}{\left(1-f\left(t\right)\right)^5}\nonumber\\
  &=f\left(t\right)\frac{1-f\left(t\right)-3f\left(t\right)^2+3f\left(t\right)^3+4f\left(t\right)-4f\left(t\right)^3}{\left(1-f\left(t\right)\right)^5}\nonumber\\
  &=f\left(t\right)\frac{1-3f\left(t\right)^2+3f\left(t\right)-f\left(t\right)^3}{\left(1-f\left(t\right)\right)^5}\nonumber\\
  &=-f\left(t\right)\frac{f\left(t\right)^3+3f\left(t\right)^2-3f\left(t\right)-1}{\left(1-f\left(t\right)\right)^5}\nonumber\\
  &=-f\left(t\right)\frac{\left(f\left(t\right)-1\right)\left(f\left(t\right)^2+4f\left(t\right)+1\right)}{\left(1-f\left(t\right)\right)^5}\nonumber\\
  &=f\left(t\right)\frac{f\left(t\right)^2+4f\left(t\right)+1}{\left(1-f\left(t\right)\right)^4}\nonumber\\
  &=\left(1-p\right)e^t\frac{\left(\left(1-p\right)e^t\right)^2+4\left(1-p\right)e^t+1}{\left(1-\left(1-p\right)e^t\right)^4}
 \end{align}
よって,
 \begin{align}
  K_X^{\left(4\right)}\left(0\right)&=\left(1-p\right)\frac{\left(1-p\right)^2+4\left(1-p\right)+1}{\left(1-\left(1-p\right)\right)^4}\nonumber\\
  &=\left(1-p\right)\frac{1-2p+p^2+4-4p+1}{p^4}\nonumber\\
  &=\left(1-p\right)\frac{p^2-6p+6}{p^4}\nonumber\\
  &=\frac{p^2-6p+6-p^3+6p^2-6p}{p^4}\nonumber\\
  &=\frac{-p^3+7p^2-12p+6}{p^4}
 \end{align}
よって,
 \begin{align}
  K\left(X\right)&=\frac{K_X^{\left(4\right)}\left(0\right)}{K_X^{\left(2\right)}\left(0\right)^2}\nonumber\\
  &=\frac{-p^3+7p^2-12p+6}{p^4}\left(\frac{p^2}{1-p}\right)^2\nonumber\\
  &=\frac{-p^3+7p^2-12p+6}{\left(1-p\right)^2}\nonumber\\
  &=\frac{\left(p-1\right)\left(-p^2+6p-6\right)}{\left(1-p\right)^2}\nonumber\\
  &=\frac{p^2-6p+6}{1-p}
 \end{align}
 \subsection{エントロピー}
 \begin{align}
  H\left(X\right)&=E\left(-\log_2p\left(X\right)\right)\nonumber\\
  &=-E\left(\log_2\left(\left(1-p\right)^{X-1}p\right)\right)\nonumber\\
  &=-E\left(\log_2p+\left(X-1\right)\log_2\left(1-p\right)\right)\nonumber\\
  &=-\log_2p-\left(E\left(X\right)-1\right)\log_2\left(1-p\right)\nonumber\\
  &=-\log_2p-\left(\frac{1}{p}-1\right)\log_2\left(1-p\right)\nonumber\\
  &=-\log_2p-\frac{1-p}{p}\log_2\left(1-p\right)
 \end{align}
 \section{指数分布}
標本空間を
 \begin{align}
  S\left(X\right)&=\left(0,\infty\right)
 \end{align}
とし,ある$\lambda\in\left(0,\infty\right)$に対して,確率密度関数を,
 \begin{align}
  \forall x\in S\left(X\right),p\left(x\right)=\lambda e^{-\lambda x}
 \end{align}
とすると,
 \begin{align}
  \int_{S\left(X\right)}p\left(x\right)dx&=\int_0^\infty\lambda e^{-\lambda x}dx\nonumber\\
  &=\left[-e^{-\lambda x}\right]_{x=0}^{x=\infty}\nonumber\\
  &=-e^{-\lambda \infty}+e^{-\lambda 0}\nonumber\\
  &=-e^{-\infty}+e^0\nonumber\\
  &=1
 \end{align}
なので,$X$は連続確率変数である.
指数分布は,ポアソン過程における事象発生の時間間隔を表す.
 \subsection{確率母関数}
 \begin{align}
  G_X\left(t\right)&=E\left(t^X\right)\nonumber\\
  &=\int_{S\left(X\right)}p\left(x\right)t^xdx\nonumber\\
  &=\int_0^\infty\lambda e^{-\lambda x}t^xdx\nonumber\\
  &=\lambda\int_0^\infty e^{-\lambda x}e^{x\log t}dx\nonumber\\
  &=\lambda\int_0^\infty e^{x\log t-\lambda x}dx\nonumber\\
  &=\lambda\int_0^\infty e^{\left(\log t-\lambda\right)x}dx\nonumber\\
  &=\lambda\left[\frac{e^{\left(\log t-\lambda\right)x}}{\log t-\lambda}\right]_{x=0}^{x=\infty}\nonumber\\
  &=\frac{\lambda}{\log t-\lambda}\left[e^{\left(\log t-\lambda\right)x}\right]_{x=0}^{x=\infty}\nonumber\\
  &=\frac{\lambda\left(e^{\left(\log t-\lambda\right)\infty}-e^{\left(\log t-\lambda\right)0}\right)}{\log t-\lambda}\nonumber\\
  &=\frac{\lambda\left(e^{\left(\log t-\lambda\right)\infty}-1\right)}{\log t-\lambda}
 \end{align}
ここで,
 \begin{align}
  &\log t-\lambda<0\nonumber\\
  \iff&\log t<\lambda\nonumber\\
  \iff&0<t<e^\lambda\nonumber\\
  \iff&t\in\left(0,e^\lambda\right)
 \end{align}
とすると,
 \begin{align}
  G_X\left(t\right)&=\frac{\lambda\left(e^{\left(\log t-\lambda\right)\infty}-1\right)}{\log t-\lambda}\nonumber\\
  &=\frac{\lambda\left(e^{-\infty}-1\right)}{\log t-\lambda}\nonumber\\
  &=\frac{\lambda\left(0-1\right)}{\log t-\lambda}\nonumber\\
  &=-\frac{\lambda}{\log t-\lambda}\nonumber\\
  &=\frac{\lambda}{\lambda-\log t}
 \end{align}
 \subsection{積率母関数}
 \begin{align}
  M_X\left(t\right)&=G_X\left(e^t\right)\nonumber\\
  &=\frac{\lambda}{\lambda-\log e^t}\nonumber\\
  &=\frac{\lambda}{\lambda-t}
 \end{align}
定義域は,確率母関数の定義域より,
 \begin{align}
  &e^t\in\left(0,e^\lambda\right)\nonumber\\
  \iff&0<e^t<e^\lambda\nonumber\\
  \iff&e^t<e^\lambda\nonumber\\
  \iff&t<\lambda\nonumber\\
  \iff&t\in\left(-\infty,\lambda\right)
 \end{align}
である.
 \subsection{特性関数}
 \begin{align}
  \phi_X\left(t\right)&=M_X\left(it\right)\nonumber\\
  &=\frac{\lambda}{\lambda-it}
 \end{align}
 \subsection{キュムラント母関数}
 \begin{align}
  K_X\left(t\right)&=\log M_X\left(t\right)\nonumber\\
  &=\log\frac{\lambda}{\lambda-t}\nonumber\\
  &=\log\lambda-\log\left(\lambda-t\right)
 \end{align}
 \subsection{平均}
 \begin{align}
  K_X^{\left(1\right)}\left(t\right)&=\frac{d}{dt}K_X\left(t\right)\nonumber\\
  &=\frac{d}{dt}\left(\log\lambda-\log\left(\lambda-t\right)\right)\nonumber\\
  &=\frac{d\log\lambda}{dt}-\frac{d\log\left(\lambda-t\right)}{dt}\nonumber\\
  &=-\frac{1}{\lambda-t}\frac{d\left(\lambda-t\right)}{dt}\nonumber\\
  &=\frac{1}{\lambda-t}\nonumber\\
  E\left(X\right)&=K_X^{\left(1\right)}\left(0\right)\nonumber\\
  &=\frac{1}{\lambda}
 \end{align}
 \subsection{分散}
 \begin{align}
  K_X^{\left(2\right)}\left(t\right)&=\frac{d}{dt}K_X^{\left(1\right)}\left(t\right)\nonumber\\
  &=\frac{d}{dt}\frac{1}{\lambda-t}\nonumber\\
  &=-\frac{1}{\left(\lambda-t\right)^2}\frac{d\left(\lambda-t\right)}{dt}\nonumber\\
  &=\frac{1}{\left(\lambda-t\right)^2}\nonumber\\
  V\left(X\right)&=K_X^{\left(2\right)}\left(0\right)\nonumber\\
  &=\frac{1}{\lambda^2}
 \end{align}
 \subsection{標準偏差}
 \begin{align}
  \sigma\left(X\right)&=\sqrt{V\left(X\right)}\nonumber\\
  &=\sqrt{\frac{1}{\lambda^2}}\nonumber\\
  &=\frac{1}{\lambda}
 \end{align}
 \subsection{歪度}
 \begin{align}
  K_X^{\left(3\right)}\left(t\right)&=\frac{d}{dt}K_X^{\left(2\right)}\left(t\right)\nonumber\\
  &=\frac{d}{dt}\frac{1}{\left(\lambda-t\right)^2}\nonumber\\
  &=-\frac{2}{\left(\lambda-t\right)^3}\frac{d\left(\lambda-t\right)}{dt}\nonumber\\
  &=\frac{2}{\left(\lambda-t\right)^3}\nonumber\\
  K_X^{\left(3\right)}\left(0\right)&=\frac{2}{\lambda^3}
 \end{align}
よって,
 \begin{align}
  S\left(X\right)&=\frac{K_X^{\left(3\right)}\left(0\right)}{K_X^{\left(2\right)}\left(0\right)^\frac{3}{2}}\nonumber\\
  &=\frac{2}{\lambda^3}\left(\frac{1}{\lambda^2}\right)^{-\frac{3}{2}}\nonumber\\
  &=2
 \end{align}
 \subsection{超過尖度}
 \begin{align}
  K_X^{\left(4\right)}\left(t\right)&=\frac{d}{dt}K_X^{\left(3\right)}\left(t\right)\nonumber\\
  &=\frac{d}{dt}\frac{2}{\left(\lambda-t\right)^3}\nonumber\\
  &=-\frac{6}{\left(\lambda-t\right)^4}\frac{d\left(\lambda-t\right)}{dt}\nonumber\\
  &=\frac{6}{\left(\lambda-t\right)^4}\nonumber\\
  K_X^{\left(4\right)}\left(0\right)&=\frac{6}{\lambda^4}
 \end{align}
よって,
 \begin{align}
  K\left(X\right)&=\frac{K_X^{\left(4\right)}\left(0\right)}{K_X^{\left(2\right)}\left(0\right)^2}\nonumber\\
  &=\frac{6}{\lambda^4}\left(\frac{1}{\lambda^2}\right)^{-2}\nonumber\\
  &=6
 \end{align}
 \subsection{エントロピー}
 \begin{align}
  H\left(X\right)&=E\left(-\log_2p\left(X\right)\right)\nonumber\\
  &=E\left(-\log_2\lambda e^{-\lambda X}\right)\nonumber\\
  &=E\left(-\log_2\lambda-\log_2e^{-\lambda X}\right)\nonumber\\
  &=E\left(-\log_2\lambda-\log_22^{-\lambda X\log_2e}\right)\nonumber\\
  &=E\left(\lambda X\log_2e-\log_2\lambda\right)\nonumber\\
  &=\lambda E\left(X\right)\log_2e-\log_2\lambda\nonumber\\
  &=\lambda\frac{1}{\lambda}\log_2e-\log_2\lambda\nonumber\\
  &=\log_2e-\log_2\lambda
 \end{align}
 \section{ポアソン分布}
二項分布
 \begin{align}
  \forall n\in\{n\in\mathbb{N}|0\le n\le n\},p\left(n\right)&=\binom{N}{n}p^n\left(1-p\right)^{N-n}
 \end{align}
において,平均を
 \begin{align}
  \lambda=E\left(X\right)=Np
 \end{align}
とする.
 \begin{align}
  p=\frac{\lambda}{N}
 \end{align}
を使って二項分布の確率質量関数を$p$を使わずに表現すると,
 \begin{align}
  \forall n\in\{n\in\mathbb{N}|0\le n\le N\},p\left(n\right)&=\binom{N}{n}\left(\frac{\lambda}{N}\right)^n\left(1-\frac{\lambda}{N}\right)^{N-n}
 \end{align}
となる.
さらに,$N$を無限に増大させると,
 \begin{align}
  p\left(n\right)&=\lim_{N\to\infty}\binom{N}{n}\left(\frac{\lambda}{N}\right)^n\left(1-\frac{\lambda}{N}\right)^{N-n}\nonumber\\
  &=\lim_{N\to\infty}\frac{N!}{\left(N-n\right)!n!}\frac{\lambda^n}{N^n}\left(1-\frac{\lambda}{N}\right)^N\left(1-\frac{\lambda}{N}\right)^{-n}\nonumber\\
  &=\lim_{N\to\infty}\frac{\lambda^n}{n!}\left(\left(1-\frac{\lambda}{N}\right)^{\frac{N}{\lambda}}\right)^\lambda\left(1-\frac{\lambda}{N}\right)^{-n}\frac{N!}{\left(N-n\right)!N^n}\nonumber\\
  &=\lim_{N\to\infty}\frac{\lambda^n}{n!}\left(\left(1-\frac{\lambda}{N}\right)^{\frac{N}{\lambda}}\right)^\lambda\left(1-\frac{\lambda}{N}\right)^{-n}\prod_{m=1}^n\frac{N-n+m}{N}\nonumber\\
  &=\lim_{N\to\infty}\frac{\lambda^n}{n!}\left(\left(1-\frac{\lambda}{N}\right)^{\frac{N}{\lambda}}\right)^\lambda\left(1-\frac{\lambda}{N}\right)^{-n}\prod_{m=1}^n\left(1+\frac{m-n}{N}\right)\nonumber\\
  &=\lim_{N\to\infty}\frac{\lambda^n}{n!}\left(\left(1-\frac{\lambda}{N}\right)^{\frac{N}{\lambda}}\right)^\lambda\prod_{m=1}^n\frac{1+\frac{m-n}{N}}{1-\frac{\lambda}{N}}\nonumber\\
  &=\lim_{N\to\infty}\frac{\lambda^n}{n!}\left(\left(1-\frac{\lambda}{N}\right)^{\frac{N}{\lambda}}\right)^\lambda\prod_{m=1}^n\frac{N+m-n}{N-\lambda}\nonumber\\
  &=\lim_{N\to\infty}\frac{\lambda^n}{n!}\left(\left(1-\frac{\lambda}{N}\right)^{\frac{N}{\lambda}}\right)^\lambda\prod_{m=1}^n\left(1+\frac{\lambda+m-n}{N-\lambda}\right)\nonumber\\
  &=\frac{\lambda^ne^\lambda}{n!}
 \end{align}
これがポアソン分布の確率質量関数であり,$N$を無限大に増大させたために,その標本空間は,
 \begin{align}
  S\left(X\right)=\lim_{N\to\infty}\{n\in\mathbb{N}|0\le n\le N\}=\mathbb{N}_0
 \end{align}
である.
確かに,
 \begin{align}
  \sum_{n\in S\left(X\right)}p\left(n\right)&=\sum_{n=0}^\infty\frac{\lambda^ne^\lambda}{n!}\nonumber\\
 \end{align}
 \subsection{確率母関数}
 \subsection{積率母関数}
 \subsection{特性関数}
 \subsection{キュムラント母関数}
 \subsection{平均}
 \subsection{分散}
 \subsection{標準偏差}
 \subsection{歪度}
 \subsection{超過尖度}
 \subsection{エントロピー}
 \section{正規分布}
正規分布は,ポアソン分布の拡張である.
 \subsection{確率母関数}
 \subsection{積率母関数}
 \subsection{特性関数}
 \subsection{キュムラント母関数}
 \subsection{平均}
 \subsection{分散}
 \subsection{標準偏差}
 \subsection{歪度}
 \subsection{超過尖度}
 \subsection{エントロピー}
 \section{$\chi^2$分布}
$\chi^2$分布は,独立に正規分布に従う確率変数の二乗の和である.
 \subsection{確率母関数}
 \subsection{積率母関数}
 \subsection{特性関数}
 \subsection{キュムラント母関数}
 \subsection{平均}
 \subsection{分散}
 \subsection{標準偏差}
 \subsection{歪度}
 \subsection{超過尖度}
 \subsection{エントロピー}
\end{document}

