\documentclass[dvipdfmx]{jsarticle}
\usepackage{amsfonts}
\usepackage{amsmath}
\title{確率論}
\author{伊藤 太清}
\date{\today}
\begin{document}
 \maketitle
 \section{離散確率変数}
$n$通りの場合$S=\{0,\cdots,n-1\}$の中からそれぞれの確率$p\left(0\right),\cdots,p\left(0\right)$で定まる変数がある.
 \begin{itemize}
  \item 全ての場合の集合$S=\{0,\cdots,n-1\}$を標本空間という.
  \item $p:S\to\left[0,1\right]$を確率質量関数という.
  \item $P:S\to\left[0,1\right];a\mapsto\sum_{x\in S|x<a}p\left(x\right)$を累積分布関数という.
 \end{itemize}
累積分布関数$P$が
 \begin{align}
P\left(n\right)=1
 \end{align}
を満たすとき,$X=\left(S,P\right)$を離散確率変数という.
累積分布関数$P$が確率質量関数$p$を用いて
 \begin{align}
P\left(a\right)=\sum_{x\in S|x<a}p\left(x\right)
 \end{align}
と表されるのに対し,確率質量関数$p$は累積分布関数$P$を用いて
 \begin{align}
p\left(a\right)&=\left(\sum_{x\in S|x<a+1}p\left(x\right)\right)-\left(\sum_{x\in S|x<a}p\left(x\right)\right)\nonumber\\
&=P\left(a+1\right)-P\left(a\right)
 \end{align}
と表される.
 \subsection{平均}
 \begin{align}
E\left(X\right)=\sum_{x\in S}xp\left(x\right)
 \end{align}
を,離散確率変数$X$の平均という.
 \subsection{分散}
 \begin{align}
V\left(X\right)=\sum_{x\in S}\left(x-E\left(X\right)\right)^2p\left(x\right)
 \end{align}
を,離散確率変数$X$の分散という.
 \subsection{関数の適用}
関数$f:S\to T$を離散確率変数$X=\left(S,P\right)$に適用した離散確率変数を$Y=\left(T,Q\right)$とし,その確率質量関数を$q$とすると,累積分布関数$Q$は,
 \begin{align}
Q\left(b\right)&=\sum_{y\in T|y<b}q\left(y\right)\nonumber\\
&=\sum_{x\in S|f\left(x\right)<b}p\left(x\right)
 \end{align}
であり,確率質量関数$q$は,
 \begin{align}
q\left(b\right)&=Q\left(b+1\right)-Q\left(b\right)\nonumber\\
&=\left(\sum_{x\in S|f\left(x\right)<b+1}p\left(x\right)\right)-\left(\sum_{x\in S|f\left(x\right)<b}p\left(x\right)\right)\nonumber\\
&=\sum_{x\in S|f\left(x\right)=b}p\left(x\right)
 \end{align}
となる.
 \section{連続確率変数}
実数全体の集合$\mathbb{R}$の中から確率的に実数が選択される変数がある.
 \begin{itemize}
  \item 選択されうる実数の集合$S=\mathbb{R}$を標本空間という.
  \item $p:S\to\left[0,\infty\right)$を確率密度関数という.
  \item $P:S\to\left[0,1\right];a\mapsto\int_{-\infty}^ap\left(x\right)dx$を累積分布関数という.
 \end{itemize}
累積分布関数$P$が
 \begin{align}
\lim_{a\to\infty}P\left(a\right)=1
 \end{align}
を満たすとき,$X=\left(S,P\right)$を連続確率変数という.
累積分布関数$P$が確率密度関数$p$を用いて
 \begin{align}
P\left(a\right)=\int_{-\infty}^ap\left(x\right)dx
 \end{align}
と表されるのに対し,確率密度関数$p$は累積分布関数$P$を用いて,
 \begin{align}
p\left(a\right)=P'\left(a\right)
 \end{align}
と表される.
 \subsection{平均}
 \begin{align}
E\left(X\right)=\int_{-\infty}^\infty xp\left(x\right)dx
 \end{align}
を,連続確率変数$X$の平均という.
 \subsection{分散}
 \begin{align}
V\left(X\right)=\int_{-\infty}^\infty \left(x-E\left(X\right)\right)^2p\left(x\right)dx
 \end{align}
を,連続確率変数$X$の分散という.
 \subsection{関数の適用}
関数$f:S\to T$を連続確率変数$X=\left(S,P\right)$に適用した連続確率変数を$Y=\left(T,Q\right)$とし,その確率密度関数を$q$とすると,累積分布関数$Q$は,
 \begin{align}
Q\left(b\right)&=\int_{-\infty}^bq\left(y\right)dy\nonumber\\
&=\int_{x\in S|f\left(x\right)<b}p\left(x\right)\frac{dy}{dx}dx\nonumber\\
&=\int_{x\in S|f\left(x\right)<b}p\left(x\right)f'\left(x\right)dx
 \end{align}
であり,確率密度関数$q$は,
\end{document}

