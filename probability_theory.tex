\documentclass[dvipdfmx]{jsarticle}
\usepackage{amsfonts}
\usepackage{amsmath}
\usepackage{here}
\usepackage{mathrsfs}
\usepackage{tikz}
\usetikzlibrary{intersections, calc, arrows.meta}
\title{確率論}
\author{伊藤 太清}
\date{\today}
\begin{document}
 \maketitle
 \section{離散確率変数}
$n$通りの場合$S=\{0,\cdots,n-1\}$の中からそれぞれの確率$p\left(0\right),\cdots,p\left(n-1\right)$で定まる変数がある.
 \begin{itemize}
  \item 全ての場合の集合$S=\{0,\cdots,n-1\}$を標本空間という.
  \item $p:S\to\left[0,1\right]$を確率質量関数という.
  \item $P:S\to\left[0,1\right];a\mapsto\sum_{x\in S|x<a}p\left(x\right)$を累積分布関数という.
 \end{itemize}
累積分布関数$P$が
 \begin{align}
  P\left(n\right)=1
 \end{align}
を満たすとき,$X=\left(S,P\right)$を離散確率変数という.
累積分布関数$P$が確率質量関数$p$を用いて
 \begin{align}
  P\left(a\right)=\sum_{x\in S|x<a}p\left(x\right)
 \end{align}
と表されるのに対し,確率質量関数$p$は累積分布関数$P$を用いて
 \begin{align}
  p\left(a\right)&=\left(\sum_{x\in S|x<a+1}p\left(x\right)\right)-\left(\sum_{x\in S|x<a}p\left(x\right)\right)\nonumber\\
  &=P\left(a+1\right)-P\left(a\right)
 \end{align}
と表される.
 \subsection{確率}
離散確率変数$X$から取り出された値$a$が,$S$の部分集合$S'$の元である確率は,事象からその事象の確率への関数である確率関数$\mathscr{P}$を用いて,
 \begin{align}
  \mathscr{P}\left(a\in S'\right)&=\sum_{x\in S'}p\left(x\right)
 \end{align}
である.
また,累積分布関数$P$は,確率関数$\mathscr{P}$を用いて,
 \begin{align}
  P\left(x\right)&=\mathscr{P}\left(a<x\right)
 \end{align}
と表される.
 \subsection{平均}
 \begin{align}
  E\left(X\right)=\sum_{x\in S}xp\left(x\right)
 \end{align}
を,離散確率変数$X$の平均という.
 \subsection{分散}
 \begin{align}
  V\left(X\right)=\sum_{x\in S}\left(x-E\left(X\right)\right)^2p\left(x\right)
 \end{align}
を,離散確率変数$X$の分散という.
 \subsection{関数の適用}
関数$f:S\to T$を離散確率変数$X=\left(S,P\right)$に適用した離散確率変数を$Y=\left(T,Q\right)$とし,その確率質量関数を$q$とすると,累積分布関数$Q$は,
 \begin{align}
  Q\left(b\right)&=\sum_{y\in T|y<b}q\left(y\right)\nonumber\\
  &=\sum_{x\in S|f\left(x\right)<b}p\left(x\right)
 \end{align}
であり,確率質量関数$q$は,
 \begin{align}
  q\left(b\right)&=Q\left(b+1\right)-Q\left(b\right)\nonumber\\
  &=\left(\sum_{x\in S|f\left(x\right)<b+1}p\left(x\right)\right)-\left(\sum_{x\in S|f\left(x\right)<b}p\left(x\right)\right)\nonumber\\
  &=\sum_{x\in S|f\left(x\right)=b}p\left(x\right)
 \end{align}
となる.
 \subsubsection{平均}
離散確率変数$Y$の平均は,
 \begin{align}
  E\left(Y\right)&=\sum_{y\in T}yq\left(y\right)\nonumber\\
  &=\sum_{y\in T}y\sum_{x\in S|f\left(x\right)=y}p\left(x\right)\nonumber\\
  &=\sum_{y\in T}\sum_{x\in S|f\left(x\right)=y}yp\left(x\right)\nonumber\\
  &=\sum_{y\in T}\sum_{x\in S|f\left(x\right)=y}f\left(x\right)p\left(x\right)\nonumber\\
  &=\sum_{x\in S|f\left(x\right)\in T}f\left(x\right)p\left(x\right)\nonumber\\
  &=\sum_{x\in S}f\left(x\right)p\left(x\right)
 \end{align}
となる.
 \subsubsection{分散}
離散確率変数$Y$の分散は,
 \begin{align}
  V\left(Y\right)&=\sum{y\in T}\left(y-E\left(Y\right)\right)^2q\left(y\right)\nonumber\\
  &=\sum{y\in T}\left(y-E\left(Y\right)\right)^2\sum_{x\in S|f\left(x\right)=y}p\left(x\right)\nonumber\\
  &=\sum{y\in T}\sum_{x\in S|f\left(x\right)=y}\left(y-E\left(Y\right)\right)^2p\left(x\right)\nonumber\\
  &=\sum{y\in T}\sum_{x\in S|f\left(x\right)=y}\left(f\left(x\right)-E\left(Y\right)\right)^2p\left(x\right)\nonumber\\
  &=\sum_{x\in S|f\left(x\right)\in T}\left(f\left(x\right)-E\left(Y\right)\right)^2p\left(x\right)\nonumber\\
  &=\sum_{x\in S}\left(f\left(x\right)-E\left(Y\right)\right)^2p\left(x\right)
 \end{align}
となる.
 \section{連続確率変数}
実数全体の集合$\mathbb{R}$の中から確率的に実数が選択される変数がある.
 \begin{itemize}
  \item 選択されうる実数の集合$S=\mathbb{R}$を標本空間という.
  \item $p:S\to\left[0,\infty\right)$を確率密度関数という.
  \item $P:S\to\left[0,1\right];a\mapsto\int_{-\infty}^ap\left(x\right)dx$を累積分布関数という.
 \end{itemize}
累積分布関数$P$が
 \begin{align}
  \lim_{a\to\infty}P\left(a\right)=1
 \end{align}
を満たすとき,$X=\left(S,P\right)$を連続確率変数という.
累積分布関数$P$が確率密度関数$p$を用いて
 \begin{align}
  P\left(a\right)=\int_{-\infty}^ap\left(x\right)dx
 \end{align}
と表されるのに対し,確率密度関数$p$は累積分布関数$P$を用いて,
 \begin{align}
  p\left(a\right)=P'\left(a\right)
 \end{align}
と表される.
 \subsection{平均}
 \begin{align}
  E\left(X\right)=\int_{-\infty}^\infty xp\left(x\right)dx
 \end{align}
を,連続確率変数$X$の平均という.
 \subsection{分散}
 \begin{align}
  V\left(X\right)=\int_{-\infty}^\infty \left(x-E\left(X\right)\right)^2p\left(x\right)dx
 \end{align}
を,連続確率変数$X$の分散という.
 \subsection{関数の適用}
微分可能な関数$f:S\to T$を連続確率変数$X=\left(S,P\right)$に適用した連続確率変数を$Y=\left(T,Q\right)$とし,その確率密度関数を$q$とすると,累積分布関数$Q$は,
 \begin{align}
  Q\left(b\right)&=\int_{-\infty}^bq\left(y\right)dy\nonumber\\
  &=\int_{x\in S|f\left(x\right)<b}p\left(x\right)dx
 \end{align}
であり,確率密度関数$q$は,
 \begin{align}
  q\left(b\right)&=Q'\left(b\right)\nonumber\\
  &=\frac{d}{db}\int_{x\in S|f\left(x\right)<b}p\left(x\right)dx
 \end{align}
となる.ここで,上の微分を図を用いて解く.
 \begin{figure}[H]
  \begin{center}
   \begin{tikzpicture}
    \draw(0,0)node[below left]{$O$};
    \draw[->](-0.5,0)--(8,0)node[right]{$x$};
    \draw[->](0,-0.5)--(0,4)node[above]{$y$};
    \draw[domain=-2.5:2.5]plot({\x+4}, {(\x^3-4*\x)/3+2})node[right]{$y=f\left(x\right)$};
    \draw[dashed](2,2)--(2,0)node[below]{$a_0$};
    \draw[dashed](2.2236,2.5)--(2.2236,-0.5)node[below]{$a_0+\Delta a_0$};
    \draw[dashed](4,2)--(4,0)node[below]{$a_1$};
    \draw[dashed](3.61019,2.5)--(3.61019,-1)node[below]{$a_1+\Delta a_1$};
    \draw[dashed](6,2)--(6,0)node[below]{$a_2$};
    \draw[dashed](6.16621,2.5)--(6.16621,-0.5)node[below]{$a_2+\Delta a_2$};
    \draw[dashed](6,2)--(-0.5,2)node[left]{$b$};
    \draw[dashed](6.5,2.5)--(-0.5,2.5)node[left]{$b+\Delta b$};
   \end{tikzpicture}
   \caption{$b$の微小な変化による$\{x\in S|f\left(x\right)<b\}$の変化}\label{IntegrationRangeChange}
  \end{center}
 \end{figure}
図\ref{IntegrationRangeChange}のように,$f\left(x\right)=b$の全ての解を$\{a_0,\cdots,a_N\}=\{x\in S|f\left(x\right)=b\}$とし,$b$の微小な変化$\Delta b$に対する$a_0,\cdots,a_N$の反応をそれぞれ$\Delta a_0,\cdots,\Delta a_N$とする.
ここで,$\Delta b$は十分に小さいから,$\forall n\in\{0,\cdots,N\}$に対して,
 \begin{align}
  f'\left(a_n\right)&=\frac{\Delta b}{\Delta a_n}\nonumber\\
  \Delta a_n&=\frac{\Delta b}{f'\left(a_n\right)}
 \end{align}
である.
ただしここで$f'\left(a_n\right)=0$の場合,それは$a_n$が$f$の極大若しくは極小であることにより,$b$の微小な変化によって$a_n$が消滅または2つに分裂することを意味し,$\Delta a_n$は定義できない.
 \begin{figure}[H]
  \begin{center}
   \begin{tikzpicture}
    \draw(0,0)node[below left]{$O$};
    \draw[->](-0.5,0)--(8,0)node[right]{$x$};
    \draw[->](0,-3)--(0,3)node[above]{$z$};
    \draw[domain=1.5:6.5]plot({\x}, {5*(1/(pi*(1+(\x-4)^2)))})node[right]{$z=p\left(x\right)$};
    \draw[dashed](2,{5*(1/(pi*(1+(2-4)^2)))})--(2,0)node[below]{$a_0$};
    \draw[dashed](2.2236,{5*(1/(pi*(1+(2.2236-4)^2)))})--(2.2236,-0.5)node[below]{$a_0+\Delta a_0$};
    \draw({(2+2.2236)/2},{5*(1/(pi*(1+(2-4)^2)))})node[above]{$\Delta s_0$};
    \draw[dashed](4,{5*(1/(pi*(1+(4-4)^2)))})--(4,0)node[below]{$a_1$};
    \draw[dashed](3.61019,{5*(1/(pi*(1+(3.61019-4)^2)))})--(3.61019,-1)node[below]{$a_1+\Delta a_1$};
    \draw({(4+3.61019)/2},{5*(1/(pi*(1+(4-4)^2)))})node[above]{$\Delta s_1$};
    \draw[dashed](6,{5*(1/(pi*(1+(6-4)^2)))})--(6,0)node[below]{$a_2$};
    \draw[dashed](6.16621,{5*(1/(pi*(1+(6.16621-4)^2)))})--(6.16621,-0.5)node[below]{$a_2+\Delta a_2$};
    \draw({(6+6.16621)/2},{5*(1/(pi*(1+(6-4)^2)))})node[above]{$\Delta s_2$};
   \end{tikzpicture}
   \caption{$b$の微小な変化による$Q\left(b\right)$の変化}\label{QChange}
  \end{center}
 \end{figure}
図\ref{QChange}において,領域$\{(x,y)|a_n\le x\le a_n+\Delta a_n\land \left(0-y\right)\left(p\left(x\right)-y\right)\le0\}$を台形とみなすと,その面積$\Delta s_n$は,
 \begin{align}
  \Delta s_n&=\frac{1}{2}\Delta a_n\left(p\left(a_n\right)+p\left(a_n+\Delta a_n\right)\right)\nonumber\\
  &=\frac{\Delta b}{2f'\left(a_n\right)}\left(p\left(a_n\right)+p\left(a_n+\frac{\Delta b}{f'\left(a_n\right)}\right)\right)
 \end{align}
となり,これを$\Delta b$で割ると,
 \begin{align}
  \frac{\Delta s_n}{\Delta b}&=\frac{p\left(a_n\right)+p\left(a_n+\frac{\Delta b}{f'\left(a_n\right)}\right)}{2f'\left(a_n\right)}\nonumber\\
  &=\frac{p\left(a_n\right)+p\left(a_n+\frac{0}{f'\left(a_n\right)}\right)}{2f'\left(a_n\right)}\nonumber\\
  &=\frac{p\left(a_n\right)+p\left(a_n\right)}{2f'\left(a_n\right)}\nonumber\\
  &=\frac{2p\left(a_n\right)}{2f'\left(a_n\right)}\nonumber\\
  &=\frac{p\left(a_n\right)}{f'\left(a_n\right)}
 \end{align}
となる.$q\left(b\right)$は,これを$\forall a_n\in\{a_0,\cdots,a_N\}=\{x\in S|f\left(x\right)=b\}$について足し合わせたものであるから,
 \begin{align}
  q\left(b\right)&=Q'\left(b\right)\nonumber\\
  &=\frac{d}{db}\int_{x\in S|f\left(x\right)<b}p\left(x\right)dx\nonumber\\
  &=\sum_{x\in S|f\left(x\right)=b}\frac{p\left(x\right)}{f'\left(x\right)}
 \end{align}
となる.
 \subsubsection{平均}
 \subsubsection{分散}
\end{document}

