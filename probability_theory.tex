\documentclass[dvipdfmx]{jsarticle}
\usepackage{amsfonts}
\usepackage{amsmath}
\usepackage{tikz}
\usetikzlibrary{intersections, calc, arrows.meta}
\title{確率論}
\author{伊藤 太清}
\date{\today}
\begin{document}
 \maketitle
 \section{離散確率変数}
$n$通りの場合$S=\{0,\cdots,n-1\}$の中からそれぞれの確率$p\left(0\right),\cdots,p\left(0\right)$で定まる変数がある.
 \begin{itemize}
  \item 全ての場合の集合$S=\{0,\cdots,n-1\}$を標本空間という.
  \item $p:S\to\left[0,1\right]$を確率質量関数という.
  \item $P:S\to\left[0,1\right];a\mapsto\sum_{x\in S|x<a}p\left(x\right)$を累積分布関数という.
 \end{itemize}
累積分布関数$P$が
 \begin{align}
P\left(n\right)=1
 \end{align}
を満たすとき,$X=\left(S,P\right)$を離散確率変数という.
累積分布関数$P$が確率質量関数$p$を用いて
 \begin{align}
P\left(a\right)=\sum_{x\in S|x<a}p\left(x\right)
 \end{align}
と表されるのに対し,確率質量関数$p$は累積分布関数$P$を用いて
 \begin{align}
p\left(a\right)&=\left(\sum_{x\in S|x<a+1}p\left(x\right)\right)-\left(\sum_{x\in S|x<a}p\left(x\right)\right)\nonumber\\
&=P\left(a+1\right)-P\left(a\right)
 \end{align}
と表される.
 \subsection{平均}
 \begin{align}
E\left(X\right)=\sum_{x\in S}xp\left(x\right)
 \end{align}
を,離散確率変数$X$の平均という.
 \subsection{分散}
 \begin{align}
V\left(X\right)=\sum_{x\in S}\left(x-E\left(X\right)\right)^2p\left(x\right)
 \end{align}
を,離散確率変数$X$の分散という.
 \subsection{関数の適用}
関数$f:S\to T$を離散確率変数$X=\left(S,P\right)$に適用した離散確率変数を$Y=\left(T,Q\right)$とし,その確率質量関数を$q$とすると,累積分布関数$Q$は,
 \begin{align}
Q\left(b\right)&=\sum_{y\in T|y<b}q\left(y\right)\nonumber\\
&=\sum_{x\in S|f\left(x\right)<b}p\left(x\right)
 \end{align}
であり,確率質量関数$q$は,
 \begin{align}
q\left(b\right)&=Q\left(b+1\right)-Q\left(b\right)\nonumber\\
&=\left(\sum_{x\in S|f\left(x\right)<b+1}p\left(x\right)\right)-\left(\sum_{x\in S|f\left(x\right)<b}p\left(x\right)\right)\nonumber\\
&=\sum_{x\in S|f\left(x\right)=b}p\left(x\right)
 \end{align}
となる.
 \section{連続確率変数}
実数全体の集合$\mathbb{R}$の中から確率的に実数が選択される変数がある.
 \begin{itemize}
  \item 選択されうる実数の集合$S=\mathbb{R}$を標本空間という.
  \item $p:S\to\left[0,\infty\right)$を確率密度関数という.
  \item $P:S\to\left[0,1\right];a\mapsto\int_{-\infty}^ap\left(x\right)dx$を累積分布関数という.
 \end{itemize}
累積分布関数$P$が
 \begin{align}
\lim_{a\to\infty}P\left(a\right)=1
 \end{align}
を満たすとき,$X=\left(S,P\right)$を連続確率変数という.
累積分布関数$P$が確率密度関数$p$を用いて
 \begin{align}
P\left(a\right)=\int_{-\infty}^ap\left(x\right)dx
 \end{align}
と表されるのに対し,確率密度関数$p$は累積分布関数$P$を用いて,
 \begin{align}
p\left(a\right)=P'\left(a\right)
 \end{align}
と表される.
 \subsection{平均}
 \begin{align}
E\left(X\right)=\int_{-\infty}^\infty xp\left(x\right)dx
 \end{align}
を,連続確率変数$X$の平均という.
 \subsection{分散}
 \begin{align}
V\left(X\right)=\int_{-\infty}^\infty \left(x-E\left(X\right)\right)^2p\left(x\right)dx
 \end{align}
を,連続確率変数$X$の分散という.
 \subsection{関数の適用}
関数$f:S\to T$を連続確率変数$X=\left(S,P\right)$に適用した連続確率変数を$Y=\left(T,Q\right)$とし,その確率密度関数を$q$とすると,累積分布関数$Q$は,
 \begin{align}
Q\left(b\right)&=\int_{-\infty}^bq\left(y\right)dy\nonumber\\
&=\int_{x\in S|f\left(x\right)<b}p\left(x\right)\frac{dy}{dx}dx\nonumber\\
&=\int_{x\in S|f\left(x\right)<b}p\left(x\right)f'\left(x\right)dx
 \end{align}
であり,確率密度関数$q$は,
 \begin{align}
q\left(b\right)&=Q'\left(b\right)\nonumber\\
&=\frac{d}{db}\int_{x\in S|f\left(x\right)<b}p\left(x\right)f'\left(x\right)dx
 \end{align}
となる.ここで,上の微分を下の図を用いて解く.
 \begin{figure}
  \begin{center}
   \begin{tikzpicture}
    \draw(0,0)node[below left]{$O$};
    \draw[->](-0.5,0)--(8,0)node[right]{$x$};
    \draw[->](0,-0.5)--(0,4)node[above]{$y$};
    \draw[domain=-2.5:2.5]plot({\x+4}, {(\x^3-4*\x)/3+2})node[right]{$y=f\left(x\right)$};
    \draw[dashed](2,2)--(2,0)node[below]{$a_0$};
    \draw[dashed](2.2236,2.5)--(2.2236,-0.5)node[below]{$a_0+\Delta a_0$};
    \draw[dashed](4,2)--(4,0)node[below]{$a_1$};
    \draw[dashed](3.61019,2.5)--(3.61019,-1)node[below]{$a_1+\Delta a_1$};
    \draw[dashed](6,2)--(6,0)node[below]{$a_2$};
    \draw[dashed](6.16621,2.5)--(6.16621,-0.5)node[below]{$a_2+\Delta a_2$};
    \draw[dashed](6,2)--(-0.5,2)node[left]{$b$};
    \draw[dashed](6.5,2.5)--(-0.5,2.5)node[left]{$b+\Delta b$};
   \end{tikzpicture}
   \caption{$b$の微小な変化による$\{x\in S|f\left(x\right)<b\}$の変化}
  \end{center}
 \end{figure}
 \begin{figure}
  \begin{center}
   \begin{tikzpicture}
    \draw(0,0)node[below left]{$O$};
    \draw[->](-0.5,0)--(8,0)node[right]{$x$};
    \draw[->](0,-3)--(0,3)node[above]{$z$};
    \draw[domain=1.5:6.5]plot({\x}, {5*(1/(pi*(1+(\x-4)^2)))*(\x^2-8*\x+44/3)})node[right]{$z=p\left(x\right)f'\left(x\right)$};
    \draw[dashed](2,{5*(1/(pi*(1+(2-4)^2)))*(2^2-8*2+44/3)})--(2,0)node[below]{$a_0$};
    \draw[dashed](2.2236,{5*(1/(pi*(1+(2.2236-4)^2)))*(2.2236^2-8*2.2236+44/3)})--(2.2236,-0.5)node[below]{$a_0+\Delta a_0$};
    \draw({(2+2.2236)/2},{5*(1/(pi*(1+(2-4)^2)))*(2^2-8*2+44/3)/2})node{$s_0$};
    \draw[dashed](4,{5*(1/(pi*(1+(4-4)^2)))*(4^2-8*4+44/3)})--(4,0)node[above]{$a_1$};
    \draw[dashed](3.61019,{5*(1/(pi*(1+(3.61019-4)^2)))*(3.61019^2-8*3.61019+44/3)})--(3.61019,0.5)node[above]{$a_1+\Delta a_1$};
    \draw({(4+3.61019)/2},{5*(1/(pi*(1+(4-4)^2)))*(4^2-8*4+44/3)/2})node{$s_1$};
    \draw[dashed](6,{5*(1/(pi*(1+(6-4)^2)))*(6^2-8*6+44/3)})--(6,0)node[below]{$a_2$};
    \draw[dashed](6.16621,{5*(1/(pi*(1+(6.16621-4)^2)))*(6.16621^2-8*6.16621+44/3)})--(6.16621,-0.5)node[below]{$a_2+\Delta a_2$};
    \draw({(6+6.16621)/2},{5*(1/(pi*(1+(6-4)^2)))*(6^2-8*6+44/3)/2})node{$s_2$};
   \end{tikzpicture}
   \caption{$b$の微小な変化による$Q\left(b\right)$の変化}
  \end{center}
 \end{figure}
\end{document}

