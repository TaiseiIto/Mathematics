\documentclass[dvipdfmx]{jsarticle}
\usepackage{amsfonts}
\usepackage{amsmath}
\usepackage{here}
\usepackage{mathrsfs}
\usepackage{tikz}
\usetikzlibrary{intersections, calc, arrows.meta}
\makeatletter
\newcommand{\subsubsubsection}{\@startsection{paragraph}{4}{\z@}%
 {1.0\Cvs \@plus.5\Cdp \@minus.2\Cdp}%
 {.1\Cvs \@plus.3\Cdp}%
 {\reset@font\sffamily\normalsize}
}
\makeatother
\setcounter{secnumdepth}{4}
\title{確率論}
\author{伊藤 太清}
\date{\today}
\begin{document}
 \maketitle
 \section{離散確率変数}
$N$通りの場合$S=\{0,\cdots,N-1\}$の中からそれぞれの確率$p\left(0\right),\cdots,p\left(N-1\right)$で定まる変数がある.
 \begin{itemize}
  \item 全ての場合の集合$S=\{0,\cdots,N-1\}$を標本空間という.
  \item $p:S\to\left[0,1\right]$を確率質量関数という.
  \item $P:S\to\left[0,1\right];a\mapsto\sum_{x\in S|x<a}p\left(x\right)$を累積分布関数という.
 \end{itemize}
累積分布関数$P$が
 \begin{align}
  P\left(N\right)=1
 \end{align}
を満たすとき,$X=\left(S,P\right)$を離散確率変数という.
累積分布関数$P$が確率質量関数$p$を用いて
 \begin{align}
  P\left(a\right)=\sum_{x\in S|x<a}p\left(x\right)
 \end{align}
と表されるのに対し,確率質量関数$p$は累積分布関数$P$を用いて
 \begin{align}
  p\left(a\right)&=\left(\sum_{x\in S|x<a+1}p\left(x\right)\right)-\left(\sum_{x\in S|x<a}p\left(x\right)\right)\nonumber\\
  &=P\left(a+1\right)-P\left(a\right)
 \end{align}
と表される.
 \subsection{確率}
離散確率変数$X$から取り出された値$a$が,$S$の部分集合$S'$の元である確率は,事象からその事象の確率への関数である確率関数$\mathscr{P}$を用いて,
 \begin{align}
  \mathscr{P}\left(a\in S'\right)&=\sum_{x\in S'}p\left(x\right)
 \end{align}
である.
また,累積分布関数$P$は,確率関数$\mathscr{P}$を用いて,
 \begin{align}
  P\left(x\right)&=\mathscr{P}\left(a<x\right)
 \end{align}
と表される.
 \subsection{平均}
 \begin{align}
  E\left(X\right)=\sum_{x\in S}xp\left(x\right)
 \end{align}
を,離散確率変数$X$の平均という.
 \subsection{分散}
 \begin{align}
  V\left(X\right)=\sum_{x\in S}\left(x-E\left(X\right)\right)^2p\left(x\right)
 \end{align}
を,離散確率変数$X$の分散という.
 \subsection{関数の適用}
関数$f:S\to T$を離散確率変数$X=\left(S,P\right)$に適用した離散確率変数を$f\left(X\right)=\left(T,Q\right)$とし,その確率質量関数を$q$とすると,累積分布関数$Q$は,
 \begin{align}
  Q\left(b\right)&=\mathscr{P}\left(y<b\right)\nonumber\\
  &=\mathscr{P}\left(f\left(x\right)<b\right)\nonumber\\
  &=\sum_{x\in S|f\left(x\right)<b}p\left(x\right)
 \end{align}
であり,確率質量関数$q$は,
 \begin{align}
  q\left(b\right)&=Q\left(b+1\right)-Q\left(b\right)\nonumber\\
  &=\left(\sum_{x\in S|f\left(x\right)<b+1}p\left(x\right)\right)-\left(\sum_{x\in S|f\left(x\right)<b}p\left(x\right)\right)\nonumber\\
  &=\sum_{x\in S|f\left(x\right)=b}p\left(x\right)
 \end{align}
となる.
 \subsubsection{平均}
離散確率変数$f\left(X\right)$の平均は,
 \begin{align}
  E\left(f\left(X\right)\right)&=\sum_{y\in T}yq\left(y\right)\nonumber\\
  &=\sum_{y\in T}y\sum_{x\in S|f\left(x\right)=y}p\left(x\right)\nonumber\\
  &=\sum_{y\in T}\sum_{x\in S|f\left(x\right)=y}yp\left(x\right)\nonumber\\
  &=\sum_{y\in T}\sum_{x\in S|f\left(x\right)=y}f\left(x\right)p\left(x\right)\nonumber\\
  &=\sum_{x\in S|f\left(x\right)\in T}f\left(x\right)p\left(x\right)\nonumber\\
  &=\sum_{x\in S}f\left(x\right)p\left(x\right)\label{DiscreteFunctionAverage}
 \end{align}
となる.
 \subsubsection{分散}
離散確率変数$f\left(X\right)$の分散は,
 \begin{align}
  V\left(f\left(X\right)\right)&=\sum{y\in T}\left(y-E\left(f\left(X\right)\right)\right)^2q\left(y\right)\nonumber\\
  &=\sum{y\in T}\left(y-E\left(f\left(X\right)\right)\right)^2\sum_{x\in S|f\left(x\right)=y}p\left(x\right)\nonumber\\
  &=\sum{y\in T}\sum_{x\in S|f\left(x\right)=y}\left(y-E\left(f\left(X\right)\right)\right)^2p\left(x\right)\nonumber\\
  &=\sum{y\in T}\sum_{x\in S|f\left(x\right)=y}\left(f\left(x\right)-E\left(f\left(X\right)\right)\right)^2p\left(x\right)\nonumber\\
  &=\sum_{x\in S|f\left(x\right)\in T}\left(f\left(x\right)-E\left(f\left(X\right)\right)\right)^2p\left(x\right)\nonumber\\
  &=\sum_{x\in S}\left(f\left(x\right)-E\left(f\left(X\right)\right)\right)^2p\left(x\right)
 \end{align}
となる.
 \section{連続確率変数}
実数全体の集合$\mathbb{R}$の中から確率的に実数が選択される変数がある.
 \begin{itemize}
  \item 選択されうる実数の集合$S=\mathbb{R}$を標本空間という.
  \item $p:S\to\left[0,\infty\right)$を確率密度関数という.
  \item $P:S\to\left[0,1\right];a\mapsto\int_{-\infty}^ap\left(x\right)dx$を累積分布関数という.
 \end{itemize}
累積分布関数$P$が
 \begin{align}
  \lim_{a\to\infty}P\left(a\right)=1
 \end{align}
を満たすとき,$X=\left(S,P\right)$を連続確率変数という.
累積分布関数$P$が確率密度関数$p$を用いて
 \begin{align}
  P\left(a\right)=\int_{-\infty}^ap\left(x\right)dx
 \end{align}
と表されるのに対し,確率密度関数$p$は累積分布関数$P$を用いて,
 \begin{align}
  p\left(a\right)=P'\left(a\right)
 \end{align}
と表される.
 \subsection{確率}
連続確率変数$X$から取り出された値$a$が,$S$の部分集合$S'$の元である確率は,事象からその事象の確率への関数である確率関数$\mathscr{P}$を用いて,
 \begin{align}
  \mathscr{P}\left(a\in S'\right)&=\int_{x\in S'}p\left(x\right)dx
 \end{align}
である.また,累積分布関数$P$は,確率関数$\mathscr{P}$を用いて,
 \begin{align}
  P\left(x\right)&=\mathscr{P}\left(a<x\right)
 \end{align}
である.
 \subsection{平均}
 \begin{align}
  E\left(X\right)=\int_{-\infty}^\infty xp\left(x\right)dx
 \end{align}
を,連続確率変数$X$の平均という.
 \subsection{分散}
 \begin{align}
  V\left(X\right)=\int_{-\infty}^\infty \left(x-E\left(X\right)\right)^2p\left(x\right)dx
 \end{align}
を,連続確率変数$X$の分散という.
 \subsection{関数の適用}
微分可能で$f'\left(x\right)=0$の解が高々有限個である関数$f:S\to T$を連続確率変数$X=\left(S,P\right)$に適用した連続確率変数を$f\left(X\right)=\left(T,Q\right)$とし,その確率密度関数を$q$とすると,$f'\left(b\right)=0$を満たすような$b$の範囲内において,累積分布関数$Q$は,
 \begin{align}
  Q\left(b\right)&=\mathscr{P}\left(y<b\right)\nonumber\\
  &=\mathscr{P}\left(f\left(x\right)<b\right)\nonumber\\
  &=\int_{x\in S|f\left(x\right)<b}p\left(x\right)dx
 \end{align}
であり,確率密度関数$q$は,
 \begin{align}
  q\left(b\right)&=Q'\left(b\right)\nonumber\\
  &=\frac{d}{db}\int_{x\in S|f\left(x\right)<b}p\left(x\right)dx
 \end{align}
となる.ここで,上の微分を図を用いて解く.
 \begin{figure}[H]
  \begin{center}
   \begin{tikzpicture}
    \draw(0,0)node[below left]{$O$};
    \draw[->](-0.5,0)--(8,0)node[right]{$x$};
    \draw[->](0,-0.5)--(0,4)node[above]{$y$};
    \draw[domain=-2.5:2.5]plot({\x+4}, {(\x^3-4*\x)/3+2})node[right]{$y=f\left(x\right)$};
    \draw[dashed](2,2)--(2,0)node[below]{$a_0$};
    \draw[dashed](2.2236,2.5)--(2.2236,-0.5)node[below]{$a_0+\Delta a_0$};
    \draw[dashed](4,2)--(4,0)node[below]{$a_1$};
    \draw[dashed](3.61019,2.5)--(3.61019,-1)node[below]{$a_1+\Delta a_1$};
    \draw[dashed](6,2)--(6,0)node[below]{$a_2$};
    \draw[dashed](6.16621,2.5)--(6.16621,-0.5)node[below]{$a_2+\Delta a_2$};
    \draw[dashed](6,2)--(-0.5,2)node[left]{$b$};
    \draw[dashed](6.5,2.5)--(-0.5,2.5)node[left]{$b+\Delta b$};
   \end{tikzpicture}
   \caption{$b$の微小な変化による$\{x\in S|f\left(x\right)<b\}$の変化}\label{IntegrationRangeChange}
  \end{center}
 \end{figure}
図\ref{IntegrationRangeChange}のように,$f\left(x\right)=b$の全ての解を$\{a_0,\cdots,a_N\}=\{x\in S|f\left(x\right)=b\}$とし,$b$の微小な変化$\Delta b$に対する$a_0,\cdots,a_N$の反応をそれぞれ$\Delta a_0,\cdots,\Delta a_N$とする.
ここで,$\Delta b$は十分に小さいから,$\forall n\in\{0,\cdots,N\}$に対して,
 \begin{align}
  f'\left(a_n\right)&=\frac{\Delta b}{\Delta a_n}\nonumber\\
  \Delta a_n&=\frac{\Delta b}{f'\left(a_n\right)}
 \end{align}
である.
ただしここで$f'\left(a_n\right)=0$の場合,それは$a_n$が$f$の極大若しくは極小であることにより,$b$の微小な変化によって$a_n$が消滅または2つに分裂することを意味し,$\Delta a_n$は定義できない.
 \begin{figure}[H]
  \begin{center}
   \begin{tikzpicture}
    \draw(0,0)node[below left]{$O$};
    \draw[->](-0.5,0)--(8,0)node[right]{$x$};
    \draw[->](0,-3)--(0,3)node[above]{$z$};
    \draw[domain=1.5:6.5]plot({\x}, {5*(1/(pi*(1+(\x-4)^2)))})node[right]{$z=p\left(x\right)$};
    \draw[dashed](2,{5*(1/(pi*(1+(2-4)^2)))})--(2,0)node[below]{$a_0$};
    \draw[dashed](2.2236,{5*(1/(pi*(1+(2.2236-4)^2)))})--(2.2236,-0.5)node[below]{$a_0+\Delta a_0$};
    \draw({(2+2.2236)/2},{5*(1/(pi*(1+(2-4)^2)))})node[above]{$\Delta s_0$};
    \draw[dashed](4,{5*(1/(pi*(1+(4-4)^2)))})--(4,0)node[below]{$a_1$};
    \draw[dashed](3.61019,{5*(1/(pi*(1+(3.61019-4)^2)))})--(3.61019,-1)node[below]{$a_1+\Delta a_1$};
    \draw({(4+3.61019)/2},{5*(1/(pi*(1+(4-4)^2)))})node[above]{$\Delta s_1$};
    \draw[dashed](6,{5*(1/(pi*(1+(6-4)^2)))})--(6,0)node[below]{$a_2$};
    \draw[dashed](6.16621,{5*(1/(pi*(1+(6.16621-4)^2)))})--(6.16621,-0.5)node[below]{$a_2+\Delta a_2$};
    \draw({(6+6.16621)/2},{5*(1/(pi*(1+(6-4)^2)))})node[above]{$\Delta s_2$};
   \end{tikzpicture}
   \caption{$b$の微小な変化による$Q\left(b\right)$の変化}\label{QChange}
  \end{center}
 \end{figure}
図\ref{QChange}において,領域$\{(x,y)|a_n\le x\le a_n+\Delta a_n\land \left(0-y\right)\left(p\left(x\right)-y\right)\le0\}$を台形とみなすと,その面積$\Delta s_n$は,
 \begin{align}
  \Delta s_n&=\frac{1}{2}\Delta a_n\left(p\left(a_n\right)+p\left(a_n+\Delta a_n\right)\right)\nonumber\\
  &=\frac{\Delta b}{2f'\left(a_n\right)}\left(p\left(a_n\right)+p\left(a_n+\frac{\Delta b}{f'\left(a_n\right)}\right)\right)
 \end{align}
となり,これを$\Delta b$で割ると,
 \begin{align}
  \frac{\Delta s_n}{\Delta b}&=\frac{p\left(a_n\right)+p\left(a_n+\frac{\Delta b}{f'\left(a_n\right)}\right)}{2f'\left(a_n\right)}\nonumber\\
  &=\frac{p\left(a_n\right)+p\left(a_n+\frac{0}{f'\left(a_n\right)}\right)}{2f'\left(a_n\right)}\nonumber\\
  &=\frac{p\left(a_n\right)+p\left(a_n\right)}{2f'\left(a_n\right)}\nonumber\\
  &=\frac{2p\left(a_n\right)}{2f'\left(a_n\right)}\nonumber\\
  &=\frac{p\left(a_n\right)}{f'\left(a_n\right)}
 \end{align}
となる.$q\left(b\right)$は,これを$\forall a_n\in\{a_0,\cdots,a_N\}=\{x\in S|f\left(x\right)=b\}$について足し合わせたものであるから,
 \begin{align}
  q\left(b\right)&=Q'\left(b\right)\nonumber\\
  &=\frac{d}{db}\int_{x\in S|f\left(x\right)<b}p\left(x\right)dx\nonumber\\
  &=\sum_{x\in S|f\left(x\right)=b}\frac{p\left(x\right)}{f'\left(x\right)}
 \end{align}
となる.
 \subsubsection{平均}
連続確率変数$f\left(X\right)$の平均は
 \begin{align}
  E\left(f\left(X\right)\right)&=\int_{-\infty}^\infty yq\left(y\right)dy\nonumber\\
  &=\int_{-\infty}^\infty y\sum_{x\in S|f\left(x\right)=y}\frac{p\left(x\right)}{f'\left(x\right)}dy\nonumber\\
  &=\int_{-\infty}^\infty\sum_{x\in S|f\left(x\right)=y}\frac{yp\left(x\right)}{f'\left(x\right)}dy\nonumber\\
  &=\int_{-\infty}^\infty\sum_{x\in S|f\left(x\right)=y}\frac{f\left(x\right)p\left(x\right)}{f'\left(x\right)}dy\nonumber\\
  &=\int_{-\infty}^\infty\frac{f\left(x\right)p\left(x\right)}{f'\left(x\right)}\frac{dy}{dx}dx\nonumber\\
  &=\int_{-\infty}^\infty\frac{f\left(x\right)p\left(x\right)}{f'\left(x\right)}f'\left(x\right)dx\nonumber\\
  &=\int_{-\infty}^\infty f\left(x\right)p\left(x\right)dx\label{ContinuousFunctionAverate}
 \end{align}
となる.
ここで,$f'\left(x\right)=0$となる$x$は高々有限個しかないため,この積分の結果には影響しない.
 \subsubsection{分散}
連続確率変数$f\left(X\right)$の分散は
 \begin{align}
  V\left(f\left(X\right)\right)&=\int_{-\infty}^\infty\left(y-E\left(f\left(X\right)\right)\right)^2q\left(y\right)dy\nonumber\\
  &=\int_{-\infty}^\infty\left(y-E\left(f\left(X\right)\right)\right)^2\sum_{x\in S|f\left(x\right)=y}\frac{p\left(x\right)}{f'\left(x\right)}dy\nonumber\\
  &=\int_{-\infty}^\infty\sum_{x\in S|f\left(x\right)=y}\left(y-E\left(f\left(X\right)\right)\right)^2\frac{p\left(x\right)}{f'\left(x\right)}dy\nonumber\\
  &=\int_{-\infty}^\infty\sum_{x\in S|f\left(x\right)=y}\left(f\left(x\right)-E\left(f\left(X\right)\right)\right)^2\frac{p\left(x\right)}{f'\left(x\right)}dy\nonumber\\
  &=\int_{-\infty}^\infty\left(f\left(x\right)-E\left(f\left(X\right)\right)\right)^2\frac{p\left(x\right)}{f'\left(x\right)}\frac{dy}{dx}dx\nonumber\\
  &=\int_{-\infty}^\infty\left(f\left(x\right)-E\left(f\left(X\right)\right)\right)^2\frac{p\left(x\right)}{f'\left(x\right)}f'\left(x\right)dx\nonumber\\
  &=\int_{-\infty}^\infty\left(f\left(x\right)-E\left(f\left(X\right)\right)\right)^2p\left(x\right)dx
 \end{align}
となる.
ここで,$f'\left(x\right)=0$となる$x$は高々有限個しかないため,この積分の結果には影響しない.
 \section{平均と分散の関係に関する定理}
確率変数$X$と,それに関数$f:\mathbb{R}\to\mathbb{R};x\mapsto x^2$を適用した確率変数$X^2$があるとき,$X$の平均$E\left(X\right)$と分散$V\left(X\right)$の間には,次の関係が成り立つ.
 \begin{align}
  V\left(X\right)&=E\left(X^2\right)-E\left(X\right)^2
 \end{align}
 \subsection{証明}
確率変数$X$が離散確率変数の場合と連続確率変数の場合に分けて証明する.
 \subsubsection{確率変数$X$が離散確率変数の場合}
 \begin{align}
  V\left(X\right)&=\sum_{x\in S}\left(x-E\left(X\right)\right)^2p\left(x\right)\nonumber\\
  &=\sum_{x\in S}\left(x^2-2xE\left(X\right)+E\left(X\right)^2\right)p\left(x\right)\nonumber\\
  &=\sum_{x\in S}\left(x^2p\left(x\right)-2xE\left(X\right)p\left(x\right)+E\left(X\right)^2p\left(x\right)\right)\nonumber\\
  &=\left(\sum_{x\in S}x^2p\left(x\right)\right)-\left(\sum_{x\in S}2xE\left(X\right)p\left(x\right)\right)+\left(\sum_{x\in S}E\left(X\right)^2p\left(x\right)\right)\nonumber\\
  &=\left(\sum_{x\in S}f\left(x\right)p\left(x\right)\right)-2E\left(X\right)\left(\sum_{x\in S}xp\left(x\right)\right)+E\left(X\right)^2\left(\sum_{x\in S}p\left(x\right)\right)\nonumber\\
  &=E\left(X^2\right)-2E\left(X\right)E\left(X\right)+E\left(X\right)^2\nonumber\\
  &=E\left(X^2\right)-2E\left(X\right)^2+E\left(X\right)^2\nonumber\\
  &=E\left(X^2\right)-E\left(X\right)^2
 \end{align}
 \subsubsection{確率変数$X$が連続確率変数の場合}
 \begin{align}
  V\left(X\right)&=\int_{-\infty}^\infty\left(x-E\left(X\right)\right)^2p\left(x\right)dx\nonumber\\
  &=\int_{-\infty}^\infty\left(x^2-2xE\left(X\right)+E\left(X\right)^2\right)p\left(x\right)dx\nonumber\\
  &=\int_{-\infty}^\infty\left(x^2p\left(x\right)-2xE\left(X\right)p\left(x\right)+E\left(X\right)^2p\left(x\right)\right)dx\nonumber\\
  &=\int_{-\infty}^\infty x^2p\left(x\right)dx-\int_{-\infty}^\infty 2xE\left(X\right)p\left(x\right)dx+\int_{-\infty}^\infty E\left(X\right)^2p\left(x\right)dx\nonumber\\
  &=\int_{-\infty}^\infty f\left(x\right)p\left(x\right)dx-2E\left(X\right)\int_{-\infty}^\infty xp\left(x\right)dx+E\left(X\right)^2\int_{-\infty}^\infty p\left(x\right)dx\nonumber\\
  &=E\left(X^2\right)-2E\left(X\right)E\left(X\right)+E\left(X\right)^2\nonumber\\
  &=E\left(X^2\right)-2E\left(X\right)^2+E\left(X\right)^2\nonumber\\
  &=E\left(X^2\right)-E\left(X\right)^2
 \end{align}
 \section{平均の線形性}
 \subsection{和の平均は平均の和}
ある2つの確率変数$X$と$Y$があるとき,
 \begin{align}
  E\left(X+Y\right)&=E\left(X\right)+E\left(Y\right)
 \end{align}
である.
 \subsubsection{証明}
 \subsubsubsection{離散確率変数同士の和の平均}
互いに独立とは限らない2つの離散確率変数$X$と$Y$があり,それぞれの標本空間を$S,T$とし,それらの確率質量変数が$p\left(x,y\right)$の形で与えられている.
この時,$X$単独の確率質量関数$p_X$は,
 \begin{align}
  p_X\left(x\right)&=\sum_{y\in T}p\left(x,y\right)
 \end{align}
であり,その平均$E\left(X\right)$は,
 \begin{align}
  E\left(X\right)&=\sum_{x\in S}xp_X\left(x\right)\nonumber\\
  &=\sum_{x\in S}x\sum_{y\in T}p\left(x,y\right)\nonumber\\
  &=\sum_{x\in S}\sum_{y\in T}xp\left(x,y\right)\nonumber\\
  &=\sum_{y\in T}\sum_{x\in S}xp\left(x,y\right)
 \end{align}
である.
同様に,$Y$単独の確率質量関数$p_Y$は,
 \begin{align}
  p_Y\left(y\right)&=\sum_{x\in S}p\left(x,y\right)
 \end{align}
であり,その平均$E\left(Y\right)$は,
 \begin{align}
  E\left(Y\right)&=\sum_{y\in T}yp_Y\left(y\right)\nonumber\\
  &=\sum_{y\in T}y\sum_{x\in S}p\left(x,y\right)\nonumber\\
  &=\sum_{y\in T}\sum_{x\in S}yp\left(x,y\right)
 \end{align}
である.
一方$X+Y$の平均$E\left(X+Y\right)$は,
 \begin{align}
  E\left(X+Y\right)&=\sum_{y\in T}\sum_{x\in S}\left(x+y\right)p\left(x,y\right)\nonumber\\
  &=\sum_{y\in T}\sum_{x\in S}\left(xp\left(x,y\right)+yp\left(x,y\right)\right)\nonumber\\
  &=\sum_{y\in T}\left(\left(\sum_{x\in S}xp\left(x,y\right)\right)+\left(\sum_{x\in S}yp\left(x,y\right)\right)\right)\nonumber\\
  &=\left(\sum_{y\in T}\sum_{x\in S}xp\left(x,y\right)\right)+\left(\sum_{y\in T}\sum_{x\in S}yp\left(x,y\right)\right)\nonumber\\
  &=E\left(X\right)+E\left(Y\right)
 \end{align}
である.
 \subsubsubsection{連続確率変数同士の和の平均}
互いに独立とは限らない2つの連続確率変数$X$と$Y$があり,それぞれの標本空間を$S,T$とし,それらの確率密度関数が$p\left(x,y\right)$の形で与えられている.
この時,$X$単独の確率密度関数$p_X$は,
 \begin{align}
  p_X\left(x\right)&=\int_{-\infty}^\infty p\left(x,y\right)dy
 \end{align}
であり,その平均$E\left(X\right)$は,
 \begin{align}
  E\left(X\right)&=\int_{-\infty}^\infty xp_X\left(x\right)dx\nonumber\\
  &=\int_{-\infty}^\infty x\int_{-\infty}^\infty p\left(x,y\right)dydx\nonumber\\
  &=\int_{-\infty}^\infty \int_{-\infty}^\infty xp\left(x,y\right)dydx\nonumber\\
  &=\int_{-\infty}^\infty \int_{-\infty}^\infty xp\left(x,y\right)dxdy\nonumber\\
 \end{align}
である.
同様に,$Y$単独の確率密度関数$p_Y$は,
 \begin{align}
  p_Y\left(y\right)&=\int_{-\infty}^\infty p\left(x,y\right)dx
 \end{align}
であり,その平均$E\left(Y\right)$は,
 \begin{align}
  E\left(Y\right)&=\int_{-\infty}^\infty yp_Y\left(y\right)dy\nonumber\\
  &=\int_{-\infty}^\infty y\int_{-\infty}^\infty p\left(x,y\right)dxdy\nonumber\\
  &=\int_{-\infty}^\infty \int_{-\infty}^\infty yp\left(x,y\right)dxdy\nonumber\\
 \end{align}
である.
一方$X+Y$の平均$E\left(X+Y\right)$は,
 \begin{align}
  E\left(X+Y\right)&=\int_{-\infty}^\infty\int_{-\infty}^\infty\left(x+y\right)p\left(x,y\right)dxdy\nonumber\\
  &=\int_{-\infty}^\infty\int_{-\infty}^\infty\left(xp\left(x,y\right)+yp\left(x,y\right)\right)dxdy\nonumber\\
  &=\int_{-\infty}^\infty\left(\int_{-\infty}^\infty xp\left(x,y\right)dx+\int_{-\infty}^\infty yp\left(x,y\right)dx\right)dy\nonumber\\
  &=\int_{-\infty}^\infty\int_{-\infty}^\infty xp\left(x,y\right)dxdy+\int_{-\infty}^\infty\int_{-\infty}^\infty yp\left(x,y\right)dxdy\nonumber\\
  &=E\left(X\right)+E\left(Y\right)
 \end{align}
 \subsubsubsection{離散確率変数と連続確率変数の和の平均}
互いに独立とは限らない離散確率変数$X$と連続確率変数$Y$があり,それぞれの標本空間を$S,T$とし,それらの確率密度関数が$p\left(x,y\right)$の形で与えられている.
この時,$X$単独の確率質量関数$p_X$は,
 \begin{align}
  p_X\left(x\right)&=\int_{-\infty}^\infty p\left(x,y\right)dy
 \end{align}
であり,その平均$E\left(X\right)$は,
 \begin{align}
  E\left(X\right)&=\sum_{x\in S}xp_X\left(x\right)\nonumber\\
  &=\sum_{x\in S}x\int_{-\infty}^\infty p\left(x,y\right)dy\nonumber\\
  &=\sum_{x\in S}\int_{-\infty}^\infty xp\left(x,y\right)dy\nonumber\\
  &=\int_{-\infty}^\infty\sum_{x\in S}xp\left(x,y\right)dy\nonumber\\
 \end{align}
である.
同様に,$Y$単独の確率密度関数$p_Y$は,
 \begin{align}
  p_Y\left(y\right)&=\sum_{x\in S}p\left(x,y\right)
 \end{align}
であり,その平均$E\left(Y\right)$は,
 \begin{align}
  E\left(Y\right)&=\int_{-\infty}^\infty yp_Y\left(y\right)dy\nonumber\\
  &=\int_{-\infty}^\infty y\sum_{x\in S}p\left(x,y\right)dy\nonumber\\
  &=\int_{-\infty}^\infty\sum_{x\in S}yp\left(x,y\right)dy\nonumber\\
 \end{align}
である.
一方$X+Y$の平均$E\left(X+Y\right)$は,
 \begin{align}
  E\left(X+Y\right)&=\int_{-\infty}^\infty\sum_{x\in S}\left(x+y\right)p\left(x,y\right)dy\nonumber\\
  &=\int_{-\infty}^\infty\sum_{x\in S}\left(xp\left(x,y\right)+yp\left(x,y\right)\right)dy\nonumber\\
  &=\int_{-\infty}^\infty\left(\sum_{x\in S}xp\left(x,y\right)+\infty\sum_{x\in S}yp\left(x,y\right)\right)dy\nonumber\\
  &=\int_{-\infty}^\infty\sum_{x\in S}xp\left(x,y\right)dy+\int_{-\infty}^\infty\infty\sum_{x\in S}yp\left(x,y\right)dy\nonumber\\
  &=E\left(X\right)+E\left(Y\right)
 \end{align}
である.
 \subsection{スカラー倍の平均は平均のスカラー倍}
ある確率変数$X$とある数$a$があるとき,
 \begin{align}
  E\left(aX\right)&=aE\left(X\right)
 \end{align}
である.
 \subsubsection{証明}
 \subsubsubsection{$X$が離散確率分布の場合}
式(\ref{DiscreteFunctionAverage})より,
 \begin{align}
  E\left(aX\right)&=\sum_{x\in S}axp\left(x\right)\nonumber\\
  &=a\sum_{x\in S}xp\left(x\right)\nonumber\\
  &=aE\left(aX\right)\nonumber\\
 \end{align}
 \subsubsubsection{$X$が連続確率分布の場合}
式(\ref{ContinuousFunctionAverate})より,
 \begin{align}
  E\left(aX\right)&=\int_{-\infty}^\infty axp\left(x\right)dx\nonumber\\
  &=a\int_{-\infty}^\infty xp\left(x\right)dx\nonumber\\
  &=aE\left(aX\right)
 \end{align}
 \section{独立な確率変数の積の平均}
互いに独立な2つの確率変数$X,Y$について,
 \begin{align}
  E\left(XY\right)&=E\left(X\right)E\left(Y\right)\label{ProductAverage}
 \end{align}
が成り立つ.
ただし,2つの確率変数が互いに独立であるということは,それらの組$(X,Y)$の確率質量関数若しくは確率密度関数を,それぞれの確率質量関数若しくは確率密度関数の積で表すことができることを意味する.
 \subsection{証明}
 \subsubsection{独立な離散確率変数同士の積の平均}
互いに独立な2つの離散確率変数$X,Y$があり,それぞれの標本空間を$S,T$とし,それぞれの確率質量関数を$p,q$とすると,
 \begin{align}
  E\left(XY\right)&=\sum_{x\in S}\sum_{y\in T}xyp\left(x\right)q\left(y\right)\nonumber\\
  &=\sum_{x\in S}xp\left(x\right)\sum_{y\in T}yq\left(y\right)\nonumber\\
  &=\left(\sum_{x\in S}xp\left(x\right)\right)\left(\sum_{y\in T}yq\left(y\right)\right)\nonumber\\
  &=E\left(X\right)E\left(Y\right)
 \end{align}
となる.
 \subsubsection{独立な連続確率変数同士の積の平均}
互いに独立な2つの連続確率変数$X,Y$があり,それぞれの標本空間を$S,T$とし,それぞれの確率質量関数を$p,q$とすると,
 \begin{align}
  E\left(XY\right)&=\int_{-\infty}^\infty\int_{-\infty}^\infty xyp\left(x\right)q\left(y\right)dydx\nonumber\\
  &=\int_{-\infty}^\infty xp\left(x\right)\int_{-\infty}^\infty yq\left(y\right)dydx\nonumber\\
  &=\int_{-\infty}^\infty xp\left(x\right)dx\int_{-\infty}^\infty yq\left(y\right)dy\nonumber\\
  &=E\left(X\right)E\left(Y\right)
 \end{align}
となる.
 \subsubsection{独立な離散確率変数と連続確率変数の積の平均}
互いに独立な離散確率変数$X$と連続確率変数$Y$があり,それぞれの標本空間を$S,T$とし,それぞれの確率質量関数を$p,q$とすると,
 \begin{align}
  E\left(XY\right)&=\sum_{x\in S}\int_{-\infty}^\infty xyp\left(x\right)q\left(y\right)dy\nonumber\\
  &=\sum_{x\in S}xp\left(x\right)\int_{-\infty}^\infty yq\left(y\right)dy\nonumber\\
  &=\left(\sum_{x\in S}xp\left(x\right)\right)\left(\int_{-\infty}^\infty yq\left(y\right)dy\right)\nonumber\\
  &=E\left(X\right)E\left(Y\right)
 \end{align}
となる.
 \section{確率母関数}
ある離散確率変数$X$に対し,その確率母関数$G_X$は,
 \begin{align}
  G_X\left(t\right)&=E\left(t^X\right)\nonumber\\
  &=\sum_{x\in S}t^xp\left(x\right)
 \end{align}
と定義される.
 \subsection{確率母関数と確率質量関数の関係}
確率母関数$G_X$と確率質量関数$p$の間には,以下の関係が成り立つ.
 \begin{align}
  \frac{G_X^{\left(a\right)}\left(0\right)}{a!}&=p\left(a\right)
 \end{align}
 \subsubsection{証明}
 \begin{align}
  G_X^{\left(a\right)}\left(t\right)&=\frac{d^a}{dt^a}E\left(t^X\right)\nonumber\\
  &=\frac{d^a}{dt^a}\sum_{x\in S}t^xp\left(x\right)\nonumber\\
  &=\sum_{x\in S}p\left(x\right)\frac{d^a}{dt^a}t^x\nonumber\\
  &=\sum_{x\in S|a\le x}p\left(x\right)\frac{d^a}{dt^a}t^x\nonumber\\
  &=p\left(a\right)\frac{d^a}{dt^a}t^a+\sum_{x\in S|a<x}p\left(x\right)\frac{d^a}{dt^a}t^x\nonumber\\
  &=p\left(a\right)\left(\prod_{k=0}^{a-1}\left(a-k\right)\right)+\sum_{x\in S|a<x}p\left(x\right)\left(\prod_{k=0}^{a-1}\left(x-k\right)\right)t^{x-a}\nonumber\\
  &=p\left(a\right)a!+\sum_{x\in S|a<x}p\left(x\right)\frac{x!}{\left(x-a\right)!}t^{x-a}\nonumber\\
  G_X^{\left(a\right)}\left(0\right)&=p\left(a\right)a!+\sum_{x\in S|a<x}p\left(x\right)\frac{x!}{\left(x-a\right)!}0^{x-a}\nonumber\\
  G_X^{\left(a\right)}\left(0\right)&=p\left(a\right)a!\nonumber\\
  \frac{G_X^{\left(a\right)}\left(0\right)}{a!}&=p\left(a\right)\nonumber\\
 \end{align}
 \subsection{確率母関数と階乗積率の関係}
確率母関数$G_X$と$n$次階乗積率$E\left(\frac{X!}{\left(X-n\right)!}\right)$の間には,以下の関係が成り立つ.
 \begin{align}
  G_X^{\left(n\right)}\left(1\right)&=E\left(\frac{X!}{\left(X-n\right)!}\right)
 \end{align}
 \subsubsection{証明}
 \begin{align}
  G_X^{\left(n\right)}\left(t\right)&=\frac{d^n}{dt^n}E\left(t^X\right)\nonumber\\
  &=\frac{d^n}{dt^n}\sum_{x\in S}t^xp\left(x\right)\nonumber\\
  &=\frac{d^n}{dt^n}\sum_{x\in S}p\left(x\right)t^x\nonumber\\
  &=\frac{d^n}{dt^n}\sum_{x\in S}p\left(x\right)\sum_{k=0}^\infty\frac{1}{k!}\left(\left.\frac{d^k}{dt^k}t^x\right|_{t=1}\right)\left(t-1\right)^k\nonumber\\
  &=\frac{d^n}{dt^n}\sum_{x\in S}p\left(x\right)\sum_{k=0}^x\frac{1}{k!}\left(\left.\left(\prod_{l=0}^{k-1}\left(x-l\right)\right)t^{x-k}\right|_{t=1}\right)\left(t-1\right)^k\nonumber\\
  &=\frac{d^n}{dt^n}\sum_{x\in S}p\left(x\right)\sum_{k=0}^x\frac{1}{k!}\left(\prod_{l=0}^{k-1}\left(x-l\right)\right)1^{x-k}\left(t-1\right)^k\nonumber\\
  &=\frac{d^n}{dt^n}\sum_{x\in S}p\left(x\right)\sum_{k=0}^x\frac{1}{k!}\frac{x!}{\left(x-k\right)!}\left(t-1\right)^k\nonumber\\
  &=\frac{d^n}{dt^n}\sum_{x\in S}p\left(x\right)\sum_{k=0}^x{}_xC_k\left(t-1\right)^k\nonumber\\
  &=\frac{d^n}{dt^n}\sum_{x\in S}\sum_{k=0}^x{}_xC_kp\left(x\right)\left(t-1\right)^k\nonumber\\
  &=\sum_{x\in S}\sum_{k=0}^x{}_xC_kp\left(x\right)\frac{d^n}{dt^n}\left(t-1\right)^k\nonumber\\
  &=\sum_{x\in S}\sum_{k=n}^x{}_xC_kp\left(x\right)\frac{d^n}{dt^n}\left(t-1\right)^k\nonumber\\
  &=\sum_{x\in S}\left({}_xC_np\left(x\right)\frac{d^n}{dt^n}\left(t-1\right)^n+\sum_{k=n+1}^x{}_xC_kp\left(x\right)\frac{d^n}{dt^n}\left(t-1\right)^k\right)\nonumber\\
  &=\sum_{x\in S}\left({}_xC_np\left(x\right)\prod_{l=0}^{n-1}\left(n-l\right)+\sum_{k=n+1}^x{}_xC_kp\left(x\right)\left(\prod_{l=0}^{n-1}\left(k-l\right)\right)\left(t-1\right)^{k-n}\right)\nonumber\\
  &=\sum_{x\in S}\left({}_xC_np\left(x\right)n!+\sum_{k=n+1}^x{}_xC_kp\left(x\right)\frac{k!}{\left(k-n\right)!}\left(t-1\right)^{k-n}\right)\nonumber\\
  &=\sum_{x\in S}\left(\frac{x!}{n!\left(x-n\right)!}p\left(x\right)n!+\sum_{k=n+1}^x\frac{x!}{k!\left(x-k\right)!}p\left(x\right)\frac{k!}{\left(k-n\right)!}\left(t-1\right)^{k-n}\right)\nonumber\\
  &=\sum_{x\in S}\left(\frac{x!}{\left(x-n\right)!}p\left(x\right)+\sum_{k=n+1}^x\frac{x!}{\left(x-k\right)!\left(k-n\right)!}p\left(x\right)\left(t-1\right)^{k-n}\right)\nonumber\\
  G_X^{\left(n\right)}\left(1\right)&=\sum_{x\in S}\left(\frac{x!}{\left(x-n\right)!}p\left(x\right)+\sum_{k=n+1}^x\frac{x!}{\left(x-k\right)!\left(k-n\right)!}p\left(x\right)\left(1-1\right)^{k-n}\right)\nonumber\\
  &=\sum_{x\in S}\frac{x!}{\left(x-n\right)!}p\left(x\right)\nonumber\\
  &=E\left(\frac{X!}{\left(X-n\right)!}\right)
 \end{align}
 \section{積率母関数}
ある確率変数$X$に対し,その積率母関数$M_X$は,
 \begin{align}
  M_X\left(t\right)&=E\left(e^{tX}\right)
 \end{align}
と定義される.
 \subsection{積率母関数と原点積率の関係}
 \begin{align}
  M_X^{\left(n\right)}\left(0\right)&=E\left(X^n\right)
 \end{align}
 \subsubsection{証明}
 \begin{align}
  M_X^{\left(n\right)}\left(t\right)&=\frac{d^n}{dt^n}E\left(e^{tX}\right)\nonumber\\
  &=\frac{d^n}{dt^n}E\left(\sum_{k=0}^\infty \frac{X^k}{k!}t^k\right)\nonumber\\
  &=\frac{d^n}{dt^n}\sum_{k=0}^\infty \frac{E\left(X^k\right)}{k!}t^k\nonumber\\
  &=\sum_{k=0}^\infty \frac{E\left(X^k\right)}{k!}\frac{d^n}{dt^n}t^k\nonumber\\
  &=\sum_{k=n}^\infty \frac{E\left(X^k\right)}{k!}\frac{d^n}{dt^n}t^k\nonumber\\
  &=\frac{E\left(X^n\right)}{n!}\frac{d^n}{dt^n}t^n+\sum_{k=n+1}^\infty \frac{E\left(X^k\right)}{k!}\frac{d^n}{dt^n}t^k\nonumber\\
  &=\frac{E\left(X^n\right)}{n!}n!+\sum_{k=n+1}^\infty \frac{E\left(X^k\right)}{k!}\frac{k!}{\left(k-n\right)!}t^{k-n}\nonumber\\
  &=E\left(X^n\right)+\sum_{k=n+1}^\infty \frac{E\left(X^k\right)}{\left(k-n\right)!}t^{k-n}\nonumber\\
  &=E\left(X^n\right)+\sum_{k=1}^\infty \frac{E\left(X^{k+n}\right)}{k!}t^k\nonumber\\
  M_X^{\left(n\right)}\left(0\right)&=E\left(X^n\right)+\sum_{k=1}^\infty \frac{E\left(X^{k+n}\right)}{k!}0^k\nonumber\\
  &=E\left(X^n\right)
 \end{align}
 \subsubsection{平均と分散への応用}
この定理より,確率変数$X$の平均と分散はそれぞれ
 \begin{align}
  E\left(X\right)&=M_X^{\left(1\right)}\left(0\right)\\
  V\left(X\right)&=E\left(X^2\right)-E\left(X\right)^2\nonumber\\
  &=M_X^{\left(2\right)}\left(0\right)-M_X^{\left(1\right)}\left(0\right)^2
 \end{align}
と表される.
 \subsection{確率変数の一次式の積率母関数}
ある確率変数$X$の一次式$aX+b$の積率母関数$M_{aX+b}$は,
 \begin{align}
  M_{aX+b}\left(t\right)&=e^{bt}M_X\left(at\right)
 \end{align}
である.
 \subsubsection{証明}
 \begin{align}
  M_{aX+b}\left(t\right)&=E\left(e^{t\left(aX+b\right)}\right)\nonumber\\
  &=E\left(e^{atX+bt}\right)\nonumber\\
  &=E\left(e^{atX}e^{bt}\right)\nonumber\\
  &=e^{bt}E\left(e^{atX}\right)\nonumber\\
  &=e^{bt}M_X\left(at\right)\nonumber\\
 \end{align}
 \subsection{互いに独立な確率変数の和の積率母関数}
互いに独立な$N$個の確率変数$X_0,\cdots,X_{N-1}$の和$X=\sum_{n=0}^{N-1}X_n$の積率母関数$M_X$は,
 \begin{align}
  M_X\left(t\right)&=\prod_{n=0}^{N-1}M_{X_n}\left(t\right)
 \end{align}
である.
 \subsubsection{証明}
式(\ref{ProductAverage})を用いて証明する.
 \begin{align}
  M_X\left(t\right)&=E\left(e^{tX}\right)\nonumber\\
  &=E\left(\exp\left(tX\right)\right)\nonumber\\
  &=E\left(\exp\left(t\sum_{n=0}^{N-1}X_n\right)\right)\nonumber\\
  &=E\left(\exp\left(\sum_{n=0}^{N-1}tX_n\right)\right)\nonumber\\
  &=E\left(\prod_{n=0}^{N-1}\exp\left(tX_n\right)\right)\nonumber
 \end{align}
ここで,$X_0,\cdots,X_{N-1}$は互いに独立だから式(\ref{ProductAverage})により,
 \begin{align}
  M_X\left(t\right)&=\prod_{n=0}^{N-1}E\left(\exp\left(tX_n\right)\right)\nonumber\\
  &=\prod_{n=0}^{N-1}M_{X_n}\left(t\right)
 \end{align}
 \section{特性関数}
確率変数$X$の特性関数$\phi_X$は,
 \begin{align}
  \phi_X\left(t\right)&=E\left(e^{itX}\right)
 \end{align}
で定義される.
 \subsection{特性関数と原点積率の関係}
 \begin{align}
  \left(-i\right)^n\phi_X^{\left(n\right)}\left(0\right)&=E\left(X^n\right)
 \end{align}
 \subsubsection{証明}
 \begin{align}
  \left(-i\right)^n\phi_X^{\left(n\right)}\left(t\right)&=\left(-i\right)^n\frac{d^n}{dt^n}E\left(e^{itX}\right)\nonumber\\
  &=\left(-i\right)^n\frac{d^n}{dt^n}E\left(\sum_{k=0}^\infty\frac{1}{k!}\left(\left.\frac{d^k}{dt^k}e^{itX}\right|_{t=0}\right)t^k\right)\nonumber\\
  &=\left(-i\right)^n\frac{d^n}{dt^n}E\left(\sum_{k=0}^\infty\frac{1}{k!}\left(\left.\left(iX\right)^ke^{itX}\right|_{t=0}\right)t^k\right)\nonumber\\
  &=\left(-i\right)^n\frac{d^n}{dt^n}E\left(\sum_{k=0}^\infty\frac{1}{k!}\left(iX\right)^ke^{i0X}t^k\right)\nonumber\\
  &=\left(-i\right)^n\frac{d^n}{dt^n}E\left(\sum_{k=0}^\infty\frac{1}{k!}\left(iX\right)^ke^0t^k\right)\nonumber\\
  &=\left(-i\right)^n\frac{d^n}{dt^n}\sum_{k=0}^\infty\frac{1}{k!}i^kE\left(X^k\right)t^k\nonumber\\
  &=\left(-i\right)^n\sum_{k=0}^\infty\frac{1}{k!}i^kE\left(X^k\right)\frac{d^n}{dt^n}t^k\nonumber\\
  &=\sum_{k=0}^\infty\frac{i^{k-n}}{k!}E\left(X^k\right)\frac{d^n}{dt^n}t^k\nonumber\\
  &=\sum_{k=n}^\infty\frac{i^{k-n}}{k!}E\left(X^k\right)\frac{d^n}{dt^n}t^k\nonumber\\
  &=\frac{i^{n-n}}{n!}E\left(X^n\right)\frac{d^n}{dt^n}t^n+\sum_{k=n+1}^\infty\frac{i^{k-n}}{k!}E\left(X^k\right)\frac{d^n}{dt^n}t^k\nonumber\\
  &=\frac{i^0}{n!}E\left(X^n\right)n!+\sum_{k=n+1}^\infty\frac{i^{k-n}}{k!}E\left(X^k\right)\frac{k!}{\left(k-n\right)!}t^{k-n}\nonumber\\
  &=E\left(X^n\right)+\sum_{k=n+1}^\infty\frac{i^{k-n}}{\left(k-n\right)!}E\left(X^k\right)t^{k-n}\nonumber\\
  \left(-i\right)^n\phi_X^{\left(n\right)}\left(0\right)&=E\left(X^n\right)+\sum_{k=n+1}^\infty\frac{i^{k-n}}{\left(k-n\right)!}E\left(X^k\right)0^{k-n}\nonumber\\
  &=E\left(X^n\right)
 \end{align}
 \subsection{確率変数の一次式の特性関数}
確率変数$X$の一次式$aX+b$の特性関数$\phi_{aX+b}$は,
 \begin{align}
  \phi_{aX+b}\left(t\right)&=e^{ibt}\phi_{X}\left(at\right)
 \end{align}
である.
 \subsubsection{証明}
 \begin{align}
  \phi_{aX+b}\left(t\right)&=E\left(e^{it\left(aX+b\right)}\right)\nonumber\\
  &=E\left(e^{iatX+ibt}\right)\nonumber\\
  &=E\left(e^{iatX}e^{ibt}\right)\nonumber\\
  &=e^{ibt}E\left(e^{iatX}\right)\nonumber\\
  &=e^{ibt}\phi_{X}\left(at\right)
 \end{align}
 \subsection{互いに独立な確率変数の和の特性関数}
互いに独立な$N$個の確率変数$X_0,\cdots,X_{N-1}$の和$X=\sum_{n=0}^{N-1}X_n$の特性関数$\phi_X$は,
 \begin{align}
  \phi_X\left(t\right)&=\prod_{n=0}^{N-1}\phi_{X_n}\left(t\right)
 \end{align}
である.
 \subsubsection{証明}
式(\ref{ProductAverage})を用いて証明する.
 \begin{align}
  \phi_X\left(t\right)&=E\left(\exp\left(itX\right)\right)\nonumber\\
  &=E\left(\exp\left(it\sum_{n=0}^{N-1}X_n\right)\right)\nonumber\\
  &=E\left(\exp\left(\sum_{n=0}^{N-1}itX_n\right)\right)\nonumber\\
  &=E\left(\prod_{n=0}^{N-1}\exp\left(itX_n\right)\right)\nonumber
 \end{align}
ここで,$X_0,\cdots,X_{N-1}$は互いに独立だから式(\ref{ProductAverage})により,
 \begin{align}
  \phi_X\left(t\right)&=\prod_{n=0}^{N-1}E\left(\exp\left(itX_n\right)\right)\nonumber\\
  &=\prod_{n=0}^{N-1}\phi_{X_n}\left(t\right)
 \end{align}
 \section{キュムラント母関数}
確率変数$X$のキュムラント母関数$K_X$は,
 \begin{align}
  K_X\left(t\right)&=\log M_X\left(t\right)
 \end{align}
と定義される.
$K_X$のマクローリン展開
 \begin{align}
  K_X\left(t\right)&=\sum_{k=0}^\infty \frac{K_X^{\left(k\right)}\left(0\right)t^k}{k!}
 \end{align}
における$K_X^{\left(k\right)}\left(0\right)$を確率変数$X$の$k$次キュムラントという.
 \subsection{キュムラントと積率の関係}
キュムラント母関数の定義より,
 \begin{align}
  \log M_X\left(t\right)&=K_X\left(t\right)\nonumber\\
  M_X\left(t\right)&=\exp K_X\left(t\right)\nonumber\\
  \sum_{k=0}^\infty \frac{M_X^{\left(k\right)}\left(0\right)t^k}{k!}&=\exp\sum_{k=0}^\infty \frac{K_X^{\left(k\right)}\left(0\right)t^k}{k!}\nonumber\\
  \frac{d}{dK_X^{\left(n\right)}\left(0\right)}\sum_{k=0}^\infty \frac{M_X^{\left(k\right)}\left(0\right)t^k}{k!}&=\frac{d}{dK_X^{\left(n\right)}\left(0\right)}\exp\sum_{k=0}^\infty \frac{K_X^{\left(k\right)}\left(0\right)t^k}{k!}\nonumber\\
  \sum_{k=0}^\infty \frac{t^k}{k!}\frac{dM_X^{\left(k\right)}\left(0\right)}{dK_X^{\left(n\right)}\left(0\right)}&=\left(\frac{d}{dK_X^{\left(n\right)}\left(0\right)}\sum_{k=0}^\infty\frac{K_X^{\left(k\right)}\left(0\right)t^k}{k!}\right)\left(\exp\sum_{k=0}^\infty\frac{K_X^{\left(k\right)}\left(0\right)t^k}{k!}\right)\nonumber\\
  &=\left(\sum_{k=0}^\infty\frac{t^k}{k!}\frac{dK_X^{\left(k\right)}\left(0\right)}{dK_X^{\left(n\right)}\left(0\right)}\right)\left(\exp\sum_{k=0}^\infty\frac{K_X^{\left(k\right)}\left(0\right)t^k}{k!}\right)\nonumber\\
  &=\left(\sum_{k=0}^\infty\frac{t^k}{k!}\frac{dK_X^{\left(k\right)}\left(0\right)}{dK_X^{\left(n\right)}\left(0\right)}\right)\exp K_X\left(t\right)\nonumber\\
  &=\left(\sum_{k=0}^\infty\frac{t^k}{k!}\frac{dK_X^{\left(k\right)}\left(0\right)}{dK_X^{\left(n\right)}\left(0\right)}\right)M_X\left(t\right)\nonumber\\
  &=\left(\sum_{k=0}^\infty\frac{t^k}{k!}\frac{dK_X^{\left(k\right)}\left(0\right)}{dK_X^{\left(n\right)}\left(0\right)}\right)\left(\sum_{k=0}^\infty\frac{M_X^{\left(k\right)}\left(0\right)t^k}{k!}\right)\nonumber
 \end{align}
ここで,$K_X^{\left(k\right)}\left(0\right)$はキュムラント母関数のマクローリン展開による各項の係数である.
マクローリン展開は任意の多項式を表現できるため,各行の係数は互いに独立なパラメータであると言える.
よって,$k \ne n$のとき,$K_X^{\left(n\right)}\left(0\right)$から見て$K_X^{\left(k\right)}\left(0\right)$は定数とみなすことができるため,
 \begin{align}
  \sum_{k=0}^\infty\frac{t^k}{k!}\frac{dM_X^{\left(k\right)}\left(0\right)}{dK_X^{\left(n\right)}\left(0\right)}&=\left(\frac{t^n}{n!}\frac{dK_X^{\left(n\right)}\left(0\right)}{dK_X^{\left(n\right)}\left(0\right)}\right)\left(\sum_{k=0}^\infty\frac{M_X^{\left(k\right)}\left(0\right)t^k}{k!}\right)\nonumber\\
  &=\frac{t^n}{n!}\sum_{k=0}^\infty\frac{M_X^{\left(k\right)}\left(0\right)t^k}{k!}\nonumber\\
  &=\sum_{k=0}^\infty\frac{M_X^{\left(k\right)}\left(0\right)t^{k+n}}{k!n!}\nonumber\\
  &=\sum_{k=n}^\infty\frac{M_X^{\left(k-n\right)}\left(0\right)t^k}{\left(k-n\right)!n!}\nonumber\\
  \sum_{k=0}^\infty\frac{dM_X^{\left(k\right)}\left(0\right)}{dK_X^{\left(n\right)}\left(0\right)}t^k&=\sum_{k=n}^\infty\frac{k!}{\left(k-n\right)!n!}M_X^{\left(k-n\right)}\left(0\right)t^k\nonumber\\
 \end{align}
それぞれの項の係数を比較すると,
\end{document}

